\documentclass[letterpaper, 12pt]{report}
\usepackage[spanish]{babel}
\usepackage[latin1]{inputenc}
\usepackage[pdftex]{color, graphicx}
\usepackage{multirow}
\usepackage{float}			%Cuadro de texto
%\pagestyle{empty}			%Estilo de pagina plain para que no muestre las cabeceras
%\textheight=21cm			%Establece el alto de texto a 21cm por defecto es 19cm
					%El ancho de texto por defecto es 14cm
\addtolength{\hoffset}{-0.5cm}
\addtolength{\textwidth}{1cm}
\addtolength{\textheight}{1cm}
\floatstyle{boxed}			%Cuadro de texto
\newfloat{codigof}{tbp}{c}[section]	%Cuadro de texto
\floatname{codigof}{C\'odigo}		%Cuadro de texto

\setlength{\parindent}{0pt}		%Quitar sangria primera linea de cada parrafo

%\setcounter{secnumdepth}{4}  		%=> Numera paragraph
%\setcounter{tocdepth}{4} 		%=> Lo pone en la tabla de contenido.

%\includeonly{implementacion}

\begin{document}
\pagestyle{empty}
\begin{center}
\textbf{TariyKDD: Una herramienta gen\'erica para el Descubrimiento de Conocimiento en Bases de Datos, d\'ebilmente acoplada con el SGBD PostgreSQL.}\end{center}
\vspace*{5cm}
\begin{center}
\textbf{Andr\'es Oswaldo Calderon Romero \\
Iv\'an Dario Ramirez Freyre \\
Alvaro Fernando Guevara Unigarro \\
Juan Carlos Alvarado Perez}\end{center}
\vspace*{6cm}
\begin{center}
\textbf{Universidad de Nari\~no \\ 
Facultad de Ingenier\'ia\\
Programa de Ingenier\'ia de Sistemas\\
San Juan de Pasto\\
2006}\end{center}



\pagestyle{empty}
\begin{center}
\textbf{TariyKDD: Una herramienta gen\'erica para el Descubrimiento de Conocimiento en Bases de Datos, d\'ebilmente acoplada con el SGBD PostgreSQL.}\end{center}
\vspace*{3cm}

\begin{center}
\textbf{Andr\'es Oswaldo Calderon Romero \\
Iv\'an Dario Ramirez Freyre \\
Alvaro Fernando Guevara Unigarro \\
Juan Carlos Alvarado Perez}\end{center}
\vspace*{2cm}

\begin{center}
\textbf{Trabajo de Grado presentado como requisito\\
para optar la t\'itulo de Ingenieros de Sistemas}
\end{center}
\vspace*{2cm}

\begin{center}
\textbf{Ricardo Timaran Pereira\\
Ph.D\\
Director del Proyecto}
\end{center}
\vspace*{2cm}

\begin{center}
\textbf{Universidad de Nari\~no \\ 
Facultad de Ingenier\'ia\\
Programa de Ingenier\'ia de Sistemas\\
San Juan de Pasto\\
2006}\end{center}

\tableofcontents
\pagestyle{plain}
\setcounter{page}{5}
\chapter{INTRODUCCI\'ON}
El proceso de extraer conocimiento a partir de grandes vol\'umenes de datos ha sido reconocido por muchos
investigadores como un t\'opico de investigaci\'on clave en los sistemas de bases de datos, y por muchas compa\~n\'ias industriales como una importante \'area y una oportunidad para obtener mayo\-res ganancias \cite{31}.\\
\\
El Descubrimiento de Conocimiento en Bases de Datos (DCBD) es b\'asicamente un proceso autom\'atico en el que se combinan descubrimiento y an\'alisis. El proceso consiste en extraer patrones en forma de reglas o funciones, a partir de los datos, para que el usuario los analice. Esta tarea implica generalmente preprocesar los datos, hacer miner\'ia de datos (data mining) y presentar resultados \cite{1,4,7,19,23}. El DCBD se puede aplicar en diferentes dominios por ejemplo, para determinar perfiles de clientes fraudulentos (evasi\'on de impuestos), para descubrir relaciones impl\'icitas existentes entre s\'intomas y enfermedades, entre caracter\'isticas t\'ecnicas y diagn\'ostico del estado de equipos y m\'aquinas, para determinar perfiles de estudiantes ''acad\'emicamente exitosos'' en t\'erminos de sus caracter\'isticas socioecon\'omicas, para determinar patrones de compra de los clientes en sus canastas de mercado, entre otras.\\
\\
Las investigaciones en DCBD, se centraron inicialmente en definir nuevas operaciones de descubrimiento de patrones y desarrollar algoritmos para estas. Investigaciones posteriores se han focalizado en el problema de integrar DCBD con Sistemas Gestores de  Bases de Datos (SGBD) ofreciendo como resultado el desarrollo de herramientas  DCBD cuyas arquitecturas se pueden clasificar en una de tres categor\'ias: d\'ebilmente acopladas, medianamente acopladas y fuertemente acopladas con el SGBD \cite{30}.\\
\\
Una herramienta DCBD debe integrar una variedad de componentes (t\'ecnicas de miner\'ia de datos, consultas, m\'etodos de visualizaci\'on, interfaces, etc.), que juntos puedan eficientemente identificar y extraer patrones interesantes y \'utiles de los datos almacenados en las bases de datos.  De acuerdo a las tareas que desarrollen, las herramientas DCBD se clasifican en tres grupos: herramientas gen\'ericas de tareas sencillas, herramientas gen\'ericas de tareas m\'ultiples y herramientas de dominio espec\'ifico \cite{23}.\\
\\
En este documento se presenta el trabajo de grado para optar por el t\'itulo de Ingeniero de Sistemas. Fruto de la presente investigaci\'on es el desarrollo de ''TariyKDD: Una herramienta gen\'erica de Descubrimiento de Conocimiento en Bases de Datos d\'ebilmente acoplada con el SGBD PostgreSQL'',  en la cual tambi\'en se implementaron los algoritmos EquipAsso \cite{33, 34, 35} y MateTree \cite{35} para las tareas de Asociaci\'on y Clasificaci\'on propuestos por Timar\'an en \cite{35} y sobre los cuales se realizar\'on ciertas pruebas para medir su rendimiento.\\
\\
El resto de este documento esta organizado de la siguiente manera. En la siguiente secci\'on se especifica el Tema de la propuesta, se lo enmarca dentro de una l\'inea de investigaci\'on y se lo delimita. A continuaci\'on se describe el problema objeto de estudio. En la secci\'on 4  se especifican los objetivos generales y espec\'ificos del anteproyecto. En la secci\'on 5 se presenta la justificaci\'on de la propuesta de trabajo de grado. En la secci\'on 6 se presenta el  estado general del arte en el \'area de integraci\'on de DCBD y SGBD. En la secci\'on 7 se desarrolla todo lo concerniente al an\'alisis orientado a objetos UML que se realizo para construir la herramienta y finalmente en la secci\'on 8 se presentan las conclusiones, recomendaciones, referencias bibliogr\'aficas y anexos.

\include{tema}
\chapter{PROBLEMA OBJETO DE ESTUDIO}
\section{Descripci\'on del Problema}
Muchos investigadores \cite{7, 6, 17, 28} han reconocido la necesidad de integrar los sistemas de descubrimiento de conocimiento y bases de datos, haciendo de esta un \'area activa de investigaci\'on. La gran mayor\'ia de herramientas de DCBD tienen una arquitectura d\'ebilmente acoplada con un Sistema Gestor de Bases de Datos \cite{30}.\\
\\
Algunas herramientas como Alice \cite{20}, C5.0 RuleQuest \cite{27}, Qyield \cite{24},  Cover\-Story \cite{22} ofrecen soporte \'unicamente en la etapa de miner\'ia de datos y requieren un pre y un post procesamiento de los datos. Hay una gran cantidad de este tipo de herramientas \cite{21}, especialmente para clasificaci\'on apoyadas en \'arboles de decisi\'on, redes neuronales y aprendizaje basado en ejemplos. El usuario de este tipo de herramientas puede integrarlas  a otros m\'odulos como parte de una aplicaci\'on completa \cite{23}.\\
\\
Otros ofrecen soporte en  m\'as de una etapa del proceso de DCBD  y una variedad de tareas de descubrimiento, t\'ipicamente, combinando clasificaci\'on, visualizaci\'on, consulta y clustering, entre otras \cite{23}. En este grupo est\'an Clementine \cite{29}, DBMiner \cite{11, 12, 13}, DBLearn \cite{14}, Data Mine \cite{19}, IMACS \cite{5}, Intelligent Miner \cite{18}, Quest \cite{2} entre otras. Una evaluaci\'on de un gran n\'umero de herramientas de este  tipo se puede encontrar en \cite{10}.\\
\\
Todas estas herramientas necesitan de la adquisici\'on de costosas licencias para su utilizaci\'on. Este hecho limita a las peque\~nas y medianas empresas u organizaciones, al acceso de herramientas DCBD para la toma de decisiones, que inciden directamente en la obtenci\'on de mayores ganancias y en el aumento de su competitividad.\\
\\
Por esta raz\'on, se plantea el desarrollo de una herramienta gen\'erica de DCBD, d\'ebilmente acoplada, bajo software libre, que permita el acceso a este tipo de herramientas sin ning\'un tipo de restricciones, a las peque\~nas y medianas  empresas u organizaciones de nuestro pa\'is o de cualquier parte del mundo.\\
\\
\section{Formulaci\'on del Problema}
?`El desarrollo de una herramienta gen\'erica para el Descubrimiento de Conocimiento en bases de datos d\'ebilmente acoplada con el SGBD PostgreSQL bajo software libre, facilitar\'a a las peque\~nas y medianas empresas la toma de decisiones?

\chapter{OBJETIVOS}
\section{Objetivo General}
Desarrollar una herramienta gen\'erica para el  Descubrimiento de Conoci\-miento en bases de datos  d\'ebilmente acoplada con el  sistema gestor de bases de datos PostgreSQL, bajo los lineamientos del Software Libre, que facilite la toma de decisiones a las peque\~nas y medianas empresas u organizaciones de nuestro pa\'is y de cualquier parte del mundo.\\
\\
\section{Objetivos Espec\'ificos}
\begin{enumerate}
\item Estudiar  y Analizar  diferentes herramientas  DCBD d\'ebilmente acopladas con un SGBD.
\item Analizar, dise\~nar y desarrollar programas que permitan la selecci\'on, preprocesamiento y transformaci\'on de datos.
\item Analizar, dise\~nar y desarrollar programas que implementen los opera\-dores algebraicos y primitivas SQL para las tareas de Asociaci\'on y Clasificaci\'on de datos.
\item Analizar, dise\~nar y desarrollar programas que implementen los algoritmos EquipAsso, MateTree, Apriori H\'ibrido, FP-Growth y C4.5.
\item Analizar, dise\~nar y desarrollar programas que permitan visualizar de manera gr\'afica las reglas de asociaci\'on y clasificaci\'on.
\item Integrar todos los programas desarrollados en una sola herramienta para DCBD.
\item Implementar la conexi\'on de la herramienta DCBD con el SGBD PostgreSQL.
\item Obtener conjuntos de datos reales para la realizaci\'on de las pruebas con la herramienta DCBD.
\item Realizar las pruebas y analizar los resultados con la herramienta DCBD d\'ebilmente acoplada con PostgreSQL.
\item Realizar las pruebas de rendimiento con los diferentes algoritmos implementados.
\item Dar a conocer los resultados del proyecto a trav\'es de la publicaci\'on de un art\'iculo en una revista o conferencia nacional o internacional.
\end{enumerate}

\chapter{JUSTIFICACI\'ON}
Todas las herramientas de DCBD o com\'unmente conocidas como herra\-mientas de miner\'ia de datos sirven en las organizaciones para apoyar la toma de decisiones de tipo semiestructurado, \'unica o que cambian r\'apidamente, y que no es f\'acil especificar por adelantado. Es evidente que por su dise\~no, estas herramientas tienen mayor capacidad anal\'itica que otras. Est\'an construidas expl\'icitamente con diversos modelos para obtener patrones insospechados a partir de los datos. Apoyan la toma de decisiones al permitir a los usuarios extraer informaci\'on \'util que antes estaba enterrada en monta\~nas de datos. Diversas herramientas de miner\'ia de datos disponibles en el mercado ofrecen diferentes tipos de arquitecturas que determinan, en alguna medida, su versatilidad y su costo. Por lo general todas estas son demasiado costosas, necesitan de licenciamiento para su uso y de software espec\'ifico.\\
\\
El desarrollo de TARIYKDD, como una herramienta DCBD bajo licencia p\'ublica GNU, permitir\'a que empresas u organizaciones que por su tama\~no no puedan acceder a herramientas DCBD propietarias, utilicen esta tecnolog\'ia para mejorar la toma de decisiones, maximicen sus ganancias con decisiones acertadas y eleven su poder competitivo, ya que el ritmo actual del mundo y la globalizaci\'on as\'i lo requieren.\\
\\
Por otra parte, TARIYKDD se convierte en otro aporte m\'as en el \'area del Descubrimiento de Conocimiento en bases de datos que la Universidad de Nari\~no, a trav\'es de su programa de Ingenier\'ia de Sistemas, hace al mundo, contribuyendo a la investigaci\'on cient\'ifica y al desarrollo de la regi\'on y del pa\'is.\\
\\
Por ser TARIYKDD una herramienta gen\'erica de DCBD, puede utilizarse en diferentes campos como la industria, la banca, la salud y la educaci\'on, entre otros.
\chapter{MARCO TEORICO}

%%%%%%%%%%%%%%%%%%%%%%%%%%%%%%%%%%%%%%%%%%%%%   PROCESO KDD   %%%%%%%%%%%%%%%%%%%%%%%%%%%%%%%%%%%%%%%%%%%%%%%%%
\section{El Proceso de Descubrimiento de Conocimiento en Bases de Datos - DCBD}
El proceso de DCBD es el proceso que utiliza m\'etodos de miner\'ia de datos (algoritmos)
para extraer
(identificar) patrones que evaluados e interpretados, de acuerdo a las especificaciones
de medidas y umbrales,
usando una base de datos con alguna selecci\'on, preprocesamiento, muestreo y
transformaci\'on, se obtiene lo que
se piensa es conocimiento \cite{9}.\\
\\
El proceso de DCBD es interactivo e iterativo, involucra numerosos pasos con la
intervenci\'on del usuario en la
toma de muchas decisiones y se resumen en las siguientes etapas:

\begin{itemize}
\item Selecci\'on.
\item Preprocesamiento / Data cleaning.
\item Transformaci\'on / Reducci\'on.
\item Miner\'ia de Datos (Data Mining).
\item Interpretaci\'on / evaluaci\'on.
\end{itemize}

%%%%%%%%%%%%%%%%%%%%%%%%%%%%% Integraci\'on herramientas DCBD con SGBD %%%%%%%%%%%%%%%%%%%%%%%%%%%%%%%%%%%%%%%%%
\section{Arquitecturas de Integraci\'on de las Herra\-mientas DCBD con un SGBD}
Las arquitecturas de integraci\'on de las herramientas DCBD con un SGBD se pueden ubicar
en una de tres tipos:
herramientas d\'ebilmente acopladas, medianamente acopladas y fuertemente acopladas con
un SGBD \cite{30}.\\
\\
Una arquitectura es d\'ebilmente acoplada cuando los algoritmos de Miner\'ia de Datos y
dem\'as componentes se
encuentran en una capa externa al SGBD, por fuera del n\'ucleo y su integraci\'on con
este se hace a partir de una
interfaz \cite{30}.\\
\\
Una arquitectura es medianamente acoplada cuando ciertas tareas y algoritmos de
descubrimiento de patrones se
encuentran formando parte del SGBD mediante procedimientos almacenados o funciones
definidas por el usuario
\cite{30}.\\
\\
Una arquitectura es fuertemente acoplada cuando la totalidad de las tareas y algoritmos
de descubrimiento de
patrones forman parte del SGBD como una operaci\'on primitiva, dot\'andolo de las
capacidades de descubrimiento de
conocimiento y posibilit\'andolo para desarrollar aplicaciones de este tipo \cite{30}.\\
\\
Por otra parte, de acuerdo a las tareas que desarrollen, las herramientas DCBD se
clasifican en tres grupos:
herramientas gen\'ericas de tareas senci\-llas, herramientas gen\'ericas de tareas
m\'ultiples y herramientas de
dominio espec\'ifico.\\
\\
Las Herramientas gen\'ericas de tareas sencillas principalmente soportan solamente la
etapa de miner\'ia de datos
en el proceso de DCBD y requieren un pre y un post procesamiento de los datos. El usuario
final de estas
herramientas es t\'ipicamente un consultor o un desarrollador quien podr\'ia integrarlas
con otros m\'odulos como
parte de una completa aplicaci\'on.\\
\\
Las Herramientas gen\'ericas de tareas m\'ultiples realizan una variedad de ta\-reas de
descubrimiento,
t\'ipicamente combinando clasificaci\'on, asociaci\'on, visualizaci\'on, clustering,
entre otros. Soportan
diferentes etapas del proceso de DCBD. El usuario final de estas herramientas es un
analista quien entiende la
manipulaci\'on de los datos.\\
\\
Finalmente, las Herramientas de dominio espec\'ifico, soportan descubrimiento solamente
en un dominio espec\'ifico
y hablan el lenguaje del usuario final, quien necesita conocer muy poco sobre el proceso
de an\'alisis.
\newpage
\section{Implementaci\'on de  Herramientas DCBD d\'ebilmente acopladas con un SGBD}
\begin{figure}[h]
   \centering
   \includegraphics[width=10cm,height=8cm]{imgsMteorico/arquitectura.jpg}
   \caption{Arquitectura DCBD d\'ebilmente acoplada}
   \label{fig1}
\end{figure}

La implementaci\'on de herramientas DCBD d\'ebilmente acopladas con un SGBD se hace a
trav\'es de SQL embebido en
el lenguaje anfitri\'on del motor de miner\'ia de datos \cite{3}.  Los datos residen en
el SGBD y son le\'idos
registro por registro a trav\'es de ODBC, JDBC o de una interfaz  de cursores SQL. La
ventaja de esta arquitectura
es su portabilidad. Sus principales desventajas son la escalabilidad y el rendimiento. El
problema de
escalabilidad consiste en que las herramientas y aplicaciones bajo este tipo de
arquitectura,  cargan todo el
conjunto de datos en memoria, lo que las limita para el manejo de grandes cantidades de
datos. El bajo rendimiento
se debe a que los registros son copiados uno por uno  del espacio de direccionamiento de
la base de datos al
espacio de direccionamiento de la aplicaci\'on de miner\'ia de datos \cite{6,15}  y
estas  operaciones de
entrada/salida,  cuando se manejan grandes vol\'umenes de datos,  son bastante costosas,
a pesar de la
optimizaci\'on de lectura por bloques presente en muchos SGBD (Oracle, DB2, Informix,
PostgreSQL.) donde un bloque
de tuplas puede ser le\'ido al tiempo (figura 1).

%%%%%%%%%%%%%%%%%%%%%%%%%%%%%%%%%%%%%%%%%%%%%%%% ALGORITMOS %%%%%%%%%%%%%%%%%%%%%%%%%%%%%%%%%%%%%%%%%%%%%%%%%%
\section{Algoritmos implementados en TariyKDD}
Dentro de la herramienta de Miner\'ia de Datos TariyKDD fueron implementados los siguientes algoritmos de
Asociaci\'on: \textit{Apriori, FPGrowth y EquipAsso}, as\'i como tambi\'en algoritmos de clasificaci\'on:
\textit{C.4.5 y MateBy}, los cuales son explicados a continuaci\'on:

%------------------- APRIORI
\subsection{Apriori}
La notaci\'on del algoritmo Apriori \cite{4} es la siguiente:\\

\begin{table}[h]
\caption{Notaci\'on algoritmo Apriori}
\end{table}
\begin{center}
\begin{tabular}{|p{40mm}|p{40mm}|}\hline
\textbf{$k$-itemset} & \textbf{Un itemset con $k$ items}\\ \hline
$L_{k}$              & Conjunto de itemsets frecuentes $k$ (Aquellos con soporte m\'inimo).\\ \hline
$C_{k}$              & Conjunto de itemsets candidatos $k$ (Itemstes potencialmente frecuentes)\\ \hline
\end{tabular}
\end{center}


A continuaci\'on se muestra el algoritmo Apriori:\\ \\
\begin{footnotesize}
\texttt{
\noindent 		$L_{1} =$ \{Conjunto de itemsets frecuentes 1 \}\\
	  		for ( $k=2;\ \ L_{k-1}\neq0;\ \  k++$  ) do begin\\
\indent   		$C_{k}=$ apriori-gen$(L_{k-1})$ // Nuevos candidatos\\
\indent 		forall $transacciones\ \ t\in D$ do begin\\
\indent\indent 			$C_{t}=$ subconjunto $(C_{k}, t);$ // Candidatos en t\\
\indent\indent 			forall $candidatos\ \ c \in C_{t}$ do\\
\indent\indent\indent 			$c.cont++;$\\
\indent 			end\\
\indent 		$L_{k}=$ \{ $c \in C_{k} | c.count \geq minsup$ \}\\
			end
			Answer $\cup_{k} L_{k}$\\
}
\end{footnotesize}

La primera pasada del algoritmo cuenta las ocurrencias de los items en todo el conjunto de datos para determinar
los itemsets frecuentes 1. Los subsecuentes pasos del algoritmo son basicamente dos, primero, los itemsets
frecuentes $L_{k-1}$ encontrados en la pasada $(k-1)$ son usados para generar los itemsets candidatos $C_{k}$,
usando la funci\'on apriori-gen descrita en la siguiente subsecci\'on. Y segundo se cuenta el soporte de los
itemsets candidatos $C_{k}$ a trav\'es de un nuevo recorrido a la base de datos. Se realiza el mismo proceso
hasta que no se encuentren m\'as itemsets frecuentes.

\subsubsection{Generaci\'on de candidatos en Apriori}
La funci\'on apriori-gen toma como argumento $L_{k-1}$, o sea todos los itemsets frecuentes $(k-1)$. La funci\'on
trabaja de la siguiente manera:\\

\begin{footnotesize}
\texttt{
\noindent 	insert into $C_{k}$\\
	  	select $p.item_{1},\ \ p.item_{2},...,\ \ p.item_{k-1},\ \ q.item_{k-1}$\\
		from $L_{k-1}\ \ p,\ \ L_{k-1}\ \ q$\\
		where $p.item_{1}=q.item_{1},...,\ \ p.item_{k-2}=q.item_{k-2},$\\
\indent			$p.item_{k-1}\<q.item_{k-1};$\\
}
\end{footnotesize}

A continuaci\'on, en el paso de poda, se borran todos los itemsets $c \in C_{k}$ tal que alg\'un subconjunto
$(k-1)$ de $c$ no este en $L_{k-1}$:\\

\begin{footnotesize}
\texttt{
forall $itemsets\ \ c \in C_{k}$ do\\
\indent forall subconjuntos $(k-1)\ \ s$ de $c$ do\\
\indent\indent if $(s \notin L_{k-1})$ then\\
\indent\indent\indent delete $c$ from $C_{k};$
}
\end{footnotesize}

%------------------- FPGROWTH
\subsection{FPGrowth}
\label{fpgrowth}
Como se puede ver en \cite{15} la heur\'istica utilizada por Apriori logra buenos resultados, ganados por
(posiblemente) la reducci\'on del tama\~no del conjunto de candidatos. Sin embargo, en situaciones con largos
patrones, soportes demasiado peque\~nos, un algoritmo tipo Apriori podr\'ia sufrir de dos costos no triviales:

\begin{itemize}
\item Es costoso administrar un gran n\'umero de conjuntos candidatos. Por ejemplo, si hay $10^{4}$ Itemsets
Frecuentes, el algoritmo Apriori necesitar\'a generar m\'as de $10^{7}$ Itemsets Candidatos, as\'i como acumular
y probar su ocurrencia. Adem\'as, para descubrir un patr\'on frecuente de tama\~no 100, como
${a_{1},...,a_{100}}$, se debe generar m\'as de $2^{100}$ candidatos en total.
\item Es una tarea demasiado tediosa el tener que repetidamente leer la base de datos para revisar un gran
conjunto de candidatos.
\end{itemize}
El cuello de botella de Apriori es la generaci\'on de candidatos \cite{15}. Este problema es atacado por los
siguientes tres aspectos:\\

Primero, una innovadora y compacta estructura de datos llamada \textit{\'Arbol de Patrones Frecuentes} o 
\textit{FP-tree} por sus siglas en ingles (\textit{Frequent Pattern Tree}), la cual es una estructura que 
almacena informaci\'on crucial y cuantitativa acerca de los patrones frecuentes. \'Unicamente los itemsets
frecuentes 1 tendr\'an nodos en el \'arbol, el cual esta organizado de tal forma que los items m\'as
frecuentes de una transacci\'on tendr\'an mayores oportunidades de compartir nodos en la estructura.\\ \\
Segundo, un m\'etodo de Miner\'ia de patrones crecientes basado en un FP-tree. Este comienza con un patr\'on
frecuente tipo 1 (como patr\'on sufijo inicial), examina sus Patrones Condicionales Base (una ''sub-base de
datos'' que consiste en el conjunto de items frecuentes, que se encuentran en patr\'on sufijo), construye su
FP-tree (condicional) y dentro de este lleva a cabo recursivamente Miner\'ia. El patr\'on creciente es 
conseguido a trav\'es de la concatenaci\'on del patr\'on sufijo con los nuevos generados del FP-tree condicional.
Un itemset frecuente en cualquier transacci\'on siempre se encuentra en una ruta de los \'arboles de Patrones
Frecuentes.\\ \\
Tercero, la t\'ecnica de busqueda empleada en la Miner\'ia esta basada en particionamiento, con el m\'etodo
''divide y venceras''. Esto reduce dramaticamente el tama\~no del Patr\'on Condicional Base generado en el 
siguiente nivel de busqueda, as\'i como el tama\~no de su correspondiente FP-tree. Es m\'as, en vez de buscar
grandes patrones frecuentes, busca otros m\'as peque\~nos y los concatena al sufijo. Todas estas t\'ecnicas 
reducen los costos de busqueda.

\subsubsection{Dise\~no y construcci\'on del \'Arbol de Patrones Frecuentes (FP-tree)}
Sea $I={a_{1},a_{2},...,a_{m}}$ un conjunto de items, y $DB={T_{1},T_{2},...,T_{m}}$ una base de datos de
transacciones, donde $T_{i}(i \in [1...n])$ es una transacci\'on que contiene un conjunto de items en $I$. El
soporte (u ocurrencia) de un patr\'on $A$ o conjunto de items, es el n\'umero de veces que $A$ esta contenida en
$DB$. $A$ es un patr\'on frecuente, si el soporte de $A$ es mayor que el umbral o soporte m\'inimo, $\xi$.

\paragraph{\'Arbol de Patrones Frecuentes}
A trav\'es del siguiente ejemplo se examina como funciona el dise\~no de la estructura de datos para minar con
eficiencia patrones frecuentes.

\subparagraph{Ejemplo 1}
Sea la base de datos de transacciones, cuadro \ref{tid} y $\xi=3$. Una estructura de datos puede ser dise\~nada de
acuerdo a las siguientes observaciones:
\begin{enumerate}
\item Aunque solo los items frecuentes jugar\'an un rol en la Miner\'ia de Patrones Frecuentes, es necesario leer 
la BD para identificar este conjunto de items.
\item Si se almacenan items frecuentes de cada transacci\'on en una estructura compacta, se podr\'ia
evitar el tener que leer repetidamente la BD.
\item Si m\'ultiples transacciones comparten un conjunto id\'entico de items frecuentes, estas pueden ser
fusionadas en una sola, con el n\'umero de ocurrencias como contador.
\item Si dos transacciones comparten un prefijo com\'un, las partes compartidas pueden ser unidas usando una
estructura prefija. Si los items frecuentes son ordenados descendientemente, habr\'a mayor probabilidad de que los
prefijos de las cadenas esten compartidos.
\end{enumerate}

\begin{table}[h]
\caption{Base de Datos de transacciones}
\label{tid}
\end{table}
\begin{center}
\begin{tabular}{|p{25mm}|p{30mm}|p{40mm}|}\hline
\textbf{TID} & \textbf{Items} & \textbf{Items frecuentes (ordenados)}\\ \hline\hline
100 & f,a,c,d,g,i,m,p & f,c,a,m,p\\ \hline
200 & a,b,c,f,l,m,o   & f,c,a,b,m\\ \hline
300 & b,f,h,j,o       & f,b\\ \hline
400 & b,c,k,s,p       & c,b,p\\ \hline
500 & a,f,c,e,l,p,m,n & f,c,a,m,p\\ \hline
\end{tabular}
\end{center}

\begin{figure}[ht]
   \centering
   \includegraphics[width=0.6\textwidth]{images/fptree1.png}
   \caption{Tabla de cabeceras y Arbol FP-tree del ejemplo 1}
   \label{fptid}
\end{figure}

\newpage

Con estas observaciones se puede construir un \'Arbol de Patrones Frecuentes de la siguiente forma:\\ \\
Primero, el leer $DB$ genera una lista de items frecuentes, $\{(f:4),(c:4),(a:3),(b:3),(m:3),(p:3)\}$, (el
n\'umero indica el soporte) ordenados descendientemente.\\ \\
Segundo, crear la ra\'iz del \'arbol con un \textit{null}. Al leer la primera 
transacci\'on se construye la primera rama del \'arbol $\{(f:1),(c:1),(a:1),(m:1),(p:1)\}$. Para la segunda 
transacci\'on ya que su lista de items frecuentes $(f,c,a,b,m)$ comparte el prefijo com\'un $(f,c,a)$ con la rama
existente $(f,c,a,m,p)$, el conteo de cada nodo en el \'arbol prefijo es incrementado en 1, dos nuevos nodos son
creados, $(b:1)$ enlazado como hijo de $(a:2)$ y $(m:1)$ enlazado como hijo de $(b:1)$. Para la tercera 
transacci\'on como su lista de frecuentes solamente es $(f,b)$ y el \'unico prefijo compartido es $(f)$, el
soporte de $(f)$ se incrementa en 1, y un nuevo nodo $(b:1)$ es creado y enlazado como hijo de $(f:3)$. La
lectura de la cuarta transacci\'on lleva a la construcci\'on de la segunda rama del \'arbol,
$\{(c:1),(b:1),(p:1)\}$. Para la \'ultima transacci\'on ya que su lista de items frecuentes $(f,c,a,m,p)$ es
identica a la primera, la ruta es compartida, incrementado el conteo de cada nodo en 1.\\ \\
Para facilitar m\'as el funcionamiento del \'arbol una Tabla de Cabeceras es construida, en la cual cada item
apunta a su ocurrencia en el \'arbol. Los nodos con el mismo nombre son enlazados en secuencia y despu\'es de
leer todas las transacciones, se puede ver el \'arbol resultante en la figura \ref{fptid}.\\ \\
Por tanto un \textbf{\'Arbol de Patrones Frecuentes (FP-tree)} es una estructura como se define a
continuaci\'on:\\

\begin{itemize}
\item Consiste de una raiz etiquetada como \textit{null}, un conjunto de \'arboles hijos de la raiz y una Tabla de
Cabeceras de items frecuentes.
\item Los nodos del \'Arbol de Patrones Frecuentes o FP-tree tienen tres campos: nombre del item, contador y
enlaces a los dem\'as nodos. El nombre del item registra que item este nodo representa, el contador registra el
n\'umero de transacciones representadas por la porci\'on de la ruta que alcanzan a este nodo y los enlaces llevan
al siguiente nodo en el FP-tree, que tiene el mismo nombre o a \textit{null} si no hay nada.
\item Cada entrada en la Tabla de Cabeceras de items frecuentes tiene dos campos, nombre del item y el primer
nodo enlazado, al cual apunta la cabecera con el mismo nombre.
\end{itemize}

\subsubsection{Minando Patrones Frecuentes con FP-tree}
Existen ciertas propiedades del \'Arbol de Patrones Frecuentes que facilitar\'an la tarea de Miner\'ia de Patrones
Frecuentes:

\begin{enumerate}
\item \textbf{Propiedad de nodos enlazados}. Para cualquier nodo frecuente $a_{i}$, todos los posibles Patrones
Frecuentes que contenga $a_{i}$ pueden ser obtenidos siguiendo los nodos enlazados de $a_{i}$, comenzando desde
$a_{i}$ en la Tabla de Cabeceras de items frecuentes.
\end{enumerate}

\subparagraph{Ejemplo 2}
El siguiente es el proceso de Miner\'ia basado en el FP-tree de la figura \ref{fptid}. De acuerdo a la propiedad
de nodos enlazados para obtener todos los Patrones Frecuentes de un nodo $a_{i}$, se comienza desde la cabeza de
$a_{i}$ (en la Tabla de Cabeceras).\\ \\
Comenzando por los nodos enlazados de $p$ su Patr\'on Frecuente resultante es $(p:3)$ y sus dos rutas en el
FP-tree son $(f:4,c:3,a:3,m:2,p:2)$ y $(c:1,b:1,p:1)$. La primera ruta indica que la cadena ''$(f,c,a,m,p)$''
aparece dos veces en la base de datos. Se puede observar que ''$(f,c,a)$'' aparece tres veces y ''$(f)$'' se
encuentra cuatro veces, pero con $p$ solo aparecen dos veces. Adem\'as para estudiar que cadena aparece con $p$,
unicamente cuenta el prefijo de $p$, $(f:4,c:3,a:3,m:2)$. Similarmente, la segunda ruta indica que la cadena
''$(c,b,p)$'' aparece solo una vez en el conjunto de transacciones de $DB$, o que el prefijo de $p$ es
$(c:1,b:1)$. Estos dos prefijos de $p$, $(f:2,c:2,a:2,m:2)$ y $(c:1,b:1)$, forman los Sub-Patrones Base de $p$ o
llamados Patrones Condicionales Base. La construcci\'on de un FP-tree sobre este Patr\'on Condicional Base lleva
a unicamente una rama $(c:3)$. As\'i que solo existe un Patr\'on Frecuente $(cp:3)$.\\ \\
Para el nodo $m$, se obtiene el Patr\'on Frecuente $(m:3)$ y las rutas $(f:4,c:3,a:3,m:2)$ y
$(f:4,c:3,a:3,b:1,m:1)$. Al igual que en el an\'alisis anterior se obtiene los Patrones Condicionales Base de m,
que son $(f:2,c:2,a:2)$ y $(f:1,c:1,a:1,b:1)$. Al construir un FP-tree sobre estos, se obtiene que el FP-tree
condicional de $m$ es $(f:3,c:3,a:3)$. Despu\'es se podr\'ia llamar recursivamente a una funci\'on de Miner\'ia
basada en un FP-tree (mine$((f:3,c:3,a:3)|m)$).\\

La figura \ref{fpcondicional} muestra como funciona (mine$((f:3,c:3,a:3)|m)$) y que incluye minar tres items
$a, c, f$. La primera deriva en un Patr\'on Frecuente $(am:3)$, y una llamada a (mine$((f:3,c:3)|am)$); 
la segunda deriva en un Patr\'on Frecuente $(cm:3)$, y una llamada a (mine$((f:3)|cm)$); y la tercera deriva en
unicamente el Patr\'on Frecuente $(fm:3)$. Con adicionales llamadas recursivas a (mine$((f:3,c:3)|am)$) se obtiene
$(cam:3)$, $(fam:3)$ y con una llamada a (mine$((f:3)|cam)$), se obtendr\'a el patr\'on m\'as largo $(fcam:3)$. 
Similarmente con la llamada de (mine$((f:3)|cm)$), se obtiene el patr\'on $(fcm:3)$. Adem\'as todo el conjunto de 
Patrones Frecuentes que tienen a $m$ es $\{(m:3),(am:3),(cm:3),(fm:3),(cam:3),(fam:3),(fcam:3),(fcm:3)\}$ o que 
explicado de otra forma, por ejemplo para $m$ los Patrones Frecuentes son obtenidos combinando a $m$ con todos 
sus Patrones Condicionales Base $(f:3,c:3,a:3)$, entonces sus Patrones Frecuentes ser\'ian los ya mencionados
$\{(m:3),(am:3),(cm:3),(fm:3),(cam:3),(fam:3),(fcam:3),(fcm:3)\}$.
.\\ \\
As\'i mismo, con el nodo $b$ se obtiene $(b:3)$ y tres rutas: $(f:4,c:3,a:3,b:1)$, $(f:4,b:1)$ y $(c:1,b:1)$. Dado
que los Patrones Condicionales Base de $b$: $(f:1,c:1,a:1)$, $(f:1)$ y $(c:1)$ no generan items frecuentes, el 
proceso de Miner\'ia termina. Con el nodo $a$ solo se obtiene un Patr\'on Frecuente $\{(a:3)\}$ y un Patr\'on
Condicional Base $\{(f:3,c:3)\}$. Adem\'as su conjunto de Patrones Frecuentes puede ser generado a partir de sus
combinaciones. Concatenandolas con $(a:3)$, obtenemos $\{(fa:3),(ca:3),(fca:3),\}$. Del nodo $c$ se deriva
$(c:4)$ y un Patr\'on Condicional Base $\{(f:3)\}$, y el conjunto de Patrones Frecuentes asociados con $(c:3)$ es
$\{(fc:3)\}$. Con el nodo $f$ solo se obtiene $(f:4)$ sin Patrones Condicionales Base.\\

\begin{figure}[t]
   \centering
   \includegraphics[width=1\textwidth]{images/fpcondicional.png}
   \caption{FP-tree condicional para $m$}
   \label{fpcondicional}
\end{figure}

En la siguiente tabla se muestran los Patrones Condicionales Base y los FP-trees generados.

\begin{table}[h]
\caption{Patrones Condicionales Base y FP-trees Condicionales}
\end{table}
\begin{center}
\begin{tabular}{|p{10mm}|p{76mm}|p{42mm}|}\hline
\textbf{Item} & \textbf{Patr\'on Condicional Base} & \textbf{FP-tree condicional}\\ \hline
$p$ & $\{(f:2,c:2,a:2,m:2),(c:1,b:1)\}$			& $\{(c:3)\}|p$\\ \hline
$m$ & $\{(f:2,c:2,a:2),(f:1,c:1,a:1,b:1)\}$		& $\{(f:3,c:3,a:3)\}|m$\\ \hline
$b$ & $\{(f:1,c:1,a:1),(f:1),(c:1)\}$			& $\phi$\\ \hline
$a$ & $\{(f:3,c:3)\}$					& $\{(f:3,c:3)\}|a$\\ \hline
$c$ & $\{(f:3)\}$					& $\{(f:3)\}|c$\\ \hline
$f$ & $\phi$						& $\phi$\\ \hline
\end{tabular}
\end{center}

Como se dijo anteriormente los Patrones Frecuentes se obtienen a partir de las combinaciones de cada uno de los
items con sus Patrones Condicionales Base. Por ejemplo para $m$, sus Patrones Frecuentes 
$\{(m:3),(am:3),(cm:3),(fm:3),(cam:3),(fam:3),(fcam:3),(fcm:3)\}$, son obtenidos de combinar a $m$ con cada uno
de sus Patrones Condicionales Base $\{(f:2,c:2,a:2),(f:1,c:1,a:1,b:1)\}$.

%------------------- EQUIPASSO
\subsection{EquipAsso}

\begin{enumerate}
\item Nuevos Operadores Del Algebra Relacional Para Asociaci\'on.
\begin{enumerate}
\item Operador Associator ($\alpha$)\\
Associator($\alpha$) es un operador algebraico unario que al contrario del operador Selecci\'on o 
Restricci\'on ($\sigma$), aumenta la cardinalidad o el tama\~no de una relaci\'on ya que genera a partir de cada
tupla de una relaci\'on, todas las posibles combinaciones de los valores de sus atributos, como tuplas de una
nueva relaci\'on conservando el mismo esquema. Por esta raz\'on esta operaci\'on, debe ser posterior a la 
mayor\'ia de operaciones en el proceso de optimizaci\'on de una consulta.\\ \\
Su sintaxis es la siguiente:\\
\begin{center}$\alpha_{tam_inicial, tam_final}$(R)\end{center}
El operador Associator genera, por cada tupla de la relaci\'on R, todos sus posibles subconjuntos (itemsets)
de diferente tama\~no. Associator toma cada tupla t de R y dos par\'ametros: $\<tam\_inicial\>$ y 
$\<tam_final\>$ como entrada, y retorna, por cada tupla t, las diferentes combinaciones de atributos
$X_{i}$, de tama\~no $\<tam\_inicial\>$ hasta tama\~no $\<tam\_final\>$, como tuplas en una nueva
relaci\'on. El orden de los atributos en el esquema de R determina los atributos en los subconjuntos con valores,
el resto se hacen nulos. El tama\~no m\'aximo de un itemset y por consiguiente el tama\~no final m\'aximo
$(\<tam\_final\>)$ que se puede tomar como entrada es el correspondiente al valor del grado de la relaci\'on.\\
Formalmente, sea $A=\{A_{1},...,A_{n}\}$ el conjunto de atributos de la relaci\'on R de grado n y cardinalidad m,
IS y ES el tama\~no inicial y final respectivamente de los subconjuntos a obtener. El operador $\alpha$ aplicado
a  R.

\begin{center}
$\alpha_{IS,ES}(R)=\{ \cup_{all}\ \ X_{i}\ \ |\ \ X_{i} \subseteq t_{i},\ \ t_{i} \in R, \forall_{i}
\forall_{k}(X_{i}=
\< v_{i}(A_{1}),v_{i}(A_{2}),null,...,v_{i}(A_{k}),null\>,v_{i}(A_{k})\< \> null),(i=(2^{n}-1)*m), (k=IS,...,ES),
\ \ A_{1}\< A_{2}\< ...A_{k},\ \ IS=1,ES=n\}$
\end{center}

produce una nueva relaci\'on cuyo esquema R(A) es el mismo de R de grado n y cardinalidad $m'=(2^{n}-1)*m$ y cuya
extensi\'on r(A) est\'a formada por todos los subconjuntos $X_{i}$ generados a partir de todas las combinaciones
posibles de los valores no nulos $v_{i}(A_{k})$ de los atributos de cada tupla $t_{i}$ de R. En cada tupla $X_{i}$
\'unicamente un grupo de atributos mayor o igual que IS y menor o igual que ES tienen valores, los dem\'as
atributos se hacen nulos.\\ \\
\textbf{Ejemplo 1}. Sea la relaci\'on R(A,B,C,D):

\begin{center}
\begin{tabular}{|c|c|c|c|}\hline
\textbf{A} & \textbf{B} & \textbf{C} & \textbf{D}\\ \hline
a1 & b1 & c1 & d1\\ \hline
a1 & b2 & c1 & d2\\ \hline
\end{tabular}
\end{center}

El resultado de $R1=\alpha_{2,4}(R)$, se puede observar en la figura \ref{r1}.

\item Operador Equikeep ($\chi$)\\
Equikeep ($\chi$) es un operador unario que restringe los valores de los atributos de cada una de las tuplas de la
relaci\'on R a \'unicamente los valores de los atributos que satisfacen una expresi\'on l\'ogica.\\ \\
Su sintaxis es la siguiente:\\ \\
$\chi_{expresion\_ logica}(R)$

\begin{table}[ht]
\begin{center}
\begin{tabular}{|c|c|c|c|}\hline
\textbf{A}    & \textbf{B}    & \textbf{C}    & \textbf{D}\\ \hline
a1   & b1   & null & null\\ \hline
a1   & null & c1   & null\\ \hline
a1   & null & null & d1\\ \hline
null & b1   & c1   & null\\ \hline
null & b1   & null & d1\\ \hline
null & null & c1   & d1\\ \hline
a1   & b1   & c1   & null\\ \hline
a1   & b1   & null & d1\\ \hline
a1   & null & c1   & d1\\ \hline
null & b1   & c1   & d1\\ \hline
a1   & b1   & c1   & d1\\ \hline
a1   & b2   & null & null\\ \hline
a1   & null & c1   & null\\ \hline
a1   & null & null & d2\\ \hline
null & b2   & c1   & null\\ \hline
null & b2   & null & d2\\ \hline
null & null & c1   & d2\\ \hline
a1   & b2   & c1   & null\\ \hline
a1   & b2   & null & d2\\ \hline
a1   & null & c1   & d2\\ \hline
null & b2   & c1   & d2\\ \hline
a1   & b2   & c1   & d2\\ \hline
\end{tabular}
\end{center}
\caption{Resultado de $R1=\alpha_{2,4}(R)$}
\label{r1}
\end{table}

El operador EquiKeep restringe los valores de los atributos de cada una de las tuplas de la relaci\'on R a
\'unicamente los valores de los atributos que satisfacen una expresi\'on l\'ogica $\<expr\_log\>$, la cual esta
formada por un conjunto de cl\'ausulas de la forma Atributo=Valor, y operaciones l\'ogicas AND, OR y NOT.
En cada tupla, los valores de los atributos que no cumplen la condici\'on $\<expr\_log\>$ se hacen nulos.
EquiKeep elimina las tuplas vac\'ias, i.e. las tuplas con todos los valores de sus atributos nulos.\\ \\
Formalmente, sea $A={A_{1},...,A_{n}}$ el conjunto de atributos de la relaci\'on R de esquema R(A), de grado n y
cardinalidad m. Sea p una expresi\'on l\'ogica integrada por cl\'ausulas de la forma $A_{i}=const$ unidas por los
operadores booleanos AND (\^{}), OR $(v)$, NOT $(\neg)$. El operador $\chi$ aplicado a la relaci\'on R con la
expresi\'on l\'ogica p:
\begin{center}
$\chi_{p}(R)=\{t_{i}(A)|\forall_{i}\forall_{j}(p(v_{i}(A_{j}))=v_{i}(A_{j})\ \ si\ \ p=true\ \ y\ \
p(v_{i}(A_{j}))=null\ \ si\ \ p=false),i=1...m',j=1...n,\ \ m'\leq m \}$
\end{center}
produce una relaci\'on de igual esquema R(A) de grado n y cardinalidad $m'$, donde $m'\leq m$. En su extensi\'on,
cada n-tupla $t_{i}$, esta formada por los valores de los atributos de R, $v_{i}(A_{j})$, que cumplan la
expresi\'on l\'ogica p, es decir $p(v_{i}(Y_{j}))$ es verdadero, y por valores nulos si $p(v_{i}(Y_{j}))$ es
falso.\\ \\
\textbf{Ejemplo 2}. Sea la relaci\'on R(A,B,C,D) y la operaci\'on $\chi_{A=a1 \lor B=b1 \lor C=c2 \lor
D=d1}(R)$:

\begin{center}
\begin{tabular}{|c|c|c|c|}\hline
\textbf{A} & \textbf{B} & \textbf{C} & \textbf{D}\\ \hline
a1 & b1 & c1 & d1\\ \hline
a1 & b2 & c1 & d2\\ \hline
a2 & b2 & c2 & d2\\ \hline
a2 & b1 & c1 & d1\\ \hline
a2 & b2 & c1 & d2\\ \hline
a1 & b2 & c2 & d1\\ \hline
\end{tabular}
\end{center}

El resultado de esta operaci\'on es la siguiente:

\begin{center}
\begin{tabular}{|c|c|c|c|}\hline
\textbf{A} & \textbf{B} & \textbf{C} & \textbf{D}\\ \hline
a1   & b1   & null & d1\\ \hline
a1   & null & null & null\\ \hline
null & null & c2   & null\\ \hline
null & b1   & null & d1\\ \hline
null & null & null & null\\ \hline
a1   & null & c2   & d1\\ \hline
\end{tabular}
\end{center}

\end{enumerate}
\item Algoritmo EquipAsso\\
El primer paso del algoritmo simplemente cuenta el n\'umero de ocurrencias de cada item para determinar los
1-itemsets frecuentes. En el subsiguiente paso, con el operador EquiKeep se extraen de todas las
transacciones los itemsets frecuentes tama\~no 1 haciendo nulos el resto de valores. Luego se aplica el operador
Associator desde $I_{s}=2$ hasta el grado n. A continuaci\'on se muestra el algoritmo Equipasso:

\begin{footnotesize}
\texttt{
$L1=\{1-itemsets\ \ frecuentes\};$\\
forall $transacciones\ \ t\in D$ do begin\\
\indent $R=\chi_{L1}(D)$\\
\indent $K=2$\\
\indent $g=grado(R)$\\
\indent $R'=\alpha_{k,g}(R)$\\
end\\
$L_{k}=\{count(R')|c.count\geq minsup\}$\\
Respuesta $=\cup L_{k};$
}
\end{footnotesize}
\end{enumerate}

%------------------- MATE
\subsection{Mate}
El operador Mate genera, por cada una de las tuplas de una relaci\'on, todas los posibles combinaciones formadas
por los valores no nulos de los atributos pertenecientes a una lista de atributos denominados Atributos
Condici\'on, y el valor no nulo del atributo denominado Atributo Clase.\\

Tiene la sigiente sintaxis:
\begin{center}
$\mu$ lista\_atributos\_condici\'on;\ \ atributo\_clase(R) \\
\end{center} 
donde lista\_atributos\_condici\'on es el conjunto de atributos de la relaci\'on R a combinar con el atributo clase
y atributo\_clase es el atributo de R definido como clase.\\

Mate toma como entrada cada tupla de R y produce una nueva relaci\'on cuyo esquema esta formado por los atributos
condici\'on, lista\_atributos\_condici\'on y el atributo clase, atributo\_clase con tuplas formadas por todas las
posibles combinaciones de cada uno de los atributos condici\'on con el atributo clase, los dem\'as valores de los
atributos se hacen nulos.\\

Formalmente, sea $ A =\{A1, . . ., An\} $el conjunto de atributos de la relaci\'on R de grado n y cardinalidad m,
$LC [] A, LC \neq []$ la lista de atributos condici\'on a emparejar y $n'$ el n\'umero de atributos de LC, $[]LC[]
= n', n'\< n, Ac[]A , Ac[]L = []$ el atributo clase con el que se emparejar\'an los atributos de LC. El operador
$\mu$ aplicado a la lista de atributos LC, al atributo clase Ac de la relaci\'on R:

\begin{center}
$\mu_{LC;Ac}(R)=\{ti(M)[]M=LC\textit{[]}\ \ Ac, LC[] A,[]LC[]=n', n'\<,Ac\textit{[]}LC=\textit{[]},\ \ ti=Xi,\ \
i=1..m',\ \ m'=(2^{n'}-1)*m,\textit{[]}_{i}\textit{[]}_{k}(X_{i}=\< null,...,v_{i}(A_{k})...,null,...,v_{i}(Ac)\>,\
\ v_{i}(A_{k})v_{i}(Ac)\<\>null),\ \ k=1..n' \}$
\end{center}

produce una relaci\'on cuyo esquema es R(M) ,$ M=LC [] Ac$, de grado g,\ \ $ g= n'+1 $\ \ y cuya extensi\'on r(M)
de cardinalidad $ m', m'= (2n'-1)*m $\ \ es el conjunto de g-tuplas ti, tal que en cada g-tupla \'unicamente los
atributos que forman la combinaci\'on Xi [] LC, $(k =1..n')$ \ \ y el atributo Ac tienen valor, el resto de
atributos se hacen nulos.\\

Mate empareja en cada partici\'on todos los atributos condici\'on con el atributo clase, lo que facilita el conteo
y el posterior c\'alculo de las medidas de entrop\'\i{}a y ganancia de informaci\'on. El operador Mate genera estas
combinaciones, en una sola pasada sobre la tabla de entrenamiento (lo que redunda en la eficiencia del proceso de
construcci\'on del \'arbol de decisi\'on).\\

\textbf{Ejemplo 4.7.} Sea la relaci\'on R(A,B,C,D) del cuadro \ref{t1} obtener las diferentes combinaciones de los
atributos A,B con el atributo D, es decir, $R1= \mu_{A,B;D}(R).$\\

\begin{center}
\begin{table}[h]
\begin{center}
\begin{tabular}{|p{10mm}|p{10mm}|p{10mm}|p{10mm}|} \hline
\textbf{A} & \textbf{B} & \textbf{C} & \textbf{D}\\ \hline
a1 & b1 & c1 & d1\\ \hline
a1 & b1 & c1 & d1\\ \hline
\end{tabular}
\end{center}
\caption{Relaci\'on R}
\label{t1}
\end{table}
\end{center}

El resultado de la operaci\'on $R1= \mu_{A,B;D}(R)$, se muestra en el cuadro \ref{t2}:\\

\begin{center}
\begin{table}[h]
\begin{center}
\begin{tabular}{|p{10mm}|p{10mm}|p{10mm}|} \hline
\textbf{A} & \textbf{B} & \textbf{C}\\ \hline
a1 & null & d1\\ \hline
null & b1 & d1\\ \hline
a1 & b1 & d1\\ \hline
a1 & null & d2\\ \hline
null & b2 & d2\\ \hline
a1 & b2 & d2\\ \hline
\end{tabular}
\end{center}
\caption{Resultado de la Operaci\'on $R1= \mu_{A,B;D}(R)$}
\label{t2}
\end{table}
\end{center}

\subsubsection{Funci\'on Algebraica Agregada Entro}
La funci\'on Entro() permite calcular la entrop\'\i{}a de una relaci\'on R con respecto a un atributo denominado
atributo condici\'on y un atributo clase.\\

Tiene la siguiente sintaxis:\\

\begin{center}
 Entro(Atributo; Atributo\_clase; R)\\
\end{center}

donde \textit{atributo} es el atributo condici\'on de la relaci\'on R y \textit{atributo\_clase} es el atributo
con el que se combina el atributo \textit{atributo}.\\

Formalmente, sea $A=\{A1, ..., An,Ac\}$ el conjunto de atributos de la relaci\'on $R$ con esquema $R(A)$,
extensi\'on $r(A)$, grado $n$ y cardinalidad $m$. Sea $t$ el n\'umero de distintos valores del $atributo\_clase$
$Ac$, $Ac [] R(A)$ que divide a $r(A)$ en $t$ diferentes clases, $Ci (i=1..t).$ Sea $ri$ el n\'umero de tuplas de
$r(A)$ que pertenecen a la clase $Ci$. Sea $q$ el n\'umero de destintos valores
$\{v_{1}(A_{k}),v_{2}(A_{k}),..,v_{q}(A_{k})\}$ del $atributo$ $A_{k}$ , $A_{k}[] R(A)$, el cual particiona a
$r(A)$ en $q$ subconjuntos $\{S_{1},S_{2},...S_{q}\},$ donde $S_{j}$ contiene todos las tuplas de $r(A)$ que
tienen el valor $v_{j}(A_{k})$ del atributo $A_{k}$. Sea sij el n\'umero de tuplas de la clase $C_{i}$ en el
subconjunto $S_{j}$. La funci\'on $Entro(A_{k}; Ac; R)$, retorna la entrop\'\i{}a de $R$ con respecto al atributo
$A_{k}$, que se obtiene de la siguiente manera :\\

\begin{center}
$Entro(A_{k;}Ac;R) = \{ y | y = - \sum p_{ij} log_{2}(p_{ij}), i=1..t, j=1 ..q, p_{ij} = s_{ij}/|S_{j}| \} $ \\
\end{center}

donde $ p_{ij}= s_{ij} / | S_{j} es la probabilidad que una tupla en S_{j} pertenezca a la clase C_{i}.$ \\

La entrop\'\i{}a de R con respecto al atributo clase Ac es:\\

\begin{center}
$ Entro(Ac;Ac; R)=\{ y | y = - \sum p_{i} log_{2}(p_{i}), i =1 ..t, p_{i} = r_{i} / m\} $ \\ 
\end{center}

donde $p_{i}$ es la probabilidad que un tupla cualquier pertenezca a la clase $C_{i} y r_{i} $ el n\'umero de
tuplas de r(A) que pertenecen a la clase $C_{i}.$

\subsubsection{Funci\'on Algebraica Agregada Gain}
La funci\'on Gain() permite calcular la reducci\'on de la entrop\'\i{}a causada por el conocimiento del valor de
un atributo de una relaci\'on.\\

Su sintaxis es:\\
\begin{center}
Gain(atributo;atrib\_clase;R).\\
\end{center}

donde \textit{atributo} es el atributo condici\'on de la relaci\'on R y \textit{atributo\_clase} es el atributo
con el que se combina el atributo \textit{atributo}.\\

La funci\'on Gain() permite calcular la ganancia de informaci\'on obtenida por el particionamiento de la
relaci\'on R de acuerdo con el atributo \textit{atributo} y y se define:\\

\begin{center}
$Gain(A_{k;}Ac; R)=\{ y | y = Entro(Ac;Ac; R) - Entro(A_{k;}Ac; R) \} $\\
\end{center}

donde Entro (Ac; Ac; R) es la entrop\'\i{}a de la relaci\'on R con respecto al atributo clase Ac y
$Entro(A_{k};Ac; R)$ es la entrop\'\i{}a de la relaci\'on R con respecto al atributo $A_{k}.$

\subsubsection{Operador Describe Classifier $([]\mu)$}
Describe Classifier $([]\mu)$ es un operador unario que toma como entrada la relaci\'on resultante de los
operadores Mate By, Entro() y Gain() y produce una nueva relaci\'on donde se almacenan los valores de los
atributos que formar\'an los diferentes nodos del arbol de decisi\'on.\\

La sintaxis del operador Describe Classifier es la siguiente:\\

\begin{center}
$ []\mu(R)$\\
\end{center}

Formalmente, sea $A=\{A_{1}, .., A_{n},E,G\}$ el conjunto de atributos de la relaci\'on R de grado $n+2$ y
cardinalidad m. El operador []\textit{[]} aplicado a R:\\

\begin{center}
$ []\textit{[]}(R) \ \ = \ \ \{ t_{i}(Y)[] \ \ Y=\{ N.P,A.V,C\}, \ \ si \ \ t_{i}=
\textit{val(N),null,val(A),null,null} \ \ [] raiz, \ \ si \ \ t_{i}= \textit{val(N),val(P),val(A),val(V),val(C)} \
\ [] \ \ hoja,\ \ si \ \ t_{i} =\textit{val(N),val(P),val(A),val(V),null} \ \ [] \ \ nodo \ \ interno  \}$
\end{center}

produce una nueva relaci\'on con esquema R(Y), Y={N,P,A,V,C }donde N es el atributo que identifica el n\'umero de
nodo, P identifica el nodo padre, A identifica el nombre del atributo asociado a ese nodo, V es el valor del 
atributo A y C es el atributo clase. Su extensi\'on r (Y), est\'a formada por un conjunto de tuplas en las cuales
si los valores de los atributos son:\\

$N \< \> null,\ \ P=null,\ \ A\< \> null,\ \ V=null \ \ y \ \ C=null \ \ corresponde \ \ a \ \ un \ \ nodo \ \
raiz;\ \ si \ \ N\<\>null,\\
P\<\>null,\ \ A\<\>null,V\<\>null \ \ y \ \ C\<\>null \ \ corresponde \ \ a \ \ una \ \ hoja \ \ o \ \ nodo \ \
terminal \ \ y \ \ si \ \ N\<\>null,\\
P\<\>null,\ \ A\<\>null,V\<\>null\ \ y\ \ C=null\ \ corresponde\ \ a\ \ un\ \ nodo\ \ interno.$\\

Describe Classifier facilita la construcci\'on del \'arbol de decisi\'on y por consiguiente la generaci\'on de
reglas de clasificaci\'on.\\

\textbf{Ejemplo 4.8.} Sea la relaci\'on ARBOL(NODO,PADRE,ATRIBUTO,VALOR,CLASE) del cuadro \ref{t3} resultado del
operador Describe Classifier . Construir el \'arbol de decisi\'on.\\

\begin{center}
\begin{table}[h]
\begin{center}
\begin{tabular}{|p{20mm}|p{20mm}|p{30mm}|p{20mm}|p{20mm}|} \hline
\textbf{NODO} & \textbf{PADRE} & \textbf{ATRIBUTO} & \textbf{VALOR} & \textbf{CLASE}\\ \hline
N0 & Null & Temperatura & Null & Null\\ \hline
N1 & N0 & Temperatura & Alta & Si\\ \hline
N2 & N0 & Temperatura & Media & No \\ \hline
N3 & N0 & Temperatura & Normal & Null \\ \hline
N4 & N3 & D\_muscular & Si & Si \\ \hline
N5 & N3 & D\_muscular & No & No \\ \hline
\end{tabular}
\end{center}
\label{t3}
\caption{Relaci\'on \'Arbol}
\end{table}
\end{center}

El resultado se muestra en la figura \ref{t4}

\begin{figure}[ht]
\centering
\includegraphics[width=1\textwidth]{images/image01.png}
\caption{\'Arbol de decisi\'on}
\label{t4}
\end{figure}

%%%%%%%%%%%%%%%%%%%%%%%%%%%%%%%%%%%%%%%%%% Estado del arte %%%%%%%%%%%%%%%%%%%%%%%%%%%%%%%%%%%%%%%%%%%%%%%%%%
\newpage
\section{Estado del Arte}
En el desarrollo de nuestro trabajo de grado realizamos un estudio de herramientas de miner\'ia de datos
elaboradas por otras universidades y centros de investigaci\'on que nos permitieron ver el estado actual de este
tipo de aplicaciones. A continuaci\'on se describen las herramientas estudiadas.

\subsection{WEKA - Waikato Environment for Knowledge Analysis}
WEKA es una herramienta libre de miner\'ia de Datos realizada en el departamento de Ciencias de la Computaci\'on
de la Universidad de Waikato en Hamilton, Nueva Zelanda.  Los principales gestores de este proyecto son Ian H.
Waitten y Eibe Frank autores del libro Data Mining: Practical Machine Learning Tools and Techniques with Java
Implementations \cite{37} cuyo octavo cap\'itulo sirve como tutorial de Weka y es libremente distribuido junto con
el ejecutable y el c\'odigo fuente.  Las ultimas versiones estables de Weka pueden ser descargadas de la p\'agina
oficial del Proyecto \cite{38}.\\

La implementaci\'on de la herramienta fue hecha bajo la plataforma Java utilizando la versi\'on 1.4.2 de su
m\'aquina virtual.  Al ser desarrollada en Java, Weka posee todas las caracter\'isticas de una herramienta
orientada a objetos con la ventaja de ser multiplataforma, la ultima versi\'on (3.4) ha sido evaluada bajo
ambientes GNU/Linux, Macintosh y Windows siguiendo una arquitectura Cliente/Servidor lo que permite ejecutar
ciertas tareas de mane\-ra distribuida.\\

La conexi\'on a la fuente de datos puede hacerse directamente hacia un archivo plano, considerando varios formatos
como C4.5 y CVS aunque el formato oficial es el ARFF que se explica m\'as adelante, o a trav\'es de un driver JDBC
hacia diferentes Sistemas Gestores de Bases de Datos(SGBD).\\

Entre las principales caracter\'isticas de Weka esta su modularidad la cual se fundamenta en un estricto y
estandarizado formato de entrada de datos que denominan ARFF (Atributte - Relation File Format), a partir de este
tipo de archivo todos los algoritmos de miner\'ia implementados en Weka son trabajados.  Nuevas implementaciones y
adiciones deben ajustarse a este formato.\\

El formato ARFF consiste en una archivo sencillo de texto que funciona a partir de etiquetas, similar a XML, y
que debe cumplir con dos secciones: una cabecera y un conjunto de datos.  La cabecera contiene las etiquetas del
nombre de la relaci\'on (@relation) y los atributos (@atributte), que describen los tipos y el orden de los
datos. La secci\'on datos (@data) en un archivo ARFF contiene todos los datos del conjunto que se quiere evaluar
separados por comas y en el mismo orden en el que aparecen en la secci\'on de atributos \cite{arff}.\\

La conexi\'on al SGBD se hace en primera instancia a trav\'es de una interfaz gr\'afica de usuario construyendo
una sentencia SQL siguiendo un modelo relacional pero a partir de construida la tabla a minar se trabaja en
adelante fuera de l\'inea a trav\'es de flujos (streams) que se comunican con archivos en disco duro.\\

WEKA cubre gran parte del proceso de Descubrimiento de Conocimiento, las caracter\'isticas especificas de
miner\'ia de datos que se pueden ver son pre-procesamiento y an\'alisis de datos, clasificaci\'on, clustering,
asociaci\'on y visualizaci\'on de resultados.\\

WEKA posee una rica colecci\'on de algoritmos que soportan el proceso de miner\'ia que aplican diversas
metodolog\'ias como arboles de decisi\'on, conjuntos difusos, reglas de inducci\'on, m\'etodos estad\'isticos y
redes bayesianas.\\

Por poseer diferentes modos de interfaces gr\'aficas, WEKA se puede considerar una herramienta orientada al
usuario aunque cabe aclarar que exige un buen dominio de conceptos de miner\'ia de datos y del objeto de
an\'alisis.  Muchas tareas se encuentran soportadas aunque no del todo automatizadas por lo que el proceso de
descubrimiento debe ser guiado a\'un por un analista.\\

Un an\'alisis completo de las caracter\'isticas de WEKA con respecto a otras aplicaciones presentes en el mercado
se puede encontrar en \cite{10}.

\subsection{ADaM - Algorithm Development and Mining System}
El proyecto ADaM es desarrollado por el Centro de Sistemas y Tecnolog\'ias de la Informaci\'on de la Universidad
de Alabama en Huntsville, Estados Unidos.  Este sistema es usado principalmente para aplicar t\'ecnicas de
miner\'ia a datos cient\'ificos obtenidos de sensores remotos \cite{adam}.\\

ADaM es un conjunto de herramientas de miner\'ia y procesamiento de im\'agenes que consta de varios componentes
interoperables que pueden u\-sarse en conjunto para realizar aplicaciones en la soluci\'on de diversos
pro\-blemas. En la actualidad, ADaM (en su versi\'on 4.0.2) cubre cerca de 120 componentes \cite{adamComp} que
pueden ser configurados para crear procesos de mine\-r\'ia personalizados.  Nuevos componentes pueden f\'acilmente
integrarse a un sistema o proceso existente.\\

ADaM 4.0.2 provee soporte a trav\'es del uso de componentes aut\'onomos dentro de una arquitectura distribuida. 
Cada componente esta desarrollado en C, C++ u otra interfaz de programaci\'on de aplicaciones y entrega un
ejecutable a modo de script donde cada componente recibe sus par\'ametros a trav\'es de l\'inea de comandos y
arroja los resultados en ficheros que pueden ser utilizados a su vez por otros componentes ADaM.  Eventualmente
se ofrece Servicios Web de algunos componentes por lo que se puede acceder a ellos a trav\'es de la Web.\\

Lastimosamente el acceso a fuentes de datos es limitada no ofreciendo conexi\'on directa hacia un sistema gestor
de bases de datos. ADaM trabaja generalmente con ficheros ARFF \cite{arff} que son generados por sus
componentes.\\

Dentro de las herramientas ofrecidas por ADaM encontramos soporte al preprocesamiento de datos y de im\'agenes,
clasificaci\'on, clustering y asociaci\'on.  La visualizaci\'on y an\'alisis de resultados se deja para ser 
implementado por el sistema que invoca a los componentes. ADaM 4.0.2 ofrece un amplio conjunto de herramienta que
implementa diversos metodolog\'ias dentro del \'area del descubrimiento de conocimiento.  Existen m\'odulos que
implementan \'arboles de decisi\'on, reglas de asociaci\'on, m\'etodos estad\'isticos, algoritmos gen\'eticos y
redes bayesianas.\\

ADaM 4.0.2 no soporta una interfaz gr\'afica de usuario, se limita a ofrecer un conjunto de herramientas para ser
utilizadas en la construcci\'on de sistemas que cubran diferentes \'ambitos.  Por tal motivo, es necesario un buen
conocimiento de los conceptos de miner\'ia de datos, a parte de fundamentos en el an\'alisis y procesamiento de
im\'agenes.  No obstante, ADaM ofrece un muy buen soporte a sistemas donde se busque descubrimiento de
conocimiento, un ejemplo de la implementaci\'on de ADaM puede verse en \cite{adamImpl} donde se utilizan
componentes ADaM en el an\'alisis e interpretaci\'on de im\'agenes satelitales de ciclones para estimar la
m\'axima velocidad de los vientos.

\subsection{Orange - Data Mining Fruitful and Fun}
Constru\'ido en el Labor\'atorio de Inteligencia Artif\'icial de la F\'acultad de Computaci\'on y Ciencias de la
Informaci\'on de la Universidad de Liubliana, en Eslovenia y ya que fu\'e liberado bajo la Licencia P\'ublica
General (GPL) puede ser descargado desde su sitio oficial \cite{oran}.\\

Orange \cite{oranwp} en su n\'ucleo es una librer\'ia de objetos C++ y rutinas que incluyen entre otros,
algoritmos estandar y no estandar de Miner\'ia de Datos y Aprendizaje Maquinal, adem\'as de rutinas para la
entrada y manipulaci\'on de datos. Orange provee un ambiente para que el usuario final pueda acceder a la
herramienta a trav\'es de scripts hechos en Python y que se encuentran un nivel por encima del n\'ucleo en  C++.
Entonces el usuario puede incorporar nuevos algoritmos o utilizar c\'odigo ya elaborado y que le permiten cargar,
limpiar y minar datos, as\'i como tambi\'en imprimir \'arboles de desici\'on y reglas de asociaci\'on.\\

Otra caracter\'istica de Orange es la inclusi\'on de un conjunto de Widgets gr\'aficos que usan m\'etodos y
m\'odulos del n\'ucleo central (C++) brindando una intef\'az agradable e intuitiva al usuario.\\

Los Widgets de la interfaz gr\'afica y los m\'odulos en Python incluyen tareas de Miner\'ia de Datos desde 
preprocesamiento hasta modelamiento y evaluaci\'on. Entre otras funciones estos componentes poseen t\'ecnicas
para:

\begin{description}
\item [Entrada de datos] Orange proporciona soporte para varios formatos popu\-lares de datos. Como por ejemplo:
	\begin{description}
	\item  [.tab] Formato nativo de Orange. La primera l\'inea tiene los nombres de los atributos, la segunda
	l\'inea dice cual es el tipo de datos de cada columna (discreto o continuo) y en adelante se encuentran
	los datos separados a trav\'es de tabuladores.
	\item [.c45] Estos archivos est\'an compuestos por dos archivos uno con extenci\'on .names que tiene los
	nombres de las columnas separados por comas y otro .data con los datos separados tambi\'en con comas.
	\end{description}
	
\item [Manipulaci\'on de datos y preprocesamiento] Entre las tareas que Oran\-ge incluye en este apartado tenemos
visualizaci\'on gr\'afica y estad\'istica de datos, procesamiento de filtros, discretizaci\'on y construcci\'on
de nuevos atributos entre otros m\'etodos.

\item [Miner\'ia de Datos y Aprendizaje M\'aquinal] Dentro de esta rama Oran\-ge incluye variedad de algoritmos
de asociaci\'on, clasificaci\'on, regresi\'on logistica, regresi\'on lineal, \'arboles de regresi\'on y
acercamientos basados en instancias (instance-based approaches).

\item [Contenedores] Para la calibraci\'on de la predicci\'on de probabilidades de modelos de clasificaci\'on.

\item [M\'etodos de evaluaci\'on] Que ayudan a medir la exactiud de un clasificador.
\end{description}

Podr\'ia catalogarse como una falencia el hecho de que Orange no incluya un modulo para la conexi\'on a un Sistema
Gestor de Bases de Datos, pero si se revisa bien la documentaci\'on de los modulos se encuentra que ha sido
desarrollado uno para la conexi\'on a MySQL (orngMySQL \cite{oransql}). El modulo provee una entrada a MySQL a
trav\'es de este modulo y de sencillos comandos en Python los datos de las tablas pueden transferidos desde y
hacia MySQL. As\'i mismo programadores independientes han desarrollado varios modulos entre los que se destaca
uno para algoritmos de Inteligencia Artificial.\\

Dentro de las herramientas de Miner\'ia de Datos, Orange podr\'ia catalogar\-se como una Herramienta Gen\'erica
Multitarea. Estas herramientas realizan una variedad de tareas de descubrimiento, t\'ipicamente combinando
clasificaci\'on, asociaci\'on , visualizaci\'on, clustering, entre otros. Soportan diferentes etapas del proceso
de DCBD.  El usuario final de estas herramientas  es un analista quien entiende la manipulaci\'on de los datos.\\

Orange funciona en varias plataformas como Linux, Mac y Windows y desde su sitio web \cite{oran} se puede
descargar toda la documentaci\'on disponible para su instalaci\'on. Al instalar Orange el usuario puede acceder a
una completa informaci\'on de la herramienta, as\'i como a pr\'acticos tutoriales y manuales que permiten
familiarizarse con la misma. Su sitio web \cite{oran} incluye tutoriales b\'asicos sobre Python y Orange,
manuales para desarrolladores m\'as avanzados y adem\'as tener acceso a los foros de Orange en donde se despejan
todos los interrogantes sobre esta herramienta.\\

En si Orange esta hecho para usuarios experimentados e investigadores en aprendizaje maquinal con la ventaja de
que el software fue liberado con licencia GPL, por tanto cualquier persona es libre de desarrollar y probar sus
propios algoritmos reusando tanto c\'odigo como sea posible.

\subsection{TANAGRA - A Free Software for Research and Academic Purposes}

TANAGRA \cite{rico} es software de Miner\'ia de Datos con pr\'opositos ac\'ademicos e investigativos,
desarrollado por Ricco Rakotomalala, miembro del Equipo de Investigaci\'on en Ingenier\'ia del Conocimiento (ERIC
- Equipe de Recherche en Ing\'enierie des Connaissances \cite{eric}) de la Universidad de Lyon, Francia. Conjuga
varios m\'etodos de Miner\'ia de Datos como an\'alisis exploratorio de datos, aprendizaje estad\'istico y
aprendizaje maquinal.\\

Este proyecto es el sucesor de SIPINA, el cu\'al implementa varios algoritmos de aprendizaje supervisado,
especialmente la construcci\'on visual de \'arboles de desici\'on. TANAGRA es m\'as potente, contiene adem\'as de
algunos paradigmas supervisados, clustering, an\'alisis factorial, estad\'istica param\'etrica y
no-param\'etrica, reglas de asociaci\'on, selecci\'on de caracter\'isticas y cons\-trucci\'on de algoritmos.\\

TANAGRA es un proyecto open source, as\'i que cualquier investigador puede acceder al c\'odigo fuente y a\~nadir
sus propios algoritmos, en cuanto este de acuerdo con la licencia de distribuci\'on.\\

El principal pr\'oposito de TANAGRA es proponer a los investigadores y estudiantes una arquitectura de software
para Miner\'ia de Datos que permita el an\'alisis de datos tanto reales como sint\'eticos.\\

El segundo pr\'oposito de TANAGRA es proponer a los investigadores una arquitectura que les permita a\~nadir sus
propios algoritmos y m\'etodos, para as\'i comparar sus rendimientos. TANAGRA es m\'as una plataforma
experimental, que permite ir a lo esencial y obviarse la programaci\'on del manejo de los datos.\\

El tercero y \'ultimo pr\'oposito va dirigido a desarrolladores novatos y consiste en difundir una metodolog\'ia
para construir este tipo de software con la ventaja de tener acceso al c\'odigo fuente, de esta forma un
desarrollador puede ver como ha sido construido el software, cuales son los problemas a evadir, cuales son los
principales pasos del proyecto y que herramientas y librer\'ias usar.\\

Lo que no incluye TANAGRA es lo que constituye la fuerza del software comercial en el campo de la Miner\'ia de
Datos: Grandes fuentes de datos, acceso directo a bodegas y bases de datos (Data Warehouses y Data Bases) as\'i
como la aplicaci\'on de data cleaning.\\

Para realizar trabajos de Miner\'ia de Datos con Tanagra, se deben crear esquemas y diagramas de flujo y utilizar
sus diferentes componentes que van desde la visualizaci\'on de datos, estad\'isticas, construcci\'on y
selecci\'on de ins\-tancias, regresi\'on y asociaci\'on entre otros.\\

El formato del conjunto de datos de Tanagra es un archivo plano (.txt) separado por tabulaciones, en la primera
l\'inea tiene un encabezado con el nombre de los atributos y de la clase, a continuaci\'on vienen los datos con
el mismo formato de separaci\'on con tabuladores. Tanagra tambi\'en acepta conjuntos de datos generados en Excel
y uno de los estandares en Miner\'ia de Datos, el formato de WEKA (archivos .arff). Tanagra permite cargar un
solo conjunto de datos.\\

A medida que se trabaja con TANAGRA y se van generando reportes de resultados, existe la posibilidad de
guardarlos en formato HTML.\\

La paleta de componentes de Tanagra tiene implementaciones de tareas de Miner\'ia de Datos que van desde
preprocesamiento hasta modelamiento y evaluaci\'on. Entre otras TANAGRA permite usar t\'ecnicas para:

\begin{description}
\item [Entrada de datos] Con soporte para varios formatos populares de datos.
\item [Manipulaci\'on de datos y preprocesamiento] Como muestreo, filtrado, discretizaci\'on y construcci\'on de nuevos atributos entre otros m\'etodos.
\item [Construcci\'on de Modelos] M\'etodos para la construcci\'on de modelos de clasificaci\'on, que incluyen \'arboles de clasificaci\'on, clasificadores Naive-Bayes y regresi\'on log\'istica.
\item [M\'etodos de regresi\'on] Como regresi\'on multiple lineal.
\item [M\'etodos de evaluaci\'on] Utilizados para medir la exactitud de un clasificador.
\end{description}

Al ser TANAGRA un proyecto Open Source es facilmente descargable desde su sitio web \cite{tana}, el cual contiene
tutoriales, referencias y conjuntos de datos.

\subsection{AlphaMiner}

AlphaMiner es desarrollado por el E-Business Technology Institute (ETI) de la Universidad de Hong Kong \cite{eti}
bajo el apoyo del Fondo para la Innovaci\'on y la Tecnolog\'ia (ITF) \cite{itf} del Gobierno de la Regi\'on
Especial Admi\-nistrativa de Hong Kong (HKSAR).\\

AlphaMiner es un sistema de Miner\'ia de Datos, Open Source, desarro\-llado en java de prop\'osito general pero
m\'as enfocado al ambiente  empresa\-rial. Implementa algoritmos de asociaci\'on, clasificaci\'on, clustering y
regresi\'on log\'istica. Posee una gama amplia de funcionalidad para el usuario, al realizar cualquier proceso
minero dando la posibilidad al usuario de escoger los pasos del proceso KDD que m\'as se ajusten a sus
necesidades, por medio de nodos que el analista integra a un \'arbol KDD siguiendo la metodolog\'ia Drag \& Drop. 
Es por eso que el proceso KDD en AlphaMiner no sigue un orden estructurado, sino que sigue la secuencia  brindada
por el prop\'osito u objetivo  del analista, ademas brinda la visualicion estad\'istica de distintas maneras, 
permitiendo un analisis m\'as preciso por parte del analista.\\

El principal objetivo de AlphaMiner es la inteligencia Comercial Econ\'omica o Inteligencia de Negocios (BI -
Business Inteligence) es uno de los  medios m\'as importantes para que las compa\~n\'ias tomen las decisiones
comerciales m\'as acertadas. Las soluciones de BI son costosas y s\'olo empresas grandes pueden permitirse el
lujo de tenerlas. Por tanto las compa\~n\'ias peque\~nas tienen una gran desventaja. AlphaMiner proporciona las
tecnolog\'ias de BI de forma econ\'omica para dichas empresas para que den soporte a sus decisiones en el
ambiente de negocio cambiante r\'apido.\\
   
AlphaMiner tiene dos componentes principales:
\begin{enumerate}
\item Una base de conocimiento.
\item Un \'arbol KDD, que permite integrar nodos artefacto que proporcionan varias funciones para crear, revisar,
anular, interpretar y argumentar los distintos an\'alisis de datos
\end{enumerate}

AlphaMiner implementa los siguientes pasos del proceso KDD:
\begin{enumerate}
\item Acceso de diferentes formas a las fuentes datos.
\item Exploraci\'on de datos de diferentes maneras.
\item Preparaci\'on de datos.
\item Vinculaci\'on de los distintos modelos mineros.
\item An\'alisis a partir de los  modelos.
\item Despliegue de modelos al ambiente  empresarial.   
\end{enumerate}

La caracter\'istica m\'as importante de AlphaMiner, es su capacidad para almacenar los datos despu\'es del
n\'ucleo minero en una base de conocimiento, que puede ser reutilizada. Esta funci\'on aumenta su utilidad
significativamente, y  brinda un gran apoyo log\'istico al nivel estrat\'egico de una empresa. Aventajando
cualquier sistema tradicional de manipulaci\'on de datos. AlphaMiner proporciona una funcionalidad adicional para
construir los modelos de Miner\'ia de Datos, conformando una sinergia entre los distintos algoritmos de
miner\'ia.\\

AlphaMiner se asemeja a Tariy en varias caracter\'isticas, por un lado el lenguaje de programaci\'on en el que
fue implementado es JAVA, por otro lado es Open Source, tiene conectividad JDBC a los distintos gestores de bases
de datos, adem\'as de conectividad con archivos de Excel. Otra semejanza es el tipo de formato que presenta para
el manejo de  tablas un\'ivaluadas en algoritmos de asociaci\'on, espec\'ificamente en bases de datos de
supermercados, ya que despu\'es de escoger los campos de dicha base de datos se la lleva a un formato de dos
columnas una para la correspondiente transacci\'on y otra para el \'item.

\subsection{YALE - Yet Another Learning Environment}

YALE \cite{yalew, yalep} es un ambiente para la realizaci\'on de experimentos de aprendizaje maquinal y Miner\'ia
de Datos.  A trav\'es de un gran n\'umero de operadores se pueden crear los experimentos, los cuales pueden ser
anidados arbitrariamente y cuya configuraci\'on es descrita a trav\'es de archivos XML que pueden ser
f\'acilmente creados por medio de una interfaz gr\'afica de usuario.  Las aplicaciones de YALE cubren tanto
investigaci\'on como tareas de Miner\'ia de Datos del mundo real.\\

Desde el a\~no 2001 YALE ha sido desarrollado en la Unidad de Inteligencia Artificial de la Universidad de
Dortmund \cite{dort} en conjunto con el Centro de Investigaci\'on Colaborativa en Inteligencia Computacional
(Sonderforschungsbereich 531).\\

El concepto de operador modular permite el dise\~no de cadenas complejas de operadores anidados para el
desarrollo de un gran n\'umero de problemas de aprendizaje. El manejo de los datos es transparente para los
operadores. Ellos no tienen nada que ver con el formato de datos o con las diferentes presentaciones de los
mismos. El kernel de Yale se hace a cargo de las transformaciones necesarias.  Yale es ampliamente usada por
investigadores y compa\~n\'ias de Miner\'ia de Datos.

\subsubsection{Modelando Procesos De Descubrimiento De Conocimiento Como Arboles De Operadores}

Los procesos de descubrimiento de conocimiento son vistos frecuentemente como invocaciones secuenciales de
m\'etodos simples.  Por ejemplo, des\-pu\'es de cargar datos, uno podr\'ia aplicar un paso de preprocesamiento
seguido de una invocaci\'on a un m\'etodo de clasificaci\'on.  El resultado en este caso es modelo aprendido,
el cual puede ser aplicado a datos nuevos o no revisados todav\'ia.  Una posible abstracci\'on de estos m\'etodos
simples es el concepto de operadores. Cada operador recibe su entrada, desempe\~na una determinada funci\'on y
entrega una salida. Desde ahi, el m\'etodo secuencial de invocaciones corresponde a una cadena de operadores. 
Aunque este modelo de cadena es suficiente para la realizaci\'on de muchas tareas b\'asicas de descubrimiento de
conocimiento,  estas cadenas planas son a menudo insuficientes para el mo\-delado de procesos de descubrimiento de
conocimiento m\'as complejas.\\

Una aproximaci\'on com\'un para la realizaci\'on del dise\~no de experimentos m\'as complejos es dise\~nar las
combinaciones de operadores como un gr\'afico direccionado.  Cada vertice del gr\'afico corresponde a un operdor
simple. Si dos operadores son conectados, la salida del primer operador ser\'a usada como entrada en el segundo
operador.  Por un lado, dise\~nar procesos de des\-cubrimiento de conocimiento con la ayuda de
gr\'aficosdireccionados es muy poderoso.  Por el otro lado, existe una desventaja principal: debido a la
p\'erdida de restricciones y la necesidad de ordenar topol\'ogicamente el dise\~no de experimentos es menudo poco
intuitivo y las validaciones autom\'aticas son dif\'iciles de hacer.\\

Yale ofrece un compromiso entre la simplicidad de cadenas de operadores y la potencia de los gr\'aficos
direccionados a trav\'es del modelamiento de procesos de descubrimiento de conocimiento por medio de \'arboles de
operadores. Al igual que los lenguajes de programaci\'on , el uso de \'arboles de operadores permite el uso de
conceptos como ciclos, condiciones, u otros esquemas de aplicaci\'on.  Las hojas en el \'arbol de operadores
corresponden a pasos senci\-llos en el proceso modelado como el aprendizaje de un modelo de predicci\'on \'o la
aplicaci\'on de un filtro de preprocesamiento. Los nodos interiores del \'arbol corresponden a m\'as complejos o
abstractos pasos en el proceso.  Este es a menudo necesario si los hijos deben ser aplicados varias veces como,
por ejemplo, los ciclos.  En general, los nodos de los operadores internos definen el flujo de datos a trav\'es
de sus hijos. La raiz del \'arbol corresponde al experimento completo.\\

\textbf{Qu\'e Puede Hacer YALE?} YALE provee mas de 200 operadores incluyendo:

\begin{description}
\item [Algoritmos de aprendizaje maquinal] Un gran n\'umero de esquemas de aprendizaje para tareas de regresi\'on
y clasificaci\'on incluyendo M\'aquinas para el Soporte de Vectores (SVM), \'arboles de decisi\'on e inductores
de reglas, operadores Lazy, operadores Bayesianos y operadores Log\'isticos. Varios operadores para la miner\'ia
de reglas de asociaci\'on y Clustering son tambi\'en parte de YALE.  Adem\'as, se adicionaron varios esquemas de
Meta Aprendizaje.

\item [Operadores WEKA] Todos los esquemas de aprendizaje y evaluadores de atributos del ambiente de aprendizaje
WEKA tambi\'en est\'an disponibles y pueden ser usados como cualquier otro operador de YALE.

%%%%%%%%%%%%%%%%%%%%%%%%%%%%%% CONCEPTOS PRELIMINARES DE LA IMPLEMENTACION %%%%%%%%%%%%%%%%%%%%%%%%%%%%%%%%
\section{Conceptos Preliminares durante la Implementaci\'on de TariyKDD}
Durante la implementaci\'on del proyecto TariyKDD fueron utilizadas una serie de herramientas para su desarrollo.  Entre las principales de ellas podemos nombrar el lenguaje de programaci\'on Java, el entorno de desarrollo NetBeans y la biblioteca gr\'afica Swing para Java.\\

A continuaci\'on se explican brevemente los conceptos preliminares de estas herramientas para ser tenidas en cuenta en la posterior descripci\'on de la implementaci\'on que se realiza en este cap\'itulo.

\subsection{Lenguaje de programaci\'on Java}
Java es un lenguaje de programaci\'on orientado a objetos desarrollado por James Gosling y sus compa\~neros de Sun Microsystems al inicio de la d\'ecada de 1990. A diferencia de los lenguajes de programaci\'on convencionales, que generalmente est\'an dise\~nados para ser compilados a c\'odigo nativo, Java es compilado en un bytecode que es ejecutado (usando normalmente un compilador JIT), por una m\'aquina virtual Java.\\

El lenguaje Java se cre\'o con cinco objetivos principales:

\begin{enumerate}
 \item Deber\'ia usar la metodolog\'ia de la programaci\'on orientada a objetos.
 \item Deber\'ia permitir la ejecuci\'on de un mismo programa en m\'ultiples sistemas operativos. 
 \item Deber\'ia incluir por defecto soporte para trabajo en red.
 \item Deber\'ia dise\~narse para ejecutar c\'odigo en sistemas remotos de forma segura.
 \item Deber\'ia ser f\'acil de usar y tomar lo mejor de otros lenguajes orientados a objetos, como C++.
\end{enumerate}

\subsubsection{Caracter\'isticas Principales}
\paragraph{Orientado a Objetos}
La primera caracter\'istica, orientado a objetos (''OO''), se refiere a un m\'etodo de programaci\'on y al
dise\~no del lenguaje. Aunque hay muchas interpretaciones para OO, una primera idea es dise\~nar el software de
forma que los distintos tipos de datos que use est\'en unidos a sus operaciones. As\'i, los datos y el c\'odigo
(funciones o m\'etodos) se combinan en entidades llamadas objetos. Un objeto puede verse como un paquete que
contiene el ''comportamiento'' (el c\'odigo) y el ''estado'' (datos).\\

El principio es separar aquello que cambia de las cosas que permanecen inalterables. Frecuentemente, cambiar una estructura de datos implica un cambio en el c\'odigo que opera sobre los mismos, o viceversa. Esta separaci\'on en objetos coherentes e independientes ofrece una base m\'as estable para el dise\~no de un sistema software. El objetivo es hacer que grandes proyectos sean f\'aciles de gestionar y manejar, mejorando como consecuencia su calidad y reduciendo el n\'umero de proyectos fallidos. Otra de las grandes promesas de la programaci\'on orientada a objetos es la creaci\'on de entidades m\'as gen\'ericas (objetos) que permitan la reutilizaci\'on del software entre proyectos, una de las premisas fundamentales de la Ingenier\'ia del Software. Un objeto gen\'erico ''cliente'', por ejemplo, deber\'ia en teor\'ia tener el mismo conjunto de comportamiento en diferentes proyectos, sobre todo cuando estos coinciden en cierta medida, algo que suele suceder en las grandes organizaciones. En este sentido, los objetos podr\'ian verse como piezas reutilizables que pueden emplearse en m\'ultiples proyectos distintos, posibilitando as\'i a la industria del software a construir proyectos de envergadura empleando componentes ya existentes y de comprobada calidad; conduciendo esto finalmente a una reducci\'on dr\'astica del tiempo de desarrollo. Podemos usar como ejemplo de objeto el aluminio. Una vez definidos datos (peso, maleabilidad, etc.), y su ''comportamiento'' (soldar dos piezas, etc.), el objeto ''aluminio'' puede ser reutilizado en el campo de la construcci\'on, del autom\'ovil, de la aviaci\'on, etc.

\paragraph{Independencia de la plataforma}
La segunda caracter\'istica, la independencia de la plataforma, significa que programas escritos en el lenguaje Java pueden ejecutarse igualmente en cualquier tipo de hardware. Es lo que significa ser capaz de escribir un programa una vez y que pueda ejecutarse en cualquier dispositivo, tal como reza el axioma de Java, ''write once, run everywhere''.\\

Para ello, se compila el c\'odigo fuente escrito en lenguaje Java, para generar un c\'odigo conocido como ''bytecode'' (espec\'ificamente Java bytecode) que son instrucciones m\'aquina simplificadas espec\'ificas de la plataforma Java. Esta pieza est\'a ''a medio camino'' entre el c\'odigo fuente y el c\'odigo m\'aquina que entiende el dispositivo destino. El bytecode es ejecutado entonces en la m\'aquina virtual (VM), un programa escrito en c\'odigo nativo de la plataforma destino (que es el que entiende su hardware), que interpreta y ejecuta el c\'odigo. Adem\'as, se suministran librer\'ias adicionales para acceder a las caracter\'isticas de cada dispositivo (como los gr\'aficos, ejecuci\'on mediante hebras o threads, la interfaz de red) de forma unificada. Se debe tener presente que, aunque hay una etapa expl\'icita de compilaci\'on, el bytecode generado es interpretado o convertido a instrucciones m\'aquina del c\'odigo nativo por el compilador JIT (Just In Time).

\paragraph{El recolector de basura}
Un argumento en contra de lenguajes como C++ es que los programadores se encuentran con la carga a\~nadida de tener que administrar la memoria de forma manual. En C++, el desarrollador debe asignar memoria en una zona conocida como heap (mont\'iculo) para crear cualquier objeto, y posteriormente desalojar el espacio asignado cuando desea borrarlo. Un olvido a la hora de desalojar memoria previamente solicitada, o si no lo hace en el instante oportuno, puede llevar a una fuga de memoria, ya que el sistema operativo piensa que esa zona de memoria est\'a siendo usada por una aplicaci\'on cuando en realidad no es as\'i. As\'i, un programa mal dise\~nado podr\'ia consumir una cantidad desproporcionada de memoria. Adem\'as, si una misma regi\'on de memoria es desalojada dos veces el programa puede volverse inestable y llevar a un eventual cuelgue.\\

En Java, este problema potencial es evitado en gran medida por el recolector autom\'atico de basura (o automatic garbage collector). El programador determina cu\'ando se crean los objetos y el entorno en tiempo de ejecuci\'on de Java (Java runtime) es el responsable de gestionar el ciclo de vida de los objetos. El programa, u otros objetos pueden tener localizado un objeto mediante una referencia a \'este (que, desde un punto de vista de bajo nivel es una direcci\'on de memoria). Cuando no quedan referencias a un objeto, el recolector de basura de Java borra el objeto, liberando as\'i la memoria que ocupaba previniendo posibles fugas (ejemplo: un objeto creado y \'unicamente usado dentro de un m\'etodo s\'olo tiene entidad dentro de \'este; al salir del m\'etodo el objeto es eliminado). A\'un as\'i, es posible que se produzcan fugas de memoria si el c\'odigo almacena referencias a objetos que ya no son necesarios es decir, pueden a\'un ocurrir, pero en un nivel conceptual superior. En definitiva, el recolector de basura de Java permite una f\'acil creaci\'on y eliminaci\'on de objetos, mayor seguridad y frecuentemente m\'as r\'apida que en C++.\\

La recolecci\'on de basura de Java es un proceso pr\'acticamente invisible al desarrollador. Es decir, el programador no tiene conciencia de cu\'ando la recolecci\'on de basura tendr\'a lugar, ya que \'esta no tiene necesariamente que guardar relaci\'on con las acciones que realiza el c\'odigo fuente.\\

Debe tenerse en cuenta que la memoria es s\'olo uno de los muchos recursos que deben ser gestionados.\\

Durante la implementaci\'on de este proyecto se utilizo la versi\'on JDK1.5.0 para arquitecturas AMD64 en su actualizaci\'on n\'umero 9.

\paragraph{Entorno de Desarrollo NetBeans}
NetBeans se refiere a una plataforma para el desarrollo de aplicaciones de escritorio usando Java y a un Entorno integrado de desarrollo (IDE) desarrollado usando la Plataforma NetBeans.\\

La plataforma NetBeans permite que las aplicaciones sean desarrolladas a partir de un conjunto de componentes de software llamados m\'odulos. Un m\'odulo es un archivo Java que contiene clases de java escritas para interactuar con las APIs de NetBeans y un archivo especial (manifest file) que lo identifica como m\'odulo. Las aplicaciones construidas a partir de m\'odulos pueden ser extendidas agreg\'andole nuevos m\'odulos. Debido a que los m\'odulos pueden ser desarrollados independientemente, las aplicaciones basadas en la plataforma NetBeans pueden ser extendidas f\'acilmente por otros desarrolladores de software.\\

La Plataforma NetBeans es un framework reusable que simplifica el desarrollo de otras aplicaciones de escritorio. Cuando se ejecuta una aplicaci\'on basada en la Plataforma NetBeans, la plataforma ejecuta la clase Main. Los m\'odulos disponibles est\'an localizados y puestos en un registro en memoria, y son ejecutadas las tareas de inicializaci\'on de los m\'odulos. Generalmente el c\'odigo de un m\'odulo es cargado en memoria solo cuando se necesita.\\

La plataforma ofrece servicios comunes a las aplicaciones de escritorio, permiti\'endole al desarrollador enfocarse en la l\'ogica espec\'ifica de su aplicaci\'on. Entre las caracter\'isticas de la plataforma est\'an:

\begin{enumerate}
 \item Administraci\'on de las interfases de usuario (ej. men\'us y barras de herramientas).

 \item Administraci\'on de las configuraciones del usuario.

 \item Administraci\'on del almacenamiento (guardando y cargando cualquier tipo de dato).

 \item Administraci\'on de ventanas.

 \item Framework basado en asistentes (di\'alogos paso a paso).
\end{enumerate}

\paragraph{Controlador JDBC}
JDBC es el acr\'onimo de Java Database Connectivity, un API que permite la ejecuci\'on de operaciones sobre bases de datos desde el lenguaje de programaci\'on Java independientemente del sistema de operaci\'on donde se ejecute o de la base de datos a la cual se accede utilizando el dialecto SQL del modelo de base de datos que se utilice.\\

El API JDBC se presenta como una colecci\'on de interfaces Java y m\'etodos de gesti\'on de manejadores de conexi\'on hacia cada modelo espec\'ifico de base de datos. Un manejador de conexiones hacia un modelo de base de datos en particular es un conjunto de clases que implementan las interfaces Java y que utilizan los m\'etodos de registro para declarar los tipos de localizadores a base de datos (URL) que pueden manejar.  Para utilizar una base de datos particular, el usuario ejecuta su programa junto con la librer\'ia de conexi\'on apropiada al modelo de su base de datos, y accede a ella estableciendo una conexi\'on, para ello provee en localizador a la base de datos y los par\'ametros de conexi\'on espec\'ificos.  A partir de all\'i puede realizar cualquier tipo de tareas con la base de datos a las que tenga permiso: consultas, actualizaciones, creado modificado y borrado de tablas, ejecuci\'on de procedimientos almacenados en la base de datos, etc.\\

Cada base de datos emplea un protocolo diferente de comunicaci\'on, protocolos que normalmente son propietarios. El uso de un manejador, una capa intermedia entre el c\'odigo del desarrollador y la base de datos, permite independizar el c\'odigo Java que accede a la BD del sistema de BD concreto a la que estamos accediendo, ya que en nuestro c\'odigo Java emplearemos comandos est\'andar, y estos comandos ser\'an traducidos por el manejador a comandos propietarios de cada sistema de BD concreto.  Si queremos cambiar el sistema de BD que empleamos lo \'unico que deberemos hacer es reemplazar el antiguo manejador por el nuevo, y seremos capaces de conectarnos la nueva BD.  Durante el desarrollo de este proyecto se utilizo un controlador JDBC tipo 4 para el SGDB PostgreSQL.

\subparagraph{Manejador de Protocolo Nativo (tipo 4)}
El manejador traduce directamente las llamadas al API JDBC al protocolo nativo de la base de datos. Es el manejador que tiene mejor rendimiento, pero est\'a m\'as ligado a la base de datos que empleemos que el manejador tipo JDBC-Net, donde el uso del servidor intermedio nos da una gran flexibilidad a la hora de cambiar de base de datos.  Este tipo de manejadores tambi\'en emplea tecnolog\'ia 100\% Java.

\begin{figure}
\centering
\includegraphics[width=1\textwidth]{images/cp001.png}
\caption{Controlador JDBC tipo 4}
\label{cp001}
\end{figure}

\paragraph{Biblioteca gr\'afica Swing de Java}
Swing es una biblioteca gr\'afica para Java que forma parte de las Java Foundation Classes (JFC). Incluye componentes para interfaz gr\'afica de usuario tales como cajas de texto, botones, listas desplegables y tablas.\\

Las Internet Foundation Classes (IFC) eran una biblioteca gr\'afica para el lenguaje de programaci\'on Java desarrollada originalmente por Netscape y que se public\'o en 1996.  En 1997, Sun Microsystems y Netscape Communications Corporation anunciaron su intenci\'on de combinar IFC con otras tecnolog\'ias de las Java Foundation Classes. Adem\'as de los componentes ligeros suministrados originalmente por la IFC, Swing introdujo un mecanismo que permit\'ia que el aspecto de cada componente de una aplicaci\'on pudiese cambiar sin introducir cambios sustanciales en el c\'odigo de la aplicaci\'on. La introducci\'on de soporte ensamblable para el aspecto permiti\'o a Swing emular la apariencia de los componentes nativos manteniendo las ventajas de la independencia de la plataforma.\\

Algunos de los componentes utilizados en el desarrollo de este proyecto y pertenecientes a la biblioteca Swing se explican a continuaci\'on:

\subparagraph{JComponent}
La clase base para todos los componentes de Swing exceptuando los contenedores de nivel superior.  Para utilizar un componente que herede de JComponent, se debe poner el componente en una jerarqu\'ia de la contenci\'on cuya ra\'iz sea un contenedor de nivel superior Swing tal como Ventanas o Paneles.  Los contenedores de nivel superior son los componentes especializados que proporcionan un lugar para que otros componentes de Swing puedan pintarse.

\subparagraph{JFrame}
Un JFrame es una ventana de nivel superior con un t\'itulo y un borde. El tama\~no del JFrame incluye cualquier \'area se\~nalada por los bordes donde puede ser incluido un o mas componentes de Swing.

\subparagraph{JPanel}
El JPanel es la clase m\'as simple del contenedor. Un JPanel proporciona el espacio en el cual una aplicaci\'on puede unir cualquier otro componente, incluyendo otros paneles.

\subparagraph{JTabbedPane}
Un componente que deja a usuario cambiar entre un grupo de componentes pulsando en una etiqueta con un t\'itulo dado y/o un icono.  Las etiquetas y los componentes son agregados a un objeto de TabbedPane usando los m\'etodos del addTab e insertTab.  Una etiqueta es representada por un \'indice que corresponde a la posici\'on que fue agregado adentro, donde la primera etiqueta tiene un \'indice igual a 0 y la etiqueta pasada tiene un \'indice igual a la cuenta de la etiqueta menos 1.

\subparagraph{JSplitPane}
JSplitPane se utiliza para dividir dos (y solamente dos) componentes. Los dos componentes se pueden entonces volver a clasificar seg\'un el tama\~no rec\'iprocamente por el usuario.  La informaci\'on sobre como usar JSplitPane est\'a en c\'omo utilizar los paneles partidos en la clase particular de Java.   Se pueden incluir nuevos componentes en cada una de las divisiones de un JSplitPanne.  Los dos componentes en un panel partido se pueden alinear a la izquierda y a la derecha usando JSplitPane.HORIZONTAL\_SPLIT, o en la parte superior o inferior de la ventana con JSplitPane.VERTICAL\_SPLIT.

\subparagraph{JScrollPanne}
JScrollPanne proporciona una vista deslizante de un componente ligero. Un JScrollPane maneja un viewport, barras de desplazamiento verticales y horizontales opcionales, y viewports opcionales del t\'itulo de la fila y de columna.  El viewport, o punto de vista, proporciona una ventanasobre una fuente de datos, por ejemplo, un archivo de texto o una imagen sobre la cual se va ha deslizar.

\subparagraph{JFileChooser}
Los seleccionadores de archivos proporcionan una interfaz gr\'afica de usuario para navegar el sistema de ficheros, y despu\'es elegir un archivo o un directorio de una lista o incorporar el nombre de un archivo o un directorio.  Para exhibir un seleccionador  de archivo, se utiliza generalmente el JFileChooser para mostrar un di\'alogo modal que contiene el seleccionador de archivo.  Otra manera de presentar un seleccionador es agregar una instancia de JFileChooser a un contenedor.

\subparagraph{JTable}
Con la clase de JTable se puede exhibir una tabla de datos, permitiendo opcionalmente que el usuario edite el contenido de las celdas que contienen los datos.  JTable no contiene ni deposita datos; es simplemente una vista de los datos.

\subparagraph{JTree}
Con la clase JTree, puedes exhibir datos jer\'arquicos.  Un objeto JTree no contiene realmente los datos sino que proporciona simplemente una vista de ellos.  Como cualquier componente no trivial del Swing, el \'arbol consigue los datos conect\'andose a un modelo de los datos.

\subparagraph{JTextArea}
Un JTextArea es un \'area multil\'inea que exhibe el texto plano y que permite opcionalmente que el usuario corrija el texto. Este es un componente ligero que provee de compatibilidad a la hora de introduccion o desplejar peque\~nas cantidades de informaci\'on.

\subparagraph{JSpinner}
Esta clase provee en una sola l\'inea la posibilidad al usuario de entrar o seleccionar un valor de una secuencia pedida. Los JSpinner proporciona t\'ipicamente un par de los botones min\'usculos de  flechas para cambiar entre los elementos de la secuencia.  Las teclas de flecha arriba/abajo del teclado tambi\'en completan un ciclo a trav\'es de los elementos. El usuario puede tambi\'en mecanografiar el valor directamente en un JSpinner.

\subparagraph{JComboBox}
Un componente que combina un bot\'on o un campo editable y una lista desplegable.  El usuario puede seleccionar un valor de la lista, que aparece a petici\'on del usuario.  Si la caja de texto es editable, entonces se incluye un campo en el cual el usuario pueda mecanografiar un valor.

\subparagraph{JRadioButton}
Una puesta en pr\'actica de un bot\'on de radio.  Este es un componente que puede ser seleccionado o deseleccionado, y que exhibe su estado al usuario.  Utilizado con un objeto ButtonGroup crea un grupo de botones en los cuales solamente un bot\'on puede ser seleccionado a la vez.

\end{description}



\chapter{DESARROLLO DEL PROYECTO}
\section{An\'alisis UML}
\subsection{Funciones}
\addtolength{\hoffset}{-1cm}
\addtolength{\voffset}{-1cm}
%\documentclass[letterpaper, 10pt]{report}
%\usepackage[spanish]{babel}
%\usepackage[latin1]{inputenc}
%\usepackage[pdftex]{color, graphicx}
%\pagestyle{plain}

%\addtolength{\hoffset}{-1.5cm}
%\addtolength{\textwidth}{1cm}

%\begin{document}
\begin{center}
% use packages: array
\begin{tabular}{|l|p{4cm}|l|l|p{4cm}|l|}\hline
\textbf{Ref\#} & \textbf{Funci\'on} & \textbf{Cat.} & \textbf{Atributo} & \textbf{Detalles y Restricciones} &
\textbf{Cat.} \\ \hline
R1 & Ejecutar la herramienta. & Evidente. & Interfaz. &  & Obligatorio. \\ \hline
R1.1 & Mostrar interfaz gr\'afica de la herramienta & Evidente. & Interfaz. & Pesta\~nas desplegables. & Obligatorio. \\ \hline
R2 & Mostrar mensajes de ayuda, respuesta o error cuando sea necesario. & Evidente. & Interfaz. & Cuadros de dialogo, barra de estado. & Deseable. \\ \hline
R3 & Permitir el establecimiento de conexiones. & Evidente. & Interfaz. &  & Obligatorio. \\ \hline
R3.1 & Autorizar conexiones a Bases de Datos. & Evidente. & Interfaz. &  & Opcional. \\ \hline
R3.1.1 & Permitir la selecci\'on de atributos. & Evidente. & Interfaz. &  & Obligatoria. \\ \hline
R3.1.2 & Cargar el conjunto de datos. & Oculto. & Informaci\'on & Establecer relaciones correctas & Obligatorio.
\\ \hline
R3.2 & Autorizar conexiones a Archivos Planos. & Evidente. & Interfaz. &  & Opcional. \\ \hline

\end{tabular}

\begin{tabular}{|l|p{4cm}|l|l|p{4cm}|l|}\hline
\textbf{Ref\#} & \textbf{Funci\'on} & \textbf{Cat.} & \textbf{Atributo} & \textbf{Detalles y Restricciones} &
\textbf{Cat.} \\ \hline
R3.2.1 & Recorrer archivo y cargar datos. & Oculto. & Informaci\'on. & El archivo debe cumplir con el formato ARFF. & Obligatorio. \\ \hline
R4 & Permitir la aplicaci\'on de filtros & Evidente & Interfaz &  & Opcional \\ \hline
R4.1 & Permitir remover datos nulos & Evidente & Interfaz &  & Opcional \\ \hline
R4.2 & Permitir actualizar datos nulos & Evidente & Interfaz &  & Opcional \\ \hline
R4.3 & Permitir seleccionar un conjunto de registros & Evidente & Interfaz &  & Opcional \\ \hline
R4.4 & Permitir seleccionar un conjunto de registros seg\'un un atributo dado. & Evidente & Interfaz &  &
Opcional \\ \hline
R4.5 & Permitir reducir el rango de los datos & Evidente & Interfaz &  & Opcional \\ \hline
R4.6 & Permitir codificar los datos & Evidente & Interfaz &  & Opcional \\ \hline
R4.7 & Permitir reemplazar un valor determinado & Evidente & Interfaz &  & Opcional \\ \hline
R4.8 & Permitir seleccionar una muestra del conjunto de datos & Evidente & Interfaz &  & Opcional \\ \hline
R4.9 & Permitir discretizar valores continuos & Evidente & Interfaz &  & Opcional \\ \hline
R5 & Aplicar t\'ecnicas de Miner\'ia de Datos & Evidente & Interfaz &  & Obligatorio \\ \hline
R5.1 & Proveer algoritmos que cumplan tareas de asociaci\'on & Evidente & Interfaz &  & Opcional \\ \hline
R5.1.1 & Implementar algoritmo Apriori & Evidente & Interfaz &  & Opcional \\ \hline
R5.1.2 & Implementar algoritmo FPGrowth & Evidente & Interfaz &  & Opcional \\ \hline
R5.1.3 & Implementar algoritmo EquipAsso & Evidente & Interfaz &  & Opcional \\ \hline

\end{tabular}

\begin{tabular}{|l|p{4cm}|l|l|p{4cm}|l|}\hline
\textbf{Ref\#} & \textbf{Funci\'on} & \textbf{Cat.} & \textbf{Atributo} & \textbf{Detalles y Restricciones} &
\textbf{Cat.} \\ \hline
R5.2 & Proveer algoritmos que cumplan tareas de clasificaci\'on & Evidente & Interfaz &  & Opcional \\ \hline
R5.2.1 & Implementar algoritmo C4.5 & Evidente & Interfaz &  & Opcional \\ \hline
R5.2.2 & Implementar algoritmo MateBy & Evidente & Interfaz &  & Opcional \\ \hline
R6 & Permitir visualizaci\'on de reglas & Evidente & Interfaz &  & Obligatorio \\ \hline
R6.1 & Organizar las reglas de asociaci\'on a trav\'es de una tabla & Evidente & Interfaz &  & Obligatorio \\
\hline
R6.2 & Organizar las reglas de clasificaci\'on a trav\'es de un \'arbol & Evidente & Interfaz &  &
Obligatorio \\ \hline
\end{tabular}
\end{center}

%\end{document}

\subsection{Diagramas de Casos de Uso}
\addtolength{\hoffset}{1cm}
\addtolength{\voffset}{1cm}
\subsubsection{Tariy}
\begin{figure}[h]
 \centering
 \includegraphics[width=0.9\textwidth]{imgsCasosUso/01tariy.png}
 \caption{Diagrama de caso de uso Tariy}
\end{figure}
\newpage

\subsubsection{M\'odulo de Selecci\'on}
\begin{figure}[h]
 \centering
 \includegraphics[width=0.9\textwidth]{imgsCasosUso/02seleccion.png}
 \caption{M\'odulo de Selecci\'on}
\end{figure}

\subsubsection{M\'odulo de Conexi\'on a Base de Datos}
\begin{figure}[h]
 \centering
 \includegraphics[width=0.9\textwidth]{imgsCasosUso/04conexionBD.png}
 \caption{Base de Datos}
\end{figure}
\newpage

\subsubsection{M\'odulo de Conexi\'on a Archivo Plano}
\begin{figure}[h]
 \centering
 \includegraphics[width=0.9\textwidth]{imgsCasosUso/03archivoPlano.png}
 \caption{Archivo Plano}
\end{figure}

\subsubsection{M\'odulo de Preprocesamiento}
\begin{figure}[h]
 \centering
 \includegraphics[width=0.9\textwidth]{imgsCasosUso/05preprocesamiento.png}
 \caption{Preprocesamiento}
\end{figure}
\newpage

\subsubsection{M\'odulo de Miner\'ia de Datos}
\begin{figure}[h]
 \centering
 \includegraphics[width=0.9\textwidth]{imgsCasosUso/06mineria.png}
 \caption{Miner\'ia de Datos}
\end{figure}
\newpage

\subsubsection{M\'odulo de Reglas}
\begin{figure}[h]
 \centering
 \includegraphics[width=0.9\textwidth]{imgsCasosUso/07reglas.png}
 \caption{Reglas}
\end{figure}
\newpage

%%%%%%%%%%%%%%%%%%%%%%%%%%% CLASE APRIORI %%%%%%%%%%%%%%%%%%%%%%%%%%%%%%%%%%%%%%%%%%%
\begin{figure}
\subsection{Diagramas de Secuencia}
\subsubsection{Clase Apriori}
\centering
\includegraphics[width=1.2\textwidth]{imgsSecuencia/Apriori/run.png}
\caption{run}
\end{figure}
\newpage
\begin{figure}
\centering
\includegraphics[width=1.2\textwidth]{imgsSecuencia/Apriori/increaseSupport.png}
\caption{increaseSupport}
\end{figure}
\newpage
\begin{figure}
\centering
\includegraphics[width=0.8\textwidth]{imgsSecuencia/Apriori/CombinationsB.jpg}
\caption{Combinations}
\end{figure}
\newpage
\begin{figure}
\centering
\includegraphics[width=1\textwidth]{imgsSecuencia/Apriori/makeCandidates.png}
\caption{makeCandidates}
\end{figure}
\newpage
\begin{figure}
\centering
\includegraphics[angle=90, width=0.5\textwidth]{imgsSecuencia/Apriori/pruneCandidates.png}
\caption{pruneCandidates}
\end{figure}
\newpage

%%%%%%%%%%%%%%%%%%%%%%%%%% CLASE EQUIPASSO %%%%%%%%%%%%%%%%%%%%%%%%%%%%%%%%%%%%%%%%%%%%
\begin{figure}
\subsubsection{Clase EquipAsso}
\centering
\includegraphics[angle=90, width=0.8\textwidth]{imgsSecuencia/EquipAsso/findInDataSet.png}
\caption{findInDataSet}
\end{figure}
\newpage
\begin{figure}
\centering
\includegraphics[width=0.8\textwidth]{imgsSecuencia/EquipAsso/pruneCandidate-recursive.png}
\caption{pruneCandidate-recursive}
\end{figure}
\newpage
\begin{figure}
\centering
\includegraphics[width=1.2\textwidth]{imgsSecuencia/EquipAsso/pruneCandidates.png}
\caption{pruneCandidate-recursive}
\end{figure}
\newpage
\begin{figure}
\centering
\includegraphics[width=1.2\textwidth]{imgsSecuencia/EquipAsso/run.png}
\caption{run}
\end{figure}
\newpage

%%%%%%%%%%%%%%%%%%%%%%%% CLASE ITEMSET %%%%%%%%%%%%%%%%%%%%%%%%%%%%%%%%%%%%%
\begin{figure}
\subsubsection{Clase ItemSet}
\centering
\includegraphics[width=0.7\textwidth]{imgsSecuencia/ItemSet/addItems.png}
\caption{addItems}
\end{figure}
\newpage
\begin{figure}
\centering
\includegraphics[width=0.7\textwidth]{imgsSecuencia/ItemSet/sortItems.png}
\caption{sortItems}
\end{figure}
\newpage
\begin{figure}
\centering
\includegraphics[angle=90, width=0.5\textwidth]{imgsSecuencia/ItemSet/compare.png}
\caption{compare}
\end{figure}
\newpage

%%%%%%%%%%%%%%%%%%%%%%%% CLASE DATASET %%%%%%%%%%%%%%%%%%%%%%%%%%%%%%%%%%%%%%%%%
\begin{figure}
\subsubsection{Clase DataSet}
\centering
\includegraphics[width=1\textwidth]{imgsSecuencia/DataSet/buildDictionary.png}
\caption{buildDictionary}
\end{figure}
\newpage
\begin{figure}
\centering
\includegraphics[angle=90, width=0.6\textwidth]{imgsSecuencia/DataSet/buildNTree.png}
\caption{buildNTree}
\end{figure}
\newpage
\begin{figure}
\centering
\includegraphics[width=1\textwidth]{imgsSecuencia/DataSet/findAttrName.png}
\caption{findAttrName}
\end{figure}
\newpage
\begin{figure}
\centering
\includegraphics[angle=90, width=0.5\textwidth]{imgsSecuencia/DataSet/pruneCandidatesOne.png}
\caption{pruneCandidatesOne}
\end{figure}
\newpage
\begin{figure}
\centering
\includegraphics[width=1.2\textwidth]{imgsSecuencia/DataSet/pruneCandidatesOne_public.png}
\caption{public-pruneCandidatesOne}
\end{figure}
\newpage

% \documentclass[letterpaper,12pt]{report}
% \usepackage[pdftex]{graphicx}
% \usepackage[spanish]{babel}
% \usepackage[latin1]{inputenc}
% \pagestyle{plain}
% 
% \begin{document}
% \addtolength{\textwidth}{-3cm}
% \begin{figure}
% \chapter{UML}

%%%%%%%%%%%%%%%%%%%%%%%%%% CLASE FPGROWTH %%%%%%%%%%%%%%%%%%%%%%%%%%%%%%%%%%%%%%%%%%%%
\begin{figure}
\subsubsection{Clase FPGrowth}
\centering
\includegraphics[width=1.2\textwidth]{imgsSecuencia/FPGrowth/findNode.png}
\caption{buildFrequentsNodes}
\end{figure}
\newpage
\begin{figure}
\centering
\includegraphics[width=0.7\textwidth]{imgsSecuencia/FPGrowth/buildFrequentsNodes.png}
\caption{findNode}
\end{figure}
\newpage
\begin{figure}
\centering
\includegraphics[width=1.2\textwidth]{imgsSecuencia/FPGrowth/run.png}
\caption{run}
\end{figure}
\newpage

%%%%%%%%%%%%%%%%%%%%%%%%% CLASE AVLTREE %%%%%%%%%%%%%%%%%%%%%%%%%%%%%%%%%%%%%%%%%%%%%%%%%
\begin{figure}
\subsubsection{Clase AvlTree}
\includegraphics[width=1.2\textwidth]{imgsSecuencia/AvlTree/compareItemSet.png}
\caption{compareItemSet}
\end{figure}
\newpage

\begin{figure}
\includegraphics[width=1.2\textwidth]{imgsSecuencia/AvlTree/find.png}
\caption{find}
\end{figure}
\newpage


\begin{figure}
\centering
\includegraphics[width=0.8\textwidth]{imgsSecuencia/AvlTree/doubleWithRightChild.png}
\caption{doubleWithRightChild}
\end{figure}
\newpage


\begin{figure}
\centering
\includegraphics[width=0.8\textwidth]{imgsSecuencia/AvlTree/doubleWithLeftChild.png}
\caption{doubleWithLeftChild}
\end{figure}
\newpage

%%%%%%%%%%%%%%%%%%%%%%%%% CLASE TRANSACTION %%%%%%%%%%%%%%%%%%%%%%%%%%%%%%%%%%%%%%%%%%%%%%%%%

\begin{figure}
\subsubsection{Clase Transaction}
\centering
\includegraphics[angle=90, width=0.6\textwidth]{imgsSecuencia/Transaction/loadItemsets2.png}
\caption{loadItemset}
\end{figure}
\newpage

\begin{figure}
\centering
\includegraphics[angle=90, width=0.8\textwidth]{imgsSecuencia/Transaction/loadItemsets.png}
\caption{public-loadItemset}
\end{figure}
\newpage

\begin{figure}
\centering
\includegraphics[width=1\textwidth]{imgsSecuencia/Transaction/sortByItem.png}
\caption{sortByItem}
\end{figure}
\newpage

\begin{figure}
\centering
\includegraphics[width=1\textwidth]{imgsSecuencia/Transaction/sortBySupport.png}
\caption{sortBySupport}
\end{figure}
\newpage

%%%%%%%%%%%%%%%%%%%%%%%%% CLASE NODEF %%%%%%%%%%%%%%%%%%%%%%%%%%%%%%%%%%%%%%%%%%%%%%%%%

\begin{figure}
\subsubsection{Clase NodeF}
\centering
\includegraphics[width=1\textwidth]{imgsSecuencia/NodeF/getBranch.png}
\caption{getBranch}
\end{figure}
\newpage

\begin{figure}
\centering
\includegraphics[width=1\textwidth]{imgsSecuencia/NodeF/getPath.png}
\caption{getPath}
\end{figure}
\newpage

%%%%%%%%%%%%%%%%%%%%%%%%% CLASE NODENOF  %%%%%%%%%%%%%%%%%%%%%%%%%%%%%%%%%%%%%%%%%%%%%%%%%

\begin{figure}
\subsubsection{Clase NodeNoF}
\centering
\includegraphics[angle=90, width=0.6\textwidth]{imgsSecuencia/NodeNoF/findBro.png}
\caption{findBro}
\end{figure}
\newpage

\begin{figure}
\centering
\includegraphics[width=1\textwidth]{imgsSecuencia/NodeNoF/getChild.png}
\caption{getChild}
\end{figure}
\newpage

\begin{figure}
\centering
\includegraphics[width=1\textwidth]{imgsSecuencia/NodeNoF/getIndexOfChild.png}
\caption{getIndexOfChild}
\end{figure}
\newpage
% 
% \end{document}

\documentclass[letterpaper,12pt]{report}
\usepackage[pdftex]{graphicx}
\usepackage[spanish]{babel}
\usepackage[latin1]{inputenc}
\pagestyle{plain}

\begin{document}
\addtolength{\textwidth}{-3cm}
\begin{figure}
\chapter{UML}

%%%%%%%%%%%%%%%%%%%%%%%%%% CLASE  C45 %%%%%%%%%%%%%%%%%%%%%%%%%%%%%%%%%%%%%%%%%%%%

\section{Clase C45}
\centering
\includegraphics[width=1.2\textwidth]{c45/C45Clearequalsparcializados.png}
\caption{C45Clearequalsparcializados}
\end{figure}
\newpage
\begin{figure}
\centering
\includegraphics[angle=90, width=1\textwidth]{c45/C45Rules.png}
\caption{C45Rules}
\end{figure}
\newpage


%%%%%%%%%%%%%%%%%%%%%%%%%% CLASE  myHasMap %%%%%%%%%%%%%%%%%%%%%%%%%%%%%%%%%%%%%%%%%%%%
\begin{figure}
\section{Clase myHasMap}
\centering
\includegraphics[angle=90, width=0.8\textwidth]{myHasMap/addColumn.png}
\caption{addColumn}
\end{figure}
\newpage
\begin{figure}
\centering
\includegraphics[angle=90, width=0.8\textwidth]{myHasMap/SearchColumn.png}
\caption{SearchColumn}
\end{figure}
\newpage


%%%%%%%%%%%%%%%%%%%%%%%%%% CLASE  Route %%%%%%%%%%%%%%%%%%%%%%%%%%%%%%%%%%%%%%%%%%%%

\begin{figure}
\section{Clase Route}
\centering
\includegraphics[width=1\textwidth]{Route/avanceVariable.png}
\caption{avanceVariable}
\end{figure}
\newpage
\begin{figure}
\centering
\includegraphics[width=1\textwidth]{Route/getIndex.png}
\caption{getIndex}
\end{figure}
\newpage
\begin{figure}
\centering
\includegraphics[width=1\textwidth]{Route/setPoscicionVariable.png}
\caption{setPoscicionVariable}
\end{figure}
\newpage

%%%%%%%%%%%%%%%%%%%%%%%%%% CLASE  TreeCounter %%%%%%%%%%%%%%%%%%%%%%%%%%%%%%%%%%%%%%%%%%%%

\begin{figure}
\section{Clase TreeCounter}
\centering
\includegraphics[angle=90, width=0.8\textwidth]{TreeCounter/createTree.png}
\caption{createTree}
\end{figure}
\newpage
\begin{figure}
\centering
\includegraphics[width=1\textwidth]{TreeCounter/gananciaInicial.png}
\caption{firstGain}
\end{figure}
\newpage
\begin{figure}
\centering
\includegraphics[angle=90, width=0.7\textwidth]{TreeCounter/ganancia.png}
\caption{gain}
\end{figure}
\newpage
\begin{figure}
\centering
\includegraphics[width=1\textwidth]{TreeCounter/searchAttribute.png}
\caption{searchAttribute}
\end{figure}
\newpage
\begin{figure}
\centering
\includegraphics[width=1\textwidth]{TreeCounter/seeTree.png}
\caption{seeTree}
\end{figure}
\newpage

%%%%%%%%%%%%%%%%%%%%%%%%%% CLASE  Attribute %%%%%%%%%%%%%%%%%%%%%%%%%%%%%%%%%%%%%%%%%%%%

\begin{figure}
\section{Clase Attribute}
\centering
\includegraphics[width=1\textwidth]{Attribute/log2.png}
\caption{log2}
\end{figure}
\newpage
\begin{figure}
\centering
\includegraphics[width=1.2\textwidth]{Attribute/setEntropia.png}
\caption{setEntropia}
\end{figure}
\newpage


%%%%%%%%%%%%%%%%%%%%%%%%%% CLASE  Node %%%%%%%%%%%%%%%%%%%%%%%%%%%%%%%%%%%%%%%%%%%%

\begin{figure}
\section{Clase Node}
\centering
\includegraphics[width=1.2\textwidth]{Node/addSon.png}
\caption{addSon}
\end{figure}
\newpage
\begin{figure}
\centering
\includegraphics[width=1\textwidth]{Node/getIndexOfChild.png}
\caption{getIndexOfChild}
\end{figure}
\newpage

\end{document}

% \documentclass[letterpaper,12pt]{report}
% \usepackage[pdftex]{graphicx}
% \usepackage[spanish]{babel}
% \usepackage[latin1]{inputenc}
% \pagestyle{plain}

% \begin{document}
% \addtolength{\hoffset}{-0.5cm}
% \addtolength{\textwidth}{1cm}
% \addtolength{\textheight}{1cm}
% \chapter{UML}

%%%%%%%%%%%%%%%%%%%%%%%%%% CLASE Apriori%%%%%%%%%%%%%%%%%%%%%%%%%%%%%%%%%%%%%%%%%%%%
\begin{figure}
\section{Dise\~no}
\subsection{Diagramas de Colaboraci\'on}
\subsubsection{Clase Apriori}
\centering
\includegraphics[angle=90, width=0.33\textwidth]{imgsColaboracion/Apriori/Combinations.png}
\caption{Combinations}
\end{figure}
\newpage
\begin{figure}
\centering
\includegraphics[angle=90, width=0.41\textwidth]{imgsColaboracion/Apriori/IncreaseSuport.png}
\caption{IncreaseSuport}
\end{figure}
\newpage
\begin{figure}
\centering
\includegraphics[angle=90, width=0.47\textwidth]{imgsColaboracion/Apriori/main.png}
\caption{main}
\end{figure}
\newpage
\begin{figure}
\centering
\includegraphics[angle=90, width=0.43\textwidth]{imgsColaboracion/Apriori/makeCandidates.png}
\caption{makeCandidates}
\end{figure}
\newpage
\begin{figure}
\centering
\includegraphics[angle=90, width=0.33\textwidth]{imgsColaboracion/Apriori/pruneCandidate.png}
\caption{pruneCandidate}
\end{figure}
\newpage
%%%%%%%%%%%%%%%%%%%%%%%%%% CLASE  EuipAsso%%%%%%%%%%%%%%%%%%%%%%%%%%%%%%%%%%%%%%%%%%%%
\begin{figure}
\subsubsection{Clase EuipAsso}
\centering
\includegraphics[width=1.2\textwidth]{imgsColaboracion/EuipAsso/EquipAsso/main.png}
\caption{main}
\end{figure}
\newpage
\begin{figure}
\centering
\includegraphics[width=1.2\textwidth]{imgsColaboracion/EuipAsso/EquipAsso/run.png}
\caption{run}
\end{figure}
\newpage
\begin{figure}
\centering
\includegraphics[angle=90,width=0.4\textwidth]{imgsColaboracion/EuipAsso/EquipAsso/pruneCandidates.png}
\caption{pruneCandidates}
\end{figure}
\newpage
%%%%%%%%%%%%%%%%%%%%%%%%%% CLASE  Combinations%%%%%%%%%%%%%%%%%%%%%%%%%%%%%%%%%%%%%%%%%%%%
\begin{figure}
\subsubsection{Clase Combinations}
\centering
\includegraphics[width=1\textwidth]{imgsColaboracion/EuipAsso/Combinations/combine7.png}
\caption{combine7}
\end{figure}
\newpage
\begin{figure}
\centering
\includegraphics[width=0.6\textwidth]{imgsColaboracion/EuipAsso/Combinations/letsCombine.png}
\caption{letsCombine.png}
\end{figure}
\newpage
%%%%%%%%%%%%%%%%%%%%%%%%%% CLASE  BaseConditionals %%%%%%%%%%%%%%%%%%%%%%%%%%%%%%%%%%%%%%%%%%%%
\begin{figure}
\subsubsection{Clase BaseConditionals}
\centering
\includegraphics[width=0.3\textwidth]{imgsColaboracion/FPGrowth/BaseConditionals/addBaseConditionals.png}
\caption{addBaseConditionals}
\end{figure}
\newpage
\begin{figure}
\centering
\includegraphics[angle=90,width=0.3\textwidth]{imgsColaboracion/FPGrowth/BaseConditionals/findItem.png}
\caption{findItem}
\end{figure}
\newpage
%%%%%%%%%%%%%%%%%%%%%%%%%% CLASE  Combinations %%%%%%%%%%%%%%%%%%%%%%%%%%%%%%%%%%%%%%%%%%%%
\begin{figure}
\subsubsection{Clase Combinations}
\centering
\includegraphics[angle=90,width=0.5\textwidth]{imgsColaboracion/FPGrowth/Combinations/addCandidates.png}
\caption{addCandidates}
\end{figure}
\newpage
\begin{figure}
\centering
\includegraphics[angle=90,width=0.5\textwidth]{imgsColaboracion/FPGrowth/Combinations/Combine.png}
\caption{Combine}
\end{figure}
\newpage
%%%%%%%%%%%%%%%%%%%%%%%%%% CLASE  FPGrowth %%%%%%%%%%%%%%%%%%%%%%%%%%%%%%%%%%%%%%%%%%%%
\begin{figure}
\subsubsection{Clase FPGrowth}
\centering
\includegraphics[width=0.8\textwidth]{imgsColaboracion/FPGrowth/FPGrowth/builtFrecuenceNode.png}
\caption{builtFrecuenceNode}
\end{figure}
\newpage
\begin{figure}
\centering
\includegraphics[width=0.7\textwidth]{imgsColaboracion/FPGrowth/FPGrowth/FrequentNode.png}
\caption{FrequentNode}
\end{figure}
\newpage
%%%%%%%%%%%%%%%%%%%%%%%%%% CLASE  AvlTree %%%%%%%%%%%%%%%%%%%%%%%%%%%%%%%%%%%%%%%%%%%%
\begin{figure}
\subsubsection{Clase AvlTree}
\centering
\includegraphics[width=0.9\textwidth]{imgsColaboracion/Utils/AvlTree/compareItems.png}
\caption{compareItems}
\end{figure}
\newpage
\begin{figure}
\centering
\includegraphics[width=0.7\textwidth]{imgsColaboracion/Utils/AvlTree/insert.png}
\caption{insert}
\end{figure}
\newpage

%%%%%%%%%%%%%%%%%%%%%%%%%% CLASE  DataSet %%%%%%%%%%%%%%%%%%%%%%%%%%%%%%%%%%%%%%%%%%%%
% \subsubsection{Clase DataSet}
% \begin{figure}
% \centering
% \includegraphics[angle=90,width=0.49\textwidth]{imgsColaboracion/Utils/DataSet/builtDictionary.png}
% \caption{builtDictionary}
% \end{figure}
% \newpage
% \begin{figure}
% \centering
% \includegraphics[width=1.2\textwidth]{imgsColaboracion/Utils/DataSet/pruneCandidatesOne.png}
% \caption{pruneCandidatesOne}
% \end{figure}
% \newpage
% \begin{figure}
% \centering
% \includegraphics[width=1.2\textwidth]{imgsColaboracion/Utils/DataSet/saveTree.png}
% \caption{saveTree}
% \end{figure}
% \newpage
%%%%%%%%%%%%%%%%%%%%%%%%%% CLASE  Transaction %%%%%%%%%%%%%%%%%%%%%%%%%%%%%%%%%%%%%%%%%%%%
% \begin{figure}
% \subsubsection{Clase Transaction}
% \centering
% \includegraphics[width=1.2\textwidth]{imgsColaboracion/Utils/Transaction/loadItems.png}
% \caption{loadItems}
% \end{figure}
% \newpage
% 
% \begin{figure}
% \centering
% \includegraphics[width=0.4\textwidth]{imgsColaboracion/Utils/Transaction/sortBySupport.png}
% \caption{sortBySupport}
% \end{figure}


% \end{document}

\documentclass[letterpaper,12pt]{report}
\usepackage[pdftex]{graphicx}
\usepackage[spanish]{babel}
\usepackage[latin1]{inputenc}
\pagestyle{headings}

\begin{document}
\addtolength{\textwidth}{-5cm}
\chapter{Diagramas de Clase}

%%%%%%%%%%%%%%%%%%%%%%%%%% KnowledgeFlow %%%%%%%%%%%%%%%%%%%%%%%%%%%%%%%%%%%%%%%%%%%%
\begin{figure}
\centering
\includegraphics[width=1.8\textwidth]{KnowledgeFlow.png}
\caption{Paquete KnowledgeFlow}
\end{figure}
\newpage
%%%%%%%%%%%%%%%%%%%%%%%%%% Utils %%%%%%%%%%%%%%%%%%%%%%%%%%%%%%%%%%%%%%%%%%%%
\begin{figure}
\centering
\includegraphics[width=1.5\textwidth]{Utils.png}
\caption{Paquete Utils}
\end{figure}
\newpage
%%%%%%%%%%%%%%%%%%%%%%%%%% FPGrowth %%%%%%%%%%%%%%%%%%%%%%%%%%%%%%%%%%%%%%%%%%%%
\begin{figure}
\centering
\includegraphics[width=1.5\textwidth]{FPGrowth.png}
\caption{Paquete FPGrowth}
\end{figure}
\newpage
%%%%%%%%%%%%%%%%%%%%%%%%%% EquipAsso %%%%%%%%%%%%%%%%%%%%%%%%%%%%%%%%%%%%%%%%%%%%
\begin{figure}
\centering
\includegraphics[angle=90, width=1.1\textwidth]{EquipAsso.png}
\caption{Paquete EquipAsso}
\end{figure}
\newpage
%%%%%%%%%%%%%%%%%%%%%%%%%% Mate %%%%%%%%%%%%%%%%%%%%%%%%%%%%%%%%%%%%%%%%%%%%
\begin{figure}
\centering
\includegraphics[angle=90, width=1.4\textwidth]{Mate.png}
\caption{Paquete Mate}
\end{figure}
\newpage
%%%%%%%%%%%%%%%%%%%%%%%%%% C45 %%%%%%%%%%%%%%%%%%%%%%%%%%%%%%%%%%%%%%%%%%%%
\begin{figure}
\centering
\includegraphics[angle=90, width=1.8\textwidth]{c45c.png}
\caption{Paquete C45}
\end{figure}
\newpage

\end{document}


\subsection{Diagramas de Paquetes}

%%%%%%%%%%%%%%%%%%%%%%%%%% Tariy %%%%%%%%%%%%%%%%%%%%%%%%%%%%%%%%%%%%%%%%%%%%
\begin{figure}[h]
\centering
\includegraphics[angle=90, width=0.39\textwidth]{imgsPaquetes/tariy.png}
\caption{Paquete Principal}
\end{figure}
\newpage
%%%%%%%%%%%%%%%%%%%%%%%%%% Algoritmos %%%%%%%%%%%%%%%%%%%%%%%%%%%%%%%%%%%%%%%%%%%%
\begin{figure}
\centering
\includegraphics[angle=90, width=0.48\textwidth]{imgsPaquetes/algorithm.png}
\caption{Paquete Algoritmos}
\end{figure}
\newpage
%%%%%%%%%%%%%%%%%%%%%%%%%% GUI %%%%%%%%%%%%%%%%%%%%%%%%%%%%%%%%%%%%%%%%%%%%
\begin{figure}
\centering
\includegraphics[angle=90, width=0.8\textwidth]{imgsPaquetes/gui.png}
\caption{Paquete Interfaz Gr\'afica}
\end{figure}
\newpage
%%%%%%%%%%%%%%%%%%%%%%%%%% Filtros %%%%%%%%%%%%%%%%%%%%%%%%%%%%%%%%%%%%%%%%%%%%
\begin{figure}
\centering
\includegraphics[angle=90, width=0.9\textwidth]{imgsPaquetes/filters.png}
\caption{Paquete GUI Filtros}
\end{figure}
\newpage

\chapter{IMPLEMENTACI\'ON}
\section{Arquitectura de TariyKDD}
Para el desarrollo de TariyKDD se utilizaron computadores con procesador AMD 64 bits, disco duro Serial ATA,
\'util  al tomar los datos desde un repositorio y al momento de realizar pruebas de rendimiento de los
algoritmos, ya que su velocidad de transferencia es de 150 MB/sg; adem\'as la RAM que se utilizo fu\'e
superior a los 512 MB, ya que la Miner\'ia de Datos requiere grandes cantidades de memoria por el tama\~no de
los conjuntos de datos.\\

El sistema operativo sobre el cual se trabajo durante la implementaci\'on de TariyKDD es Fedora Core en su
versiones 3 y 5. El lenguaje de programaci\'on en el que esta elaborado TariyKDD es Java 5.0, 
actualizaci\'on 06.\\

Dentro del proceso de Descubrimiento de Conocimiento, TariyKDD comprende las etapas de Selecci\'on,
Preprocesamiento, Miner\'ia de Datos y Visualizaci\'on de Resultados. De esta forma la implementaci\'on de la
herramienta se hizo a trav\'es de los siguientes m\'odulos de software cuya estructura se muestra en la figura
\ref{arquitectura}.

\begin{figure}[!ht]
\centering
\includegraphics[width=0.35\textwidth]{images/arquitectura.png}
\caption{\'Arquitectura TariyKDD}
\label{arquitectura}
\end{figure}

%----------------------------------------- DESCRIPCION DE DESARROLLO --------------------------------------------%
\section{Descripci\'on del desarrollo de TariyKDD}
A continuaci\'on, en primera instancia se describe de manera general la estructura de TariyKDD para dar una idea
global de como esta herramienta fu\'e implementada. A continuaci\'on, se amplia y se detalla m\'as la forma en 
que TariyKDD fu\'e desarrollada.

\begin{itemize}
\item utils: Utilidades de TariyKDD. Dentro de este paquete encontramos clases como DataSet...
\item algorithm: Esta compuesta de los paquetes association y classification, los cuales implementan mediante sus
respectivas clases los algoritmos de asociaci\'on (Apriori, FPgrowth y EquipAsso) y clasificaci\'on (C4.5 y
MateBy).
\item gui: Comprende los paquetes Icons y KnowledgeFlow, los cuales a trav\'es de sus clases implementan la 
interfaz gr\'afica de la herramienta.
\end{itemize}
%------------------- PAQUETE UTILS
\subsection{Paquete Utils}
Esta clase esta compuesta por las siguientes clases:
\subsubsection{Clase AssocRules}

Esta clase es la encargada de generar reglas a partir de los \'arboles de itemsets frecuentes que generan los
algoritmos de asociaci\'on. Las reglas se generan teniendo en cuenta el par\'ametro \textit{confianza} . Los
atributos que maneja esta clase son los siguientes:\\

\textit{frequents }de tipo Vector: vector de \'arboles AVL que contienen  su ves itemsets frecuentes.\\

\textit{dictionary }de tipo ArrayList: arreglo que contiene los nombres de los itemsets frecuentes.\\

\textit{confidence }de tipo entero: variable en la que se especifica la confianza de evaluaci\'on de las reglas.\\

\textit{rules }de tipo ArrayList: arreglo en el que se almacenan las reglas que cumplan con la confianza.\\

El curso normal de los eventos empieza con el llamado al m\'etodo \textit{buildRules }en el cual se inicia el
recorrido del vector de de \'arboles AVL. La raiz de cada \'arbol es enviada como par\'ametro al m\'etodo
\textit{walkTree }que es el encargado de recorrer el \'arbol y armar por cada rama recorrida un arreglo que es
enviado al m\'etodo \textit{combinations }para que se generen todas las combinaciones posibles de esa rama y
obtener las nuevas reglas. Cada regla es evaluada para verificar que cumpla el nivel de confianza. La f\'ormula
para la evaluaci\'on de confianza es: $(soporte_Frecuentes / soporte_Antecedente) * 100$. Si una regla supera el
nivel de confianza es enviada al m\'etodo \textit{decodeFrecuents} para su decodificaci\'on. Al final para efectos
pr\'acticos de representaci\'on existen varios m\'etodos que permiten oredenar las reglas generadas de acuerdo a
varios par\'ametros: soporte, confianza u arden alfab\'etico.

\subsubsection{Clase AvlNode}
En esta clase se construye la estructura interna de datos de un nodo perteneciente a un \'arbol Avl. Los atributos
que maneja esta clase son los siguientes:\\

 \textit{element} de tipo ItemSet: aqui se guardan los datos con los que se carga el nodo.\\

 \textit{left, right} de tipo Avlnode: punteros hacia los hijos izquierdo y derecho respectivamente.\\

 \textit{height} de tipo entero: altura del \'arbol Avl.\\

 \textit{check} de tipo booleano: variable de control del balance del \'arbol.

\subsubsection{Clase AvlTree}

Esta clase es la encargada tanto de crear o armar un arbol Avl como de proveer los m\'etodos necesarios para su
manejo. Un \'arbol Avl es un tipo de \'arbol binario que es balanceado de tal manera que la profundidad de dos
hojas cualquiera en el \'arbol difiera como m\'aximo en uno. Los atributos de esta cln los siguientes:\\

 \textit{root} de tipo Avlnode: raiz del \'arbol Avl\\

 \textit{n} de tipo entero: variable de control de la profundidad o n\'umero de niveles del \'arbol.\\

 \textit{node} de tipo AvlNode: nodo para recorrer el \'arbol.\\

 \textit{stack} de tipo Stack: pila utilizada para almacenar la ruta del recorrido hecho en una inserci\'on,
 b\'usqueda o eliminaci\'on.\\

Para lograr la construcci\'on y manejo de un \'arbol Avl son necesarios algunos m\'etodos, los cuales ser\'an
mencionados m\'as no explicados debido a su complejidad y a que estos m\'etodos son bastante conocidos dentro del
\'ambito de las estructuras de datos. Los m\'etodos implementados son: \textit{insert, find, height,
rotateWithLeftChild, rotateWithRightChild, doubleWithLeftChild, doubleWithRightChild}, m\'etodos de inserci\'on,
b\'usqueda, consulta de profundidad del \'arbol, rotaci\'on por el hijo izquierdo, rotaci\'on por el hijo derecho,
doble rotaci\'on por izquierda y doble rotaci\'on por derecha respectivamente.

\subsubsection{Clase BaseDatos}
Esta clase es utilizada siempre que se necesite efectuar conexiones a bases de datos por medio del objeto de
conexi\'on java para Postgres. Cabe aclarar que el sistema gestor de bases de datos (SGBD) inicial es Postgres pero
el nombre del sistema gestor es totalmente parametrizable, permitiendo de esta manera la conexi\'on a casi
cualquier SGBD. Los atributos de esta clase son:\\

 \textit{db} de tipo Connection: objeto encargado de establecer la conexi\'on.\\

 \textit{stm} de tipo Statement: variable encargada de ejecutar un query determinado.\\

 \textit{nombre} de tipo String: nombre de la base de datos.\\

Inicialmente se estable el nombre del contolador de la base de datos que se desea consultar. En este caso el
controlador Postgres. Luego se llama al m\'etodo \textit{iniciarBD} que se encarga de establecer la conexi\'on a la
base de datos. Tiene como par\'ametros el nombre de la base de datos, el nombre del usuario y constrase\~na.\\

Existen otros m\'etodos implementados para la consulta del nombre de las tablas, inserci\'on y consulta de los
datos de una tabla. El m\'etodo \textit{getTablas} retorna un vector con los nombres de las tablas que contenga la
base de datos. El m\'etodo \textit{insertarBD} permite insertar un registro nuevo en una tabla. Tiene como
par\'ametros el nombre de la tabla, el identificador de u item y el nombre del item a insertar.  \textit{getDatos}
permite ejecutar una consulta general a una tabla. Solo requiere el nombre de la tabla a consultar.

\subsubsection{Clase FileManager}

Esta clase se encarga de todo lo que tiene que ver con el manejo de archivos de acceso a aleatorio y el flujo de
informaci\'on entre el archivo le\'ido, el DataSet y el diccionario de datos. 
Los atributos de esta clase son:\\

 \textit{out} de tipo File: archivo al cual se va a leer o escribir.\\

 \textit{outChannel} de tipo RandomAccessFile: puntero para ubicarse en el archivo de acceso aleatorio. Proporciona
 el flujo hacia el archivo.\\

 \textit{data} de tipo Object: matriz con los datos de un archivo de acceso aleatorio .arff\\

 \textit{attributes} de tipo Object: arreglo con los nombres de los atributos de un archivo de acceso aleatorio
 .arff.\\

 \textit{dictionary} de tipo Array List: arreglo en el que se almacena el diccionario de datos de un archivo
 .arff\\

 Dentro de la lista de utilidades que se pueden encontrar en esta clase se pueden listar las siguientes:\\

 \textit{writeItem}: m\'etodo utilizado para escribir un entero corto (short) en un archivo. C\'odigo \ref{c2}\\

\begin{codigof}[t]
\begin{verbatim}
...
  public void writeItem(short s) {
  try{
    outChannel.seek( out.length() );
    outChannel.writeShort(s);
  } catch( IOException e ) {
  e.printStackTrace();
  }
  }
...
\end{verbatim}
\caption{Escribir un item a archivo}
\label{c2}
\end{codigof}

\textit{writeString}: m\'etodo utilizado para escribir una cadena de caracters en un archivo. C\'odigo \ref{c3}.

\begin{codigof}[!h]
\begin{verbatim}
...
  public void writeString(String s){
  byte b[];
  try{
    b = s.getBytes();
    outChannel.seek( out.length() );
    outChannel.write(b);
  } catch( IOException e ) {
  e.printStackTrace();
  }
  }
...
\end{verbatim}
\caption{Escribir una cadena a archivo}
\label{c3}
\end{codigof}

\textit{getFileSize}: retorna el tama\~no de un archivo en bytes. Cuadro \ref{c4}\\

\begin{codigof}[!h]
\begin{verbatim}
...
  public long getFileSize() {
  return out.length();
  }
...
\end{verbatim}
\caption{Tama\~no de un archivo}
\label{c4}
\end{codigof}

 \textit{getFileName}: retorna el nombre del archivo de acceso aleatorio. Cuadro \ref{c5}\\

\begin{codigof}[t]
\begin{verbatim}
...
  public String getFileName() {
  return out.getName();
  }
...
\end{verbatim}
\caption{Nombre de un archivo}
\label{c5}
\end{codigof}

 \textit{closeFile}: cierra el flujo hacia el archivo de acceso aleatorio. Cuadro \ref{c6} \\

\begin{codigof}[h]
\begin{verbatim}
...
  public void closeFile() {
  try{
    outChannel.close();
  } catch(IOException e) {
  e.printStackTrace();
  }
  }
...
\end{verbatim}
\caption{Cierra conexi\'on a archivo}
\label{c6}
\end{codigof}
 \textit{deleteFile}: borra f\'i{}sicamente el archivo de acceso aleatorio. \ref{c7} \\

\begin{codigof}[h]
\begin{verbatim}
...
  public void deleteFile() {
  out.deleteOnExit();
  }
...
\end{verbatim}
\caption{Borrar un archivo}
\label{c7}
\end{codigof}

 \textit{setOutChannel}: ubica el flujo en una posici\'on determinada del archivo de acceso aleatorio.
C\'odigo \ref{c8}.\\

\begin{codigof}[t]
\begin{verbatim}
...
  public void setOutChannel(int pos) throws IOException {
  outChannel.seek(pos);
  }
...
\end{verbatim}
\caption{Manejo de posici\'on en un archivo}
\label{c8}
\end{codigof}

\textit{ReadTransaction}: lee y muestra el contenido de un archivo de acceso aleatorio. Esta funci\'on
es \'util para el antiguo formato de archivos Tariy, lee un flujo de tipo \textit{Short }y muestra cada
transacci\'on separada por el cero. C\'odigo \ref{c9}\\

\begin{codigof}[!h]
\begin{verbatim}
...
  public void ReadTransaction(int readposition) {
  try{
    outChannel.seek(readposition);
    short s;
    while( true ) {
    s = outChannel.readShort();
    if(s != 0)
      System.out.print(s + " ");
    else
      System.out.println();
    }
  } catch(EOFException e1) {
    this.closeFile();
  } catch(IOException e2) {
    e2.printStackTrace();
  }
  }
...
\end{verbatim}
\caption{Lectura de un archivo}
\label{c9}
\end{codigof}

\textit{getAttributes}: retorna el arreglo que tiene los nombres los atributos del archivo de acceso
aleatorio .arff. C\'odigo \ref{c10} \\

\begin{codigof}[!h]
\begin{verbatim}
...
  public Object[] getAttributes() {
  return attributes;
  }
...
\end{verbatim}
\caption{Nombres de atributos de un archivo plano}
\label{c10}
\end{codigof}

\textit{getData}: retorna la matriz que almacena los datos del archivo de acceso aleatorio .arff. C\'odigo
\ref{c11} \\

\begin{codigof}[t]
\begin{verbatim}
...
  public Object[][] getData() {
  return data;
  }
...
\end{verbatim}
\caption{Matriz de datos del archivo plano}
\label{c11}
\end{codigof}

 \textit{getDictionary}: retorna el diccionario de datos construido a partir del archivo de acceso
aleatorio .arff. \\

 \textit{buildMultivaluedDataset}: este es uno de los m\'etodos m\'as importantes en la face de carga de
los datos desde el archivo plano a memoria. El archivo de extenci\'on arff se recorre completamente pero solo se
cargan los datos necesarios al DataSet. Por ejemplo, en un archivo arff se pueden encontrar algunas etiquetas
propias como lo son \textit{@data }y \textit{@attribute}. Este tipo de etiquetas son ignoradas al igual que los
espacios en blanco. Debido a que el objetivo es tratar de ahorrar espacio en memoria y a la naturaleza del DataSet
los datos del archivo plano no pueden ser almacenados directamente como vienen. Los datos se codifican como
enteros cortos \textit{shorts} y sus nombres originales se almacenan en un diccionario para poder recuperarlos en
el proceso inverso en el momento de generar las reglas.\\

 \textit{buildUnivaluedDataSet}: este es un m\'etodo similar al anterior pero optimizado para manejar
archivos univaluados, es decir, aquellos que solo tienen dos columnas una de las cuales es el identificador y la
otra el nombre del art\'iculo.\\

 \textit{dataAndAttributes}: a partir de un archivo de acceso aleatorio .arff, almacena los nombres de
los atributos en un arreglo, los datos en una matriz y el diccionario de datos en un ArrayList.\\

 \textit{builtDataMatrix}: retorna una matriz que representa un archivo de acceso aleatorio .arff con los
nombres de sus atributos y sus datos.\\ \\

 \textbf{Clase ItemSet}\\

 Esta clase se encarga del manejo de los conjuntos de items. Un itemSet puede ser entendido como una
trasacci\'on. Los atributos de esta clase son:\\

 \textit{items} de tipo short: arreglo que almacena los items que forman el itemset.\\

 \textit{support} de tipo short: soporte asociado a este itemset.\\

 \textit{n} de tipo short: tama\~no del itemset.\\

 Los m\'etodos implementados son:\\

 \textit{getItems}: retorna el item asociado.\\

 \textit{getSupport}: retorna el soporte de un itemset\\

 \textit{gettype}: retorna el tipo de un itemset, el tipo se deriva del tama\~no del itemset.\\

 \textit{setSupport}: establece el soporte de un itemset.\\

 \textit{increaseSupport}: incrementa el soporte de un itemset, ya sea en uno o el n\'umero que se
requiera.

\subsubsection{Clase NodeNoF}
Esta clase se encarga de crear los nodos intermedios de un \'arbol de datos. Tambi\'en se implementan
los m\'etodos para enlazar un nodo a otro. Entre ellos se encuentran: addSon, addBro y findBro que se encargan de
adicionar un hijo, adicionar un hermano y buscar si un nodo tiene hermanos respectivamente.

\subsubsection{Clase NodeF}
Esta clase extiende a la clase NodeNoF. este tipo de nodos son las hojas de las ramas. Lo que las hace
diferentes, son dos atributos adicionales, uno es el soporte, que se encarga de llevar a cabo el conteo de veces
que una transacci\'on repetida ha sido tenida en cuenta, y el otro es un puntero a otra hoja. En el momento de
recorrer un \'arbol, el tipo de nodo nos permite saber que hemos llegado al final de una rama.

\subsubsection{Clase Transaction}
Esta clase es la encargada de manejar los itemsets o conjuntos de items por cada transacci\'on. Cada uno
de los itemsets son almacenados en vectores. Esta clase se encarga de alimentar esos vectores, permite hacer
consulta sobre ellos u organizarlos. Estos m\'etodos se muestran a continuaci\'on:\\

 \textit{getArticles}: permite ver los articulos en una transacci\'on.\\

 \textit{getSize}: devuelve el tama\~no de una transacci\'on.\\

 \textit{getItemset}: devuelve un item de la trnsacci\'on.\\

 \textit{srtBySupport}: ordena las transacciones por soporte.\\

 \textit{srtByItem}: ordena las transacciones por el item.\\

 \textit{loadItemsets}: carga los art\'iculos o items de una transacci\'on.\\

 \textit{clearTransaction}: borra los items de una transacci\'on.\\

%------------------- PAQUETE ALGORITHM
\subsection{Paquete algorithm}
Este paquete esta compuesto a su vez por los paquetes association y classification, los cuales implementan los 
algoritmos de Asociaci\'on (Apriori, FPgrowth y EquipAsso) y Clasificai\'on (C4.5 y MateBy).

%------------------- PAQUETE ASSOCIATION
\subsubsection{Paquete association}
El paquete association esta conformado por los paquetes que implementan los algoritmos de asociaci\'on y que se
explican a continuaci\'on:

%------------------- PAQUETE APRIORI
\paragraph{Paquete Apriori}
La implementaci\'on del algoritmo apriori se realiz\'o utilizando una clase denominada apriori.java que
interactua directamente con otras clases definidas en el paquete Utils como DataSet, Transaction, AvlTree e
Itemset.  Al igual que las otras clases que implementan algoritmos de asociaci\'on, en el constructor de la clase
Apriori se usan 2 parametros:  una instancia de la clase DataSet, donde vienen comprimidos el conjunto de datos
desde el m\'odulo de conexi\'on o el m\'odulo de filtros, y un entero corto, que es suministrado por el usuario,
que ser\'a usado como el soporte del sistema durante la ejecuci\'on de este algoritmo.\\

De igual manera, en la construcci\'on de la clase se instancia e inicializa un objeto de la clase AvlTree que
almacenar\'a los itemsets frecuentes tipo 1, el cual es alimentado haciendo uso del m\'etodo
\textit{pruneCandidatesOne} de la clase DataSet, el cual recibe el soporte del sistema como parametro.  En este
punto tambi\'en se instancia e inizializa  un arreglo que guardar\'a los diferentes \'arboles balanceados
(AvlTree), que contendr\'an los Itemsets Frecuentes de cada tipo y que ser\'an calculados por la herramienta. Al
disponer ya de los itemsets frecuentes tipo 1, estos son almacenados en la posicion 0 de este arreglo.  Los
detalles del constructor de esta clase se aclaran en el siguiente listado:\\

\begin{codigof}[h]
\begin{verbatim}

public Apriori(DataSet dataset, short support) {
  this.support = support;
  this.dataset = dataset;
  AvlTree frequentsOne = new AvlTree();
  frequentsOne = dataset.pruneCandidatesOne(support);
  Trees.addElement(frequentsOne);
  auxTree = frequentsOne;
}
\end{verbatim}
\caption{Constructor de la clase \textit{Apriori}}
\end{codigof}

Para disparar la ejecuci\'on del algoritmo usamos el m\'etodo \textit{run} el cual ejecuta un ciclo dentro del
cual el m\'etodo \textit{makeCandidates} se encarga de generar todos los itemset candidatos para cada
iteraci\'on.  Para esto el m\'etodo \textit{makeCandidates} llama al objeto \textit{auxTree}, instancia de la
clase AvlTree que conserva el ultimo \'arbol de itemset frecuentes y pregunta cuantos itemsets tiene en ese
instante utilizando el m\'etodo \textit{howMany} de esta clase.\\

Para cada elemento encontrado dentro del \'arbol de Itemsets Frecuentes se recorre un ciclo con los dem\'as
miembros del \'arbol armando combinaciones que generan un nuevo itemset candidato haciendo uso del m\'etodo
\textit{combinations} que recibe como parametros el \'arbol \textit{auxTree} de Itemset Frecuentes y un \'arbol
AvlTree donde se guardaran los Itemsets Frecuentes tipo k + 1 llamando \textit{frequents}.  Se aprovecha la
ventaja de que los itemsets est\'an ordenados en el \'arbol y se verifica que el itemset evaluado coincida en sus
$n-1$ primeros items, siendo $n$ el tama\~no de ese itemset, con el proximo itemset del \'arbol. De ser as\'i, se
instancia un nuevo itemset con los $n-1$ items coincidentes y los 2 \'ultimos items de cada itemset involucrado. 
De esta manera se generan itemsets candidatos de tama\~no $n+1$.\\

Cada itemset candidato es pasado como par\'ametro al m\'etodo \textit{increaseSupport} donde es contado el
n\'umero de ocurrencias de ese itemset candidato dentro del conjunto de datos.  En este punto se hace una
consideraci\'on importante y si los itemsets candidatos que se est\'an construyendo son de tipo 3 o superior se
evalua el paso de poda, esto es, se descompone el itemset candidato construido en sus $n-2$ posibles
combinaciones del tipo anterior, siendo $n$ el tama\~no del itemset candidato que se esta generando.  No se
evaluan las dos \'ultimas combinaciones pues fueran estas las que generaron el candidato que se esta analizando y
se tiene certeza de que existen en el \'arbol de Itemsets Frecuentes.  Cada una de las combinaciones generadas
son buscadas en \textit{auxTree}, hay que recordar que este es el \'arbol que contiene todos los Itemsets
Frecuentes del orden anterior al que se esta generando.  Si una de las combinaciones del itemset candidato actual
no es encontrada en el \'arbol \textit{auxTree}, este itemset ya no es considerado y se procede a evaluar el
pr\'oximo.  El m\'etodo \textit{pruneCandidate} se encarga de realizar este an\'alisis y se presenta en el
cuadro \ref{codapri2}.\\

\begin{codigof}[ht]
\begin{verbatim}

private boolean pruneCandidate(ItemSet candidate) {
  int size = candidate.size;
  ItemSet auxiliar;
  for(int i = 0; i < size - 2; i++) {
  aux = new ItemSet(size - 1);
  int k = 0;
  for(int j = 0; j < size; j++) {
    if(j != i) {
    auxiliar.addItem(candidate.getItems()[j]);
    }
  }
  if(auxTree.findItemset(auxiliar) == null) {
    return false;
  }
  }
  return true;
}
\end{verbatim}
\caption{Funci\'on \textit{pruneCandidates}}
\label{codapri2}
\end{codigof}

Si el itemset candidato actual es de tipo 2 o ha superado el paso de poda se proceder\'a a hacer su conteo.  Para
ello, se hace una nueva instancia de la clase \textit{Transaction} donde se carga una a una las transacciones del
conjunto de datos y se compara con el contenido del itemset.  Si coinciden los elementos del registro y el
itemset, este \'ultimo incrementa su soporte interno en uno.\\

Al final de este proceso se cuenta con un itemset candidato con su soporte establecido, este se compara con el
soporte del sistema y de superarlo es incluido en el \'arbol \textit{frequents}.  Cuando ya se han evaluado todos
los candidatos para un \'arbol \textit{auxTree}, se pregunta si se han generado nuevos Itemsets Frecuentes en
\textit{frequents}.  Si el m\'etodo \textit{howMany} de \textit{frequents} arroja existencias se almacena en el
arreglo de  \'arboles frecuentes y se asigna a \textit{auxTree} el \'arbol \textit{frequents} para iniciar de 
nuevo el proceso de generaci\'on de candidatos.  Esto se har\'a hasta que el \'arbol \textit{frequents} resulte
vac\'io (no hayan nuevos Itemsets Frecuentes).  En el arreglo de \'arboles frecuentes quedan almacenados todos
los Itemsets Frecuentes organizados por tipo que ser\'an pasados al M\'odulo de visores para ser visualizados
como reglas.
%------------------- PAQUETE FPGROWTH
\paragraph{Paquete FPgrowth}
Las clases que implementan el algoritmo FPgrowth se encuentran agrupadas en el paquete con el mismo nombre y las
m\'as importantes para su funcionamiento son FPGrowth, FPGrowthNode, FrequentNode, BaseConditional, 
BaseConditionals, Conditionals y AvlTree.\\

Tal y como se puede ver en la secci\'on \ref{fpgrowth}, el algoritmo FP-growth se basa en la generaci\'on de
itemsets candidatos a partir de un \'arbol N-Ario. Para comprender mejor la implementaci\'on de FP-growth, se
deben revisar primero las clases que forman la estructura del \'arbol N-Ario, las cuales han sido llamadas
FrequentNode y FPGrowthNode.

\subparagraph{Clase FrequentNode}
Para la posterior creaci\'on de itemsets frecuentes se almacenan en un array de objetos de tipo FrequentNode los
itemsets frecuentes-1 o de tama\~no 1. Cada uno de los items frecuentes-1 de esta lista se encuentra enlazado a
un nodo del \'arbol N-ario, de ahora en adelante FPtree, que tenga el mismo nombre que el itemset frecuente-1.

\subparagraph{Clase FPGrowthNode}
Esta clase proporciona la estructura del FPtree, en donde cada nodo tiene un valor de item, un soporte y los
punteros respectivos para realizar los enlaces entre nodos (Punteros al nodo hijo, al nodo padre, al nodo
hermano y al siguiente nodo con el mismo nombre). Su estructura se puede ver en el cuadro \ref{codfpgr1}.\\

\begin{codigof}[!h]
\begin{verbatim}

Atributos:
  short item //Dato que se va a a\~nadir al FPtree.
  short support //Soporte del item a\~nadido.
  FPGrowthNode fat //Puntero al nodo padre.
  FPGrowthNode son //Puntero al nodo hijo.
  FPGrowthNode bro //Puntero al nodo hermano.
  FPGrowthNode next //Puntero al siguiente nodo con el mismo valor.
\end{verbatim}
\caption{Estructura \textit{FPGrowthNode}}
\label{codfpgr1}
\end{codigof}

\subparagraph{Clase FPGrowth}
FPGrowth es la clase principal del paquete, tiene los m\'etodos m\'as importantes del algoritmo y a trav\'es
de los cuales se obtienen los itemsets frecuentes. Su estructura se puede ver en el cuadro \ref{codfpgr2}.\\

\begin{codigof}[ht]
\begin{verbatim}

Atributos:
  DataSet dataset //Estructura que comprime el conjunto de datos.
  short support //Soporte con el que se van a evaluar los datos.
  FrequentNode[] frequentsOne //Array que almacena a los itemsets 
  			    //frecuentes-1.
  FPGrowthNode cab //Raiz del FPtree.
  Vector Trees //Almacena los itemsets frecuentes de todos los 	
  	     //tamanos.
\end{verbatim}
\caption{Estructura \textit{FPGrowth}}
\label{codfpgr2}
\end{codigof}

Los parametros que el algoritmo FP-growth recibe en su constructor son, DataSet y el soporte suministrado por el
usuario. El primer paso de la implementaci\'on de FPGrowth es crear la lista con los itemsets frecuentes-1, la
cual se almacena en el array frequentsOne. El siguiente paso es construir el FPtree, para esto se lee cada una de
las transacciones de DataSet y para aquellos items cuyo soporte sea mayor o igual al m\'inimo se los almacena en
la clase del paquete Utils \textit{Transaction}. A partir de estos items filtrados se construye el FPtree, cuyo
seudoc\'odigo se puede observar en el cuadro \ref{codfpgr3}.\\

\begin{codigof}[!h]
\begin{verbatim}

boolean comienzo = true
puntero_referencia = null
do while (!fin de DataSet)
  transaccion_filtrada = transaccion(i)
  si (comienzo)
  Crear raiz de FPtree
  comienzo = false
  end si
  while (!fin de transaccion_filtrada)
  si (puntero_referencia == null)
    anadir nuevo_nodo como hijo
  end si
  si_no
  si (puntero_referencia = nuevo_nodo)
    incrementar_soporte ultimo_nodo
  end si
  si_no
    while(puntero_referencia tenga hermanos)
    si (nodo_hermano = nuevo_nodo)
      incrementar_soporte nodo_hermano
      terminar while
    end si
    end while
    si (puntero_referencia no tiene mas hermanos)
    anadir nuevo_nodo como hermano
    end si
  end while
end do
\end{verbatim}
\caption{Seudocodigo construcci\'on \textit{FPtree}}
\label{codfpgr3}
\end{codigof}

Para una mejor compresi\'on sobre la construcci\'on del FPtree se puede observar el conjunto de datos del cuadro
\ref{datos1} y su representaci\'on gr\'afica en la figura \ref{arbolfptree} y para profundizar m\'as en la
implementaci\'on se puede observar en el cuadro \ref{codfpgr4} el c\'odigo fuente de la funci\'on que construye el
FPtree (\textit{buildTree}):\\

\newpage
\begin{codigof}[!h]
\begin{verbatim}
  
  short frequent;
  BaseConditional baseconditional;
  BaseConditionals bcs;
  Vector path = new Vector(1,1);
  aux = cab.son;
  FPGrowthNode pcb, pcab;
  Arrays.sort(frequentsOne, new compareFrequentNode());
  for(int i = frequentsOne.length - 1; i >= 0; i--){
    FrequentNode aux = (FrequentNode) frequentsOne[i];
    pcb = aux.pcab;
    pcab = aux.pcab;
    frequent = pcb.getItem();
    bcs = new BaseConditionals(frequent, support);
    while(pcab != null){
    path.clear();
    pcb = pcb.fat;
    while(pcb != null){
      path.add((short) pcb.getItem());
      pcb = pcb.fat;
    }
    if(path.size() != 0){
      baseconditional = new BaseConditional(path, 
	          pcab.getSupport());
      bcs.addBaseConditionals(baseconditional);
    }
    pcab = pcab.next;
    pcb = pcab;
    }
    bcs.sortByElement();
    Trees = bcs.buildConditionals(Trees);
  }
\end{verbatim}
\caption{Funci\'on \textit{buildTree}}
\label{codfpgr4}
\end{codigof}

\newpage
\begin{table}[!h]
\begin{center}
\begin{tabular}{|p{25mm}|p{35mm}|}\hline
\textbf{Transacci\'on} & \textbf{Lista de items}\\ \hline\hline
T01 & \{(1,2,5)\}\\ \hline
T02 & \{(2,4)\}\\ \hline
T03 & \{(2,3)\}\\ \hline
T04 & \{(1,2,4)\}\\ \hline
T05 & \{(1,3)\}\\ \hline
T06 & \{(2,3)\}\\ \hline
T07 & \{(1,3)\}\\ \hline
T08 & \{(1,2,3,5)\}\\ \hline
T09 & \{(1,2,3)\}\\ \hline
\end{tabular}
\end{center}
\caption{Conjunto de datos transaccional}
\label{datos1}
\end{table}

\begin{figure}[!h]
\centering
\includegraphics[width=0.8\textwidth]{images/fptree.png}
\caption{\'Arbol FPtree}
\label{arbolfptree}
\end{figure}

Como se puede observar en la figura \ref{arbolfptree}, cada uno de los items frecuentes-1 se encuentra enlazado a
un nodo del FPtree, por ejemplo el item frecuente-1, 5 se encuentra enlazado al nodo de FPtree 5, cuyo soporte es
1 y a la vez cada nodo de FPtree se enlaza al siguiente nodo con el mismo nombre, en el caso del nodo 5, este se
enlaza a otro nodo con valor 5 y cuyo soporte es 1.\\

El siguiente paso de la implementaci\'on cosiste en recorrer la estructura de los items frecuentes-1 y por cada
uno de estos construir sus Patrones Condicionales Base, los cuales se obtienen recorriendo FPtree desde cada nodo
enlazado por los items frecuentes-1 a trav\'es de su puntero al nodo padre hasta la raiz. Por ejemplo como se
puede observar en la figura \ref{arbolfptree} para el item frecuente-1, 5 sus Patrones Condicionales Base son
dos: (1,2:1) y (3,1,2:1), donde el n\'umero despu\'es de '':'' es el soporte del Patr\'on, el cual corresponde al
mismo soporte que tenga el item frecuente-1 en cuestion. Cada uno de los Patrones Condicionales Base se almacenan
en la clase \textit{BaseConditional} y todo el conjunto de Patrones Condicionales Base de un item frecuente-1 se
alamcenan en la clase \textit{BaseConditionals}. En el cuadro \ref{pcb} se pueden observar los Patrones
Condicionales Base de cada uno de los items frecuentes-1.\\

Cuando se han construido todos los Patrones Condicionales Base de un item frecuente-1, el siguiente paso de la
implementaci\'on es determinar cuales de sus items tienen soporte mayor o igual al m\'inimo. Para esto se debe 
sumar cada uno de los soportes que un item tenga en cada Patr\'on Condicional Base, los items que cumplan con el
soporte van a conformar los Patrones Condicionales, los cuales se almacenan en la clase \textit{Conditionals},
por ejemplo para un soporte m\'inimo igual a 2 los Patrones Condicionales del item frecuente-1 son (1:2) y (2:2),
se puede observar que el item 3 no cumple con el soporte, por tanto no es Patr\'on Condicional. Los Patrones
Condicionales de los items frecuentes-1 se pueden observar en el cuadro \ref{pcb}.\\

\begin{table}[h]
\begin{center}
\begin{tabular}{|p{9mm}|p{35mm}|p{30mm}|p{40mm}|}\hline
\textbf{Item} & \textbf{Patrones Condicionales Base} & \textbf{Patrones Condicionales} & \textbf{Patrones
Frecuentes}\\ \hline\hline
5 & \{(1,2:1),(3,1,2:1)\}   & (2:2, 1:2)     & (2,5:2),(1,5:2)(2,1,5:2)\\ \hline
4 & \{(2:1),(1,2:1)\}     & (2:2)      & (2,4:2)\\ \hline
3 & \{(2:2),(1:2),(1,2:2)\} & (2:4, 1:2),(1:2) & (2,3:4),(1,3:2),(2,1,3:2)\\ \hline
1 & \{(2:4)\}         & (2:4)      & (2,1:4)\\ \hline
\end{tabular}
\end{center}
\caption{Patrones Condicionales e Itemsets Frecuentes}
\label{pcb}
\end{table}

El \'ultimo paso de la implementaci\'on es determinar el conjunto de Itemsets Frecuentes. Para lo cual se hace 
uso de la clase \textit{Combinations}, la cual toma los Patrones Condicionales y los combina con los items 
frecuentes-1. Por ejemplo, tenemos que los Patrones Condicionales del item frecuente-1, 5 son (2:2, 1:2),
entonces, 5 se combina con sus Patrones Condicionales y se obtienen los Itemsets Frecuentes (2,5), (1,5) y
(2,1,5). Para los dem\'as items, podemos ver sus Itemsets Frecuentes en el cuadro \ref{pcb}.\\

As\'i como para Apriori y EquipAsso los Itemsets Frecuentes se almacenan en un Vector de arboles AVL balanceados, 
en donde en cada posici\'on del Vector, se encuentran almacenados un tipo de Itemsets Frecuentes. Es decir en la 
posici\'on 0 del Vector se encuentran los Itemsets Frecuentes tipo 1, en la posici\'on 1 los Itemsets Frecuentes 
tipo 2 y as\'i mismo el resto de Itemsets Frecuentes.
%------------------- PAQUETE EQUIPASSO
\paragraph{Algoritmo EquipAsso}
El paquete EquipAsso dentro del m\'odulo de algoritmos contiene dos clases encargadas de la implementaci\'on y
aplicaci\'on del algoritmo EquipAsso orientadas a dar soporte al descubrimiento de reglas de asociaci\'on dentro 
del proceso KDD. La primera \textit{EquipAsso}, encargada de la carga de datos y un primer filtrado de ellos,
donde se carga solo aquellos items que pasan el umbral o soporte dado (proceso EquiKeep dentro del
algoritmo EquipAsso) y la otra clase \textit{Combinations} que se encarga de generar todas las combinaciones de
un determinado tama\~no para cada uno de los registros cargados del conjunto de datos y sus respectivo conteo
(proceso Associator dentro del algoritmo EquipAsso).\\

La clase principal es \textit{EquipAsso}, al hacer una instancia de esta clase se debe pasar como par\'ametros
una instancia de la clase \textit{DataSet}, llamada \textit{dataset} y que contiene una versi\'on comprimida del
conjunto de datos, y un entero corto, que se llamado \textit{support} y que cumple la tarea de soporte del
sistema.  Estas instancias son asignadas a atributos internos de esta clase.\\

La clase EquipAsso cuenta con un atributo de tipo arreglo llamado \textit{Trees} donde se almacenan los
diferentes \'arboles de Itemsets Frecuentes organizados por tama\~no.  El primer conjunto de itemsets son los de
tama\~no 1 y podemos obtenerlo al llamar al m\'etodo \textit{pruneCandidatesOne} de la clase \textit{DataSet} que
fu\'e pasada como par\'ametro al constructor de la clase.  Este m\'etodo devuelve un \'arbol AvlTree que es
insertado en la posici\'on 0 del arreglo \textit{Trees} y que contiene el conjunto de Itemsets Frecuentes
tama\~no 1. Los dem\'as conjuntos de Itemsets Frecuentes (tama\~no 2, tama\~no 3, ...) ser\'an almacenados en las
siguientes posiciones de ese arreglo organizados en \'arboles AvlTree.  En el siguiente listado de c\'odigo
podemos observar el constructor de la clase EquipAsso.\\

\begin{codigof}[!h]
\begin{verbatim}
public EquipAsso(DataSet dataset, short support) {
  this.dataset = dataset;
  this.support = support;
  AvlTree frequentsOne = new AvlTree();
  frequentsOne = dataset.pruneCandidatesOne(support);
  Trees.addElement(frequentsOne);
}
\end{verbatim}
\caption{Constructor de la clase \textit{EquipAsso}}
\end{codigof}

Una vez instanciado un objeto \textit{EquipAsso} se inicia el proceso del algoritmo llamando al m\'etodo
\textit{run} de esta clase.  Dentro de este m\'etodo se inicia un ciclo controlado por el m\'etodo
\textit{findInDataset} donde se recorre el conjunto de datos.  Este m\'etodo declara un objeto del tipo
\textit{Transaction} el cual, a trav\'es de su m\'etodo \textit{loadTransaction}, carga los registros del
conjunto de datos.  En este punto se realiza un primer filtrado cargando \'unicamente aquellos items que
sobrepasen el soporte del sistema y est\'en por ende contenidos en el conjunto de itemsets frecuentes tama\~no 1
y que ha sido almacenado en la primera posici\'on del arreglo \textit{Trees}. Lo anterior se realiza ejecutando la
siguiente l\'inea de c\'odigo:\\

\begin{codigof}[!h]
\begin{verbatim}
  transaction.loadTransaction(dataset, (AvlTree) Trees.elementAt(0));
\end{verbatim}
\caption{Llamado de \textit{loadTransaction}}
\end{codigof}

Es por eso que son enviados como par\'ametros instancias del dataset actual y del AvlTree que contiene los
Itemset Frecuentes tama\~no 1 para solo cargar del dataset los items que est\'en en este conjunto. Posterior a
este filtrado se pasa esta transacci\'on como par\'ametro a una instancia de la clase \textit{Combinations} para
encontrar sus posibles combinaciones.  Puede darse el caso de que ninguno de los items provenientes del dataset
supere el soporte, ni sea encontrado entre los Itemsets Frecuentes uno, por lo que se cargar\'ia una
transacci\'on vac\'ia, esta es descartada y se continua con la siguiente transacci\'on.\\

La clase \textit{Combinations} recibe como par\'ametro un arreglo que contiene los items validados de cada
transacci\'on provenientes del conjunto de datos.  Este arreglo es cargado en el constructor de esta clase y
posteriormente, con el llamado del m\'etodo \textit{letsCombine} de la clase \textit{Combinations}, se gener\'a
un determinado grupo de combinaciones dependiendo del tama\~no que se quiera generar, combinaciones tama\~no 2,
tama\~no 3,  etc  seg\'un sea el caso, con cada una de estas combinaciones crea objetos de la clase 
\textit{ItemSet}.  El m\'etodo \textit{letsCombine} recibe como par\'ametro un \'arbol AvlTree llamado
\textit{treeCombinations} en el cual se van insertando los objetos \textit{ItemSet} que han sido generados.\\

Se hace uso de un \'arbol balanceado AvlTree, para almacenar este tipo de conjuntos de itemsets para agilizar las
b\'usquedas de un determinado itemset generado.  Si un itemset generado a partir de un combinaci\'on ya existe en
el \'arbol \textit{treeCombinations} el soporte interno de este se incrementa en uno.  De esta manera, la primera
vez que se ejecute el m\'etodo \textit{letsCombine} se generar\'an las combinaciones tama\~no 2 de cada
transacci\'on v\'alida del conjunto de datos y quedar\'an almacenadas en \textit{treeCombinations} el total de
itemsets generados por el dataset y su correspondiente soporte.\\

Posterior a este paso se procede a barrer el arbol \textit{treeCombinations} para seleccionar aquellos Itemsets
Frecuentes que hayan superado el soporte del sistema.  Esta funci\'on se realiza haciendo uso del m\'etodo
\textit{pruneCombinations}, de la clase \textit{EquipAsso}, que recibe como par\'ametros el \'arbol 
\textit{treeCombination} y un \'arbol AvlTree llamado \textit{treeFrequents}, donde se guardaran \'unicamente los
Itemsets Frecuentes.  El m\'etodo \textit{pruneCombinations} es un m\'etodo recursivo y su implementaci\'on se
muestra en el siguiente listado:\\

\begin{codigof}[h]
\begin{verbatim}
public void pruneCombinations(AvlTree treeCombinations,
                    AvlTree treeFrequents) {
   pruneCombinations(treeCombinations.getRoot(), treeFrequents);
}

private void pruneCombinations(AvlNode node, AvlTree treeFrequents) {
  if(node != null){
  pruneCombinations(node.getLeft(), treeFrequents);
  ItemSet itemset = node.getItemset();
  if(itemset.getSupport() >= support\_of\_system){
    treeFrequents.insertItemset(itemset);
  }
  pruneCombinations(node.getRight(), treeFrequents);
  }
}
\end{verbatim}
\caption{Funci\'on \textit{pruneCombinations}}
\end{codigof}

Al final de este m\'etodo se liberan los recursos ocupados por \textit{treeCombinations} y contaremos con un
objeto \textit{treeFrequents} donde est\'an almacenados todos los Itemsets Frecuentes de un determinado tama\~no.
Para concluir el proceso se pregunta si el \'arbol \textit{treeFrequents} contiene elementos, se usa el m\'etodo
\textit{howMany} de la clase \textit{AvlTree} para este prop\'osito.  De ser as\'i, este \'arbol es almacenado en
el arreglo \textit{Trees} de la clase \textit{EquipAsso} en su respectiva posici\'on de acuerdo al tama\~no de
los itemsets que contiene y el m\'etodo \textit{findInDataset} retorna una se\~nal de \textit{true} al ciclo
principal del m\'etodo \textit{run} que controla la continuaci\'on del algoritmo.  El proceso se repetir\'a esta
vez calculando las combinaciones e Itemsets Frecuentes tama\~no 3.  El algoritmo se detiene cuando el m\'etodo
\textit{howMany} del \'arbol \textit{treeFrequents} devuelva 0 lo que quiere decir que ya no se han generado
Itemsets Frecuentes en esta iteraci\'on y el proceso de EquipAsso ha terminado.  En este caso el m\'etodo
\textit{findInDataset} retorna una se\~nal de false y el ciclo principal de la clase \textit{EquipAsso}
concluye.\\
%------------------- PAQUETE CLASSIFICATION
\subsubsection{Paquete classification}
%------------------- PAQUETE C.4.5
\paragraph{Paquete c45}
Este paquete implementa el algoritmo de clasificaci\'on C.4.5, el cual es una t\'ecnica de miner\'ia de datos que
permite descubrir conocimiento por medio de la clasificaci\'on de atributos a trav\'es de la ganancia de informaci\'on
de los mismos, aplicando formulas para obtener la entrop\'ia de cada atributo con respecto a otros.\\

En primera instancia se hace un conteo de los distintos valores en el atributo objetivo, con el prop\'osito de
encontrar una entrop\'ia inicial, con la cual calcularemos las entrop\'ias de cada atributo, para saber cual es la
que brinda la mayor ganancia de informaci\'on, el atributo con mayor ganancia es el pr\'oximo nodo de \'arbol de
decisi\'on, el cual es una estructura que inicia en el nodo ra\'iz o atributo objetivo, adem\'as esta compuesta
por nodos internos, ramas y hojas. Los nodos representan atributos, las ramas son regla de clasificaci\'on y las
hojas representan a una clase determinada.\\

El proceso es iterativo hasta que no haya atributos que clasificar. En TARYI C 45 esta implementado de la
siguiente forma: El flujo de entrada es una conjunto de datos presentados en un TabelModel, el cual permite la
conexi\'on del algoritmo con distintos filtros de la etapa datacleaning. Tambi\'en Hacemos uso  de una estructura
de \'arbol N-Ario, en una clase la cual llamamos \textit{TreeCounter}, para hacer el conteo eficiente de las
m\'ultiples combinaciones de los valores  de un atributo con respecto a otros. A continuaci\'on se presenta el
c\'odigo fuente de la estructura del \'arbol N-Ario.\\

\begin{codigof}[!h]
\begin{verbatim}
  rows = Numero de Transacciones;
  columns = Numero de Atributos;
  root = Ruta de Atributos;
  aux = Auxiliar de Atributo;
  atributos_insertados = Atributos previamente insertados en el arbol;  
  route = Ruta de Nodos;
  bd = Bandera Booleana;
    Attribute auxBrother;
    int size;
    root.frecuence--;
    size = route.getSize(); 
    if(route.firstGain) size = size - 1;
    for(int r = 0; r < rows ; r++ ) {
\end{verbatim}
\end{codigof}

\newpage
\begin{codigof}[!h]
\begin{verbatim}
      aux = root;
      root.incrementFrecuence();
      if(route.firstGain){
        route.index = 1;
      } else {
        route.resetIndex();
      }
      for(int c = 0; c < size; c++ ) {  
        String value = (String)dataIn.getValueAt(r, route.getIndex());
        if(aux.son == null){ 
          aux.son = new Attribute(value, aux, null, null);
          aux.son.route = findPath(aux.son);
          searchAttribute(aux.son);
          if(c == size - 2){
            if(r == 0){
              rootVariable = new nodeVariable(aux.son);
              currentVariable = rootVariable;
            } else {
              currentVariable.next = new nodeVariable(aux.son);
              currentVariable = currentVariable.next;
            }
          }
          aux = aux.son;
        } else { // si no esta vacio
          auxBrother = aux.son;
          aux = aux.son;
          bd = true;
          while(aux != null){
            if(aux.name.equals(value)){  
              aux.incrementFrecuence();
              bd = false;
              break;
            }
            auxBrother = aux; 
            aux = aux.brother;
          }
\end{verbatim}
\end{codigof}

\newpage
\begin{codigof}[!h]
\begin{verbatim}
          if(bd){  
            auxBrother.brother = new Attribute_
              (value, auxBrother.father, null, null);
            auxBrother.brother.route = findPath(auxBrother.brother);
            searchAttribute(auxBrother.brother);
            aux = auxBrother.brother;
          }
        }
      } 
    }
\end{verbatim}
\caption{Estructura \'arbol N-Ario}
\label{codc451}
\end{codigof}

Al resultado de este conteo, es aplicado las formulas de entrop\'ia para encontrar el atributo con mayor ganancia
de informaci\'on, la cual es obtenida a partir del siguiente criterio.\\

La ganancia de informaci\'on de un atributo A, con respecto a un conjunto de ejemplos S es:\\
\begin{displaymath}
Gain(S,A)=Entropia(S)-\sum v\in Valores(A) |Sv|/|S|Entropia(Sv)
\end{displaymath}

Donde:\\

Valores$(A)$ es el conjunto de todos los valores del atributo $A$.\\
$Sv$ es el subconjunto de $S$ para el atributo $A$ que toma valores $v$.\\
$Entropia(S)=-\sum pi\ \ Log_{2}\ \ pi$ donde $pi$ es la probabilidad que un ejemplo arbitrario pertenezca a la
clase $Ci$.\\

El siguiente fragmento de c\'odigo presenta la forma de encontrar entrop\'ia:\\

\begin{codigof}[!h]
\begin{verbatim}
  public double setEntropia(){
    double probabilidadInterna = 0.0;
    float division;
    float divExt;
    Attribute auxSon = this.son;

\end{verbatim}
\end{codigof}

\newpage
\begin{codigof}[!h]
\begin{verbatim}
    while(auxSon != null){
      division = ((float)auxSon.frecuence / (float)this.frecuence);
      probabilidadInterna += (division) * log2(division);
      auxSon = auxSon.brother;
    }
    this.entropia = probabilidadInterna;
    divExt =  (float)this.frecuence / (float)this.father.frecuence;
    return divExt * probabilidadInterna;
  }

  public double log2(double value){ 
    if(value == 0.0) return 0.0;
    return Math.log(value)/Math.log(2);
  }

\end{verbatim}
\caption{Funci\'on para encontrar la entrop\'ia}
\label{codc452}
\end{codigof}

Despu\'es de encontrar el atributo que presenta mayor ganancia de informaci\'on, es vinculado a la ruta de una
rama, para repetir el proceso recursiva e iterativa mente hasta conformar el \'arbol de decisi\'on, que es
implementado en un \'arbol eneario gr\'afico denominado \textit{FinalTree}, el cual extiende la clase de Java 
\textit{JTree}.\\

Ejemplo de funcionamiento del algoritmo C 45 en TARYI:\\

Primero en la tabla \ref{tabc451} se pueden observar datos de entrada, para el algoritmo C45:\\

\begin{table}[!t]
\begin{center}
\begin{tabular}{|l|l|l|l|l|l|}\hline
DIA & ESTADO & TEMPER. & HUMEDAD & VIENTO & Jugar Tennis \\ \hline
D1 & Soleado & Caliente & Alta & Debil & No \\ \hline
D2 & Soleado & Caliente & Alta & Fuerte & No \\ \hline
D3 & Nublado & Caliente & Alta & Debil & Si \\ \hline
D4 & Lluvioso & Templado & Alta & Debil & Si \\ \hline
D5 & Lluvioso & Fresco & Normal & Debil & Si \\ \hline
D6 & Lluvioso & Fresco & Normal & Fuerte & No \\ \hline
D7 & Nublado & Fresco & Normal & Fuerte & Si \\ \hline
D8 & Soleado & Templado & Alta & Debil & No \\ \hline
D9 & Soleado & Fresco & Normal & Debil & Si \\ \hline
D10 & Lluvioso & Templado & Normal & Debil & Si \\ \hline
D11 & Soleado & Templado & Normal & Fuerte & Si \\ \hline
D12 & Nublado & Templado & Alta & Fuerte & Si \\ \hline
D13 & Nublado & Caliente & Normal & Debil & Si \\ \hline
D14 & Lluvioso & Templado & Alta & Fuerte & No \\ \hline
\end{tabular}
\end{center}
\caption{Conjunto de datos}
\label{tabc451}
\end{table}

En la tabla \ref{tabc451} el Atributo seleccionado como target o clase es  ''JUGAR TENNIS'', lo cual significa que
el objetivo de la Miner\'ia de Datos gira en torno a este  atributo. Aplicamos el conteo, utilizando la estructura
\textit{TreeCounter}, sobre el atributo Target, con lo cual obtenemos 9 valores ''SI'' y 5 valores ''NO'', como lo
observamos en el gr\'afico \ref{grac451}.\\

\begin{figure}[!b]
\centering
\includegraphics[width=0.5\textwidth]{images/Conteo_Target.png}
\caption{Conteo target}
\label{grac451}
\end{figure}

A estos resultados aplicamos la formula para obtener la entrop\'ia inicial.\\
\begin{displaymath}
E([9+,5-])=-(9/14)\ \ Log_{2}(9/14)-(5/14)\ \ Log_{2}(5/14)=0.940
\end{displaymath}


Aplicamos nuevamente el proceso de conteo, relacionando cada atributo, con el atributo objetivo, con el
prop\'osito de encontrar aquel que brinde mayor ganancia de informaci\'on, se presenta el ejemplo de conteo con el
atributo ''VIENTO'' en el gr\'afico \ref{grac452}.

\begin{figure}[!t]
\centering
\includegraphics[width=0.7\textwidth]{images/Conteo_Viento_Target.png}
\caption{Conteo viento}
\label{grac452}
\end{figure}

Con estos resultados y la entrop\'ia inicial aplicamos la formula para obtener la entrop\'ia del atributo
''VIENTO'' relacionado con el atributo objetivo, de la siguiente forma:\\

El atributo ''VIENTO''  posee dos valores los cuales son d\'ebil y fuerte:\\
$|Sdebil |=[6+,2-] Sfuerte=[3+,3-]$\\ \\
$Gain(S,Viento)=Entropia(S)-(Sdebil*Entropia(Sdebil)+Sfuerte*Entropia(Sfuerte) = 0.940 -8/14Entropia(Sdebil) 
 -6/14Entropia(Sfuerte)$\\ \\
$Calculamos\ \ E(Sdebil)=E([6+,2-])= -(6/ 8)\ \ Log_{2}(6/ 8)-(2/ 8)\ \ Log_{2}(2/ 8)=0.811$\\
$               E(Sfuerte)=E([3+,3-])=-(3/6)log2(3/6)-(3/6)log2(3/6)=1$\\ \\
$Reemplazando:\\ 
Gain(S,viento)=0.940 -0.463-0.428 = 0.048
$

De una forma similar se aplican los anteriores procedimientos con cada atributo,  para encontrar el atributo que
brinde mayor ganancia de informaci\'on para este nivel. Los resultados obtenidos son:\\

Gain(S,Estado) = 0.246
Gain(S,Humedad) = 0.151
Gain(S,Temperatura) = 0.029\\

Podemos observar que la mayor ganancia de informaci\'on la brinda el atributo ESTADO igual a 0.246, lo cual
significa que el nodo inicial en el \'arbol de decisi\'on es ESTADO, a partir de este atributo encontraremos los
siguientes nodos que suministren mayor ganancia en los tres valores del atributo, los cuales son ''Soleado'',
''Nublado'' y ''Lluvioso'', tambi\'en encontramos que valor ''Nublado'' se parcializa hacia una decisi\'on la cual
es Jugar Tenis. El \'arbol de este nivel es el siguiente:

\begin{figure}[h]
\centering
\includegraphics[width=0.7\textwidth]{images/Arbol_1Nivel.png}
\caption{\'Arbol parcial}
\label{grac453}
\end{figure}

Ahora se encuentra el atributo que brinde la mayor ganancia para cada valor del atributo ''ESTADO'', haciendo el
conteo de forma eficiente al encontrar los resultados para los tres valores simultaneamente. Como ejemplo lo
haremos para Humedad, combinado con el atributo Estado, tal y como se puede observar en la gr\'afica \ref{grac454}.

\begin{figure}[t]
\centering
\includegraphics[width=0.8\textwidth]{images/Conteo_Humedad_Estado_Target.png}
\caption{Conteo humedad \- estado}
\label{grac454}
\end{figure}

A continuaci\'on se aplica la formula de entrop\'ia para el valor Soleado, del  atributo Estado, de la siguiente
forma\\

$Humedad(Alta[0+,3-], Normal[2+,0-])$\\
$Gain(Soleado,Humedad) = 0.970-(3/ 5)*0-(2/ 5)*0=0.970$\\ \\
$Temperatura(Caliente[0+,0-],Templado[1+,1-],fresco[1+,0-]$\\
$Gain(Soleado,Temperatura)=0.970-(2/ 5)*0-(2/ 5)*1-(1/ 5)*0=0.570$\\ \\
$Viento(Debil[1+,2-], fuerte[1+,1-])$\\
$Gain(Soleado,Viento)=0.970-(3/ 5)*0.918-(2/ 5)*1=0.019$\\ \\

Podemos observar que para el valor Soleado, el atributo ganador es Humedad cuyos valores son Alta y Normal los
cuales se parcializan en este nivel. Para el atributo Lluvioso, despu\'es de realizar el conteo se aplica la
formula de entrop\'ia:\\ \\

$Humedad(Alta[1+,1-], Normal[2+,1-])$\\
$Gain(Lluvioso,Humedad)=0.97-(2/5)*1-(3/5)*0.917= 0.0198$\\ \\
$Temperatura(Caliente[0+,0-],Templado[2+,1-],fresco[1+,1-]$\\
$Gain(Lluvioso,Temperatura)=0.97-0-(3/5)*0.917-(2/5)*1=0.0198$\\ \\
$Viento(Debil[3+,0-], fuerte[0+,2-])$\\
$Gain(Lluvioso,Viento)=0.970-(3/5)*0-(2/5)*0=0.970$

Podemos observar que el atributo ganador es Viento cuyos valores Fuerte y d\'ebil se parcializan en este nivel. A
continuaci\'on se presenta el \'arbol de decisi\'on definitivo:

\begin{figure}[!h]
\centering
\includegraphics[width=0.8\textwidth]{images/Arbol_Definitivo.png}
\caption{\'Arbol definitivo}
\label{grac455}
\end{figure}
%------------------- PAQUETE MATE
\paragraph{Paquete mate}
A continuaci\'on se describen las clases implementadas en la programaci\'on del algortimo MateBy. Se habla m\'as
en detalle acerca de las estructuras de datos y los m\'etodos m\'as utilizados.

\subparagraph{Clase MateBy}
Inicialmente es creado un objeto de la clase \textit{MateBy}. Este objeto tiene la siguiente estructura:\\

\textit{dataSet} de tipo DataSet: es un \'arbol N-ario en el que se almacenan las combinaciones que generadas y en
el cual es posible llevar un conteo de cada una de las ocurrencias de una combinaci\'on dada.\\

\textit{entros} de tipo arraylist: es un arreglo de datos en el que se va a almacenar las agrupaciones de
combinaciones que tienen que hacerce para el c\'alculo de la entropia.\\

\textit{levels }de tipo entero: se usa para almacenar el n\'umero del nivel de una hoja en el \'arbol dataSet.\\

\textit{attribute }de tipo String: aqui se almacena el nombre del atributo que se encuentra inmediatamente
despu\'es de la raiz y por el cual se inici\'o el recorrido del arbol en una b\'usqueda.\\

\textit{rules }de tipo ArrayList: aqui se almacenan las reglas generadas. Este arreglo se alimenta del \'arbol
dataset despu\'es del c\'alculo de la ganancia.\\

\textit{mateTree }de tipo Tree: este es un \'arbol N-ario se crea para ser mostrado gr\'aficamente.\\

Ya se ha descrito la estructura principal sobre la cual se va a desarrollar el algoritmo. A continuaci\'on se
describen los m\'etodos principales con las que se lleva a cabo la tarea de clasificaci\'on. Se omiten m\'etodos
triviales tales como recorridos del \'arbol, consultas e inserciones al mismo.\\

\textbf{M\'etodo MateBy}\\
Inicialmente se cargan los datos para alimentar al algoritmo. Cada transacci\'on se almacena en un vector que es
pasado como par\'ametro al m\'etodo \textit{combinations}. El atributo clase no se almacena en este vector y es
pasado como otro par\'ametro al m\'etodo \textit{combinations}. En este m\'etodo se aplica el operador MateBy
generando todas las posibles combinaciones de las transacciones con el atributo clase. Cada una de las
combinaciones es almacenada en el \'arbol N-ario \textit{dataset}, cada rama del \'arbol representa una
combinaci\'on diferente. Si se presenta el caso en el que una combinaci\'on se repite, esta no es almacenada
nuevamente. Las hojas del arbol tienen una estructura diferente al resto de nodos en el \'arbol que les permite
guardar el n\'umero de ocurrencias de una combinaci\'on. Este campo en las hojas es llamado \textit{soporte} y
ser\'a de gran utilidad posteriormente en el c\'alculo de la ganacia. La gr\'afica \ref{f1} muestra las
estructuras mencionadas anteriormente. Al final se retorna dataset.\\

\begin{figure}[t]
\centering
\includegraphics[width=0.7\textwidth]{images/dataset.png}
\caption{\'Arbol de decisi\'on}
\label{f1}
\end{figure}

\textbf{M\'etodo groupBranchs}\\
Esta clase es la encargada de agrupar ramas del \'arbol o combinaciones que se estan asociadas y que deben unirse
para luego calcular la ganancia. El criterio de agrupaci\'on se basa em la coincidencia del n\'umero de niveles y
de la rutade cada hoja  en el \'arbol. Ya que cada rama tiene un soporte asociado, al momento de agruparlas es
necesario conservar ese soporte y adem\'as calcular el soporte acumulado al agrupar distintas ramas. Tanto estos
datos como las hojas agrupadas son almacenadas en una estructura especial llamada \textit{entro}. Esta estructura
se describe en la figura \ref{f2}.

\begin{figure}[h]
\centering
\includegraphics[width=1\textwidth]{images/entro.png}
\caption{Estructura Entro}
\label{f2}
\end{figure}

Cada uno de los elementos \textit{entro} es almacenado en un arreglo llamado \textit{entros} para mejorar la
manipulaci\'on de estos objetos. El proceso continua hasta rrecorrer todas las ramas del \'arbol y hasta haber
agrupado todas las hojas relacionadas entre si.\\

\textbf{M\'etodo entroAgrupation}\\
Dentro de esta clase se recorre el arreglo \textit{entros} y se recuperan las hojas agrupadas para calcular la
entrop\'\i{}a por medio del m\'etodo \textit{calculateEntropy}. Este nuevo dato es almacenado en \textit{entro}
por cada grupo de hojas. Ver fragmento de c\'odigo \ref{c1}.\\

\begin{codigof}
\begin{verbatim}
...
if (leaf instanceof NodeF) {
  Iterator it = classValues.iterator();
  while (it.hasNext()) {
    ClassValue elem = (ClassValue) it.next();
    if ( ((NodeF)leaf).getItemF() == elem.getValue() ) {
    elem.incCounter( ((NodeF)leaf).getSupport() );
    newValue = false;
    break;
    }
  }
  if (newValue) {
    ClassValue value = new ClassValue( ((NodeF)leaf).getItemF(),
    ((NodeF)leaf).getSupport() );
    classValues.add(value);
  }
} else {
  Double token = new Double(leaf.toString());
  genEntro = genEntro + (-1)*(token / this.support);
  }
...
\end{verbatim}
\caption{C\'alculo de Entrop\'\i{}a}
\label{c1}
\end{codigof}

\textbf{M\'etodo gainCalculation}\\
Dentro de esta clase se recurre nuevamente al arreglo \textit{entros}. Hasta este momento se ha calculado la
entropia de cada grupo de hojas y ahora pasamos a agrupar nuevamente para calcular la ganancia.  La agrupaci\'on
se hace de tal manera que las rutas de cada una de las hojas almacenadas en \textit{entro} tenga el mismo n\'umero
de niveles y que los nombres de los atributos que estan inmediatamente despu\'es dela raiz coincidan, de esta
manera y debido a la organizaci\'on del \'arbol al que hacen referencia las hojas se logra organizar las
combinaciones de tal forma que se puede calcular la ganancia. Una de las partes m\'as complejas en este paso es el
control de los soportes. En el arreglo \textit{entros} se intercala, por cada grupo de hojas, el valor de la
entropia y el el soporte acumulado. Los ciclos implementados se ocupan de recorrer adecuadamente las estructuras
para poder obtener todos los datos necesarios de para calcular la ganancia. Otro punto importante es que a medida
que se recorre \textit{entros} se van comparando las ganancias obtenidas y solo se trabaja por las ramas que
obtengan el mayor valor en cada iteraci\'on.\\

\textbf{M\'etodo chooseNodes}\\
Este m\'etodo se ocupa de seleccionar los nodos que van a ir en el \'arbol de reglas. De las agrupaciones de hojas
que obtuvieron la mayor ganacia se debe establecer la hoja que hace parte de la regal que va a pasar al \'arbol de
reglas. A partir de la hoja se recorre el \'arbol de abajo hacia arriba para sacar la ruta. Esta ruta representa
la regla generada.\\

\textbf{M\'etodo buildRulesTree}\\
Este m\'etodo se encarga de construir el \'arbol de reglas que es mostrado gr\'aficamente. Este \'arbol es de tipo
Tree y ser\'a mostrado a trav\'es de un Jtree. Cada regla se decodifica al pasar del \'arbol \textit{dataset} al
\'arbol \textit{rulestree}. Para esto se utiliza el diccionario contruido al momento de cargar los datos a la
aplicaci\'on por medio del llamado al m\'etodo \textit{getRules}. Aqu\'\i{} l que se hace es hacer el proceso
inverso al de la codificaci\'on para obtener los nombres reales de los atributos. Estos nombres son los que son
mostrados finalmente en el \'arbol gr\'afico.\\ \\

\textbf{M\'etodo getRules}\\
Se instancia un arreglo de cadenas en el que se van a almacenar los nombres de los atributos que se obtengan de
cruzar los punteros de la rama en el \'arbol \textit{datdaset} y el diccinario.

%------------------- PAQUETE GUI
\subsection{Paquete GUI}
Dentro de la implementaci\'on de una interfaz gr\'afica amigable para el proyecto TariyKDD, se trabaj\'o
utilizando las funcionalidades del proyecto \textit{Matisse},   el constructor de interfaces gr\'aficas de usuario
(GUI) propia de \textit{NetBeans 5.0}  que permite un dise\~no visual de las formas y su posterior
programaci\'on.\\

Dentro del paquete \textit{gui} de TariyKDD reposan dos paquetes:  \textit{KnowledgeFlow} e \textit{Icons}.  El
primero contiene las formas utilizadas en la Interfaz principal de la aplicaci\'on.  En el paquete \textit{Icons}
se aborda la programaci\'on para cada uno de los iconos involucrados en el proceso de descubrimiento de
conocimiento y a los cuales tiene acceso el usuario para desempe\~nar una determinada tarea por ejemplo realizar
un a conexi\'on a una base de datos, ejecutar un determinado algoritmo o utilizar un visor para desplegar la
informaci\'on obtenida.
%------------------- PAQUETE KNOWLEDGEFLOW
\subsubsection{Paquete KnowledgeFlow}
Se hablar\'a primero de la implementaci\'on del paquete \textit{KnowledgeFlow}.  La interfaz principal de la
aplicaci\'on esta contenida dentro de la clase \textit{Chooser}, cuya implementaci\'on se muestra en la figura
\ref{gui001} y a su vez implementa el uso de diversas clases propias de la biblioteca gr\'afica Swing de Java como
\textit{JTabbedPanne, JSplitPane, JscrollPane, JLabel} as\'i como extensiones de la clase \textit{JPanel} donde se
implementan funcionalidades propias a la aplicaci\'on como un \'area de trabajo donde tendr\'a lugar la
construcci\'on de un experimento KDD y donde, a trav\'es de la metodolog\'ia de Arrastrar y Soltar (Drag'n Drop),
se dispondr\'an los iconos que representan una determinada tarea dentro del proceso.  Se extiende tambi\'en
\textit{JComponent} donde se implementan funcionalidades de los iconos que representan cada acci\'on dentro de la
aplicaci\'on.\\

\begin{figure}[h]
\centering
\includegraphics[width=1\textwidth]{images/gui001.png}               
\caption{Clase Chooser.  Interfaz principal de la aplicaci\'on}
\label{gui001}
\end{figure}

Dentro de la interfaz principal se pueden distinguir algunas secciones como son un Selector, en el cual se permite
escoger una etapa dentro del proceso KDD, un panel, donde se despliegan los iconos asociados a una determinada
etapa y una Area de Trabajo, donde se organizan los iconos seleccionados y se construye un experimento de
descubrimiento de conocimiento.\\

Cada una de estas partes se constituye como una nueva clase en el proyecto, que se gestionan con instancias
propias de Java las cuales se mencionaron anteriormente.  Con un \textit{JTabbedPanne} se monta la selecci\'on de
un determinado proceso desplegando etiquetas que identifican a cada uno de ellos, ''Connections'', para escoger
una conexi\'on hacia un conjunto de datos espec\'ifica, puede ser a una base de datos o cargando un archivo plano,
''Filters'', donde se da la opci\'on de escoger un determinado filtro que modificar\'a el conjunto de datos que se
haya cargado,  ''Algorithms'', secci\'on en la cual se escoje el algoritmo oportuno para realizar las tareas de
miner\'ia de datos como tal y ''Views'', donde se despliegan opciones de visualizaci\'on para organizar y mostrar
al usuario los resultados obtenidos.  Un detalle de la implementaci\'on de esta estructura se muestra en la Figura
\ref{gui002}.\\

\begin{figure}[h]
\centering
\includegraphics[width=0.4\textwidth]{images/gui002.png}               
\caption{Etiquetas dentro de Chooser para la selecci\'on de etapas }
\label{gui002}
\end{figure}

Al interactuar con las etiquetas del \textit{JTabbedPanne} se ver\'a modificado el contenido de la clase
\textit{Container}, la cual es una extensi\'on de \textit{JSplitPanne}  y divide este componente en dos, izquierda
y derecha.  En el izquierdo se cargar\'a un panel correspondiente a cada etapa seleccionada con el
\textit{JTabbedPanne}, los posibles paneles que se cargan se visualizan en la figura \ref{gui003}, y en el derecho
se carga una instancia de la clase \textit{MyCanvas}, que provee la funcionalidades de Drag'n Drop (Arrastrar y
Soltar) entre los iconos involucrados en un experimento.\\

\begin{figure}[!t]
\centering
\includegraphics[width=0.7\textwidth]{images/gui003.png}
\caption{Paneles para cada etapa del proceso KDD }
\label{gui003}
\end{figure}

Cada panel extiende a \textit{JPanel} y esta conformado por instancias de la clase \textit{JLabel}, cada una de
las cuales representar\'a un icono perteneciente a esa etapa.  En la instanciaci\'on de cada \textit{JLabel} se
carga una imagen alusiva a cada icono y un texto que lo identifica dentro del panel.  Las im\'agenes
correspondientes a cada icono, y en general todas las im\'agenes utilizadas en la herramienta, se cargan
directamente del Classpath de la aplicaci\'on dentro del paquete \textit{images}.  Es importante aclarar que en
este momento tambi\'en se asigna el nombre de cada \textit{JLabel}, en su propiedad \textit{name} a trav\'es del
m\'etodo \textit{setName}, ese mismo nombre ser\'a relacionado posteriormente para identificar el icono
seleccionado por el usuario desde el panel, luego, cada \textit{JLabel} es adicionado a su respectivo panel.  Un
fragmento de c\'odigo donde es asignado los valores para cada icono se muestra en el c\'odigo \ref{codgui1}.\\

\begin{codigof}[!h]
\begin{verbatim}
jLabel.setIcon(new ImageIcon(getClass().getResource(
                                        "/images/connection.png")));
jLabel.setText(" Connection DB ");
jLabel.setName("connection");
jLabel.setHorizontalTextPosition(javax.swing.SwingConstants.CENTER);
jLabel.setVerticalTextPosition(javax.swing.SwingConstants.BOTTOM);
panelConnections.add(jLabel);
\end{verbatim}
\caption{Asignaci\'on de valores a Iconos}
\label{codgui1}
\end{codigof}

Se puede notar en la primera l\'inea como se asigna una imagen al \textit{JLabel} utilizando el metodo
\textit{getClass().getResource(Ruta de la imagen)}.  La clase \textit{MyCanvas} es la encargada de implementar
las funcionalidades de Drag'n Drop para los iconos involucrados en un determinado experimento.   Al seleccionar
con el mouse un determinado icono desde un panel, el \textit{JLabel} asociado a ese icono es capturado escaneando
los eventos del mouse cuando un icono es presionado y posteriormente liberado.  El siguiente fragmento de
c\'odigo \ref{codgui2} ilustra como es capturado un \textit{JLabel} al ser presionado con el mouse.

\begin{codigof}[h]
\begin{verbatim}
  JLabel pressed;	
  container.addMouseListener(new java.awt.event.MouseAdapter() {
    public void mousePressed(java.awt.event.MouseEvent evt) {
      pressed = container.findComponentAt(evt.getPoint());
    }
  });
\end{verbatim}
\caption{Captura de JLabel}
\label{codgui2}
\end{codigof}

Al adicionar un \textit{MouseListener} al objeto \textit{container}, el \textit{JSplitPanne} que contiene los
paneles y el \'area de trabajo, este captura el evento \textit{pressed} del mouse cuando este ha sido presionado. 
En ese momento, el evento es capturado y a trav\'es del m\'etodo \textit{getPoint} tenemos la posici\'on dentro
del contenedor donde fue generado el click.  El m\'etodo \textit{findComponent()} devolver\'a el componente
encontrado en el punto donde fue presionado el mouse.\\

Cuando el mouse se libera, este evento es igualmente capturado y se procede a identificar cual fue el \'icono
seleccionado desde el panel para desplegarlo correctamente sobre el objeto \textit{MyCanvas}.  Esto se efect\'ua
al consultar la propiedad \textit{name} del objeto \textit{pressed} que fue capturado durante el evento anterior.
Dependiendo del nombre del componente se procede a instanciar un objeto de la clase \textit{Icon}.\\

La clase \textit{Icon} fue construida extendiendo a un \textit{JPanel} y contiene una instancia de la clase
\textit{JLabel}, que contiene la imagen y el texto asociados a ese icono, y un total de 8 instancias de la clase
\textit{Conector}, esta clase fue dise\~nada para identificar secciones dentro del icono donde el usuario pueda
hacer un click y relacionar un icono con otro.  En la figura \ref{gui004} se muestra la inclusi\'on de 2
instancias de la clase \textit{Icon} y la relaci\'on establecida entre ellos a trav\'es de sus conectores.

\begin{figure}[h]
\centering
\includegraphics[width=0.4\textwidth]{images/gui004.png}
\caption{Implementaci\'on de la clase Icon }
\label{gui004}
\end{figure}

Se implementa dentro de esta clase la funcionalidad un men\'u desplegable donde se pueda incluir funciones propias
para cada \'icono y la inclusi\'on de una animaci\'on que se activa cuando el \'icono esta realizando un proceso
determinado.  Estas funcionalidades ser\'an heredadas a todas las clases que extiendan de la clase \textit{Icon}.\\

El men\'u desplegable se logra al usar las clases \textit{JPopupMenu}  y \textit{JMenuItem}.  Por defecto, todas
las instancias de \textit{Icon} y las clases que hereden de \'el tendr\'an implementada la opci\'on
\textit{Delete}, que borrar\'a el icono del \'area de trabajo y descargar\'a la clase \textit{Icon}
correspondiente de la clase \textit{MyCanvas} que la contiene.  Para ello, se instancia un objeto de la clase
\textit{JMenuItem} y se set\'ea su propiedades de \textit{name} a ''Delete'', posteriormente se procede a
adicionar esta instancia de \textit{JMenuItem} al \textit{JPopupMenu} asociado a la clase \textit{Icon}.  Cuando
un usuario seleccione la opci\'on \textit{Delete} se dispara el m\'etodo \textit{mnuDeleteActionPerformed} que se
encarga de descargar las conexiones que tenga asociado este \'icono con otros iconos y borrar el componente del
\'area de trabajo.  Nuevas adiciones al men\'u desplegable pueden ser hechas al instanciar objetos
\textit{JMenuItem} e incluirlos dentro del \textit{JPopupMenu} de la clase \textit{Icon}.\\

La clase \textit{Conector} fue construida extendiendo \textit{JComponent} ya que deb\'ia proveer facilidades de
captura de los eventos del mouse y ser dibujado de manera especial cuando este disponible y diferente cuando ya
haya sido seleccionado.  A continuaci\'on se muestra un fragmento de c\'odigo \ref{codgui3} donde se ilustra el
dibujado de esta clase al sobrecargar el m\'etodo \textit{paint} de la clase \textit{JComponent}.

\begin{codigof}[h]
\begin{verbatim}
    public synchronized void paint(Graphics g){
        g.setColor(Color.Blue);
        g.drawRect(0, 0, 6, 6);
        if(selected){
            g.fillRect(2, 2, 3, 3);
        }
    }
\end{verbatim}
\caption{Dibujado de la clase conector}
\label{codgui3}
\end{codigof}

Vemos que al m\'etodo le llega una instancia \textit{g} del la clase \textit{Graphics} que ser\'a el objeto donde
podremos dibujar.  A trav\'es de los m\'etodos de esta clase podemos escoger colores para el conector y dibujar un
rect\'angulo de 7x7 p\'ixeles que representar\'a el \'area del conector que puede ser pulsada por el usuario para
su selecci\'on y dibujar\'a un rect\'angulo relleno en el centro del conector si este ha sido ya seleccionado.\\

La clase \textit{Icon} como tal no es incluida directamente sobre la clase \textit{MyCanvas}, o \'area de trabajo.
Existen una serie de clases que extiende a la clase \textit{Icon} y se corresponden con cada tarea desempe\~nada
dentro del proceso KDD.  Existen un total de 7 extensiones a la clase \textit{Icon} y estas son
\textit{DBConnectioIcon, FileIcon, FilterIcon, AssociationIcon, ClassificationIcon, RulesIcon} y \textit{TreeIcon}
asociadas a la conexi\'on con una base de datos, conexiones con archivos planos, filtros para la selecci\'on y
preprocesamiento de un conjunto de datos, algoritmos de asociaci\'on, algoritmos de clasificaci\'on,
visualizaci\'on de resultados de reglas de asociaci\'on y visualizaci\'on de \'arboles de decisi\'on
respectivamente.  Estas 7 clases, junto con otras que apoyan la tarea que cumplen, est\'an contenidas en 7
paquetes dentro del paquete \textit{Icons} que ser\'a explicado posteriormente y que hace parte a su vez del
paquete \textit{gui}.\\

Para cada una de las clases heredadas se maneja de manera independiente el contenido de su \textit{JPopupMenu}
as\'i como la imagen y el texto asociados a cada tipo de icono.  El resultado final para algunos tipos de icono se
muestran en la figura \ref{gui005}.

\begin{figure}[h]
\centering
\includegraphics[width=0.8\textwidth]{images/gui005.png}
\caption{Clases heredadas de la clase \textit{Icon} }
\label{gui005}
\end{figure}

La clase \textit{MyCanvas} ser\'a la encargada de gestionar las conexiones entre los \'icono y su movimiento sobre
el \'area de trabajo.  Esta clase extiende un \textit{JPanel} y monitorea los eventos del mouse referentes a los
clicks del usuario.  Los eventos que tiene contemplados son \textit{MousePressed} (mouse presionado),
\textit{MouseDragged} (click sostenido), \textit{MouseReleased} (mouse liberado), \textit{MouseClicked} (un click
sencillo).\\

Para el evento \textit{MousePressed}, se identifica si el click fue sobre un \'icono o sobre uno de sus
conectores. Si fue sobre un icono este es almacenado en la variable \textit{selectedIcon}, pero si fue sobre un
conector se almacena en la variable \textit{selectedConector}.  Para esta tarea se utiliza el m\'etodo
\textit{findComponentAt(MouseEvent)}  que se explic\'o anteriormente.  Al final de este m\'etodo, y en general
todos los m\'etodos que involucran cambios sobre el \'area de trabajo, se llama al m\'etodo \textit{repaint()}que
se encarga de redibujar la clase \textit{MyCanvas} invocando directamente el m\'etodo \textit{paint(Graphics g)}
de esta clase y que se explicar\'a posteriormente.  Se presenta el siguiente fragmento de c\'odigo \ref{codgui4}
para explicar el m\'etodo \textit{MousePressed}.\\

\begin{codigof}[h]
\begin{verbatim}
private void formMousePressed(java.awt.event.MouseEvent evt) {
    Component press = this.findComponentAt(evt.getPoint());
    //Si se presiono un Icono
    if(press instanceof Icon){
        selectedIcon = (Icon)press;
    }
    //Si se presiono un Conector
    else if(press instanceof Conector){
        selectedConector = (Conector)press;
    //Se redibuja el area de trabajo para actualizar los cambios
    repaint();
}
\end{verbatim}
\caption{Click sobre un \'icono}
\label{codgui4}
\end{codigof}

Para el evento \textit{MouseDragged} se identifica cual fue el componente seleccionado en el evento
\textit{MousePressed}, si se trata de un \'icono este m\'etodo se encarga de redibujarlo a medida que se mueve el
mouse pero si se trata de un conector, se traza una l\'inea entre el conector seleccionado y el puntero del mouse
a medida que este se mueve.  Estas dos acciones se realizan en el m\'etodo\textit{ paint(Graphics g}) de la clase
\textit{MyCanvas}.  Se muestra el siguiente fragmento de c\'odigo \ref{codgui5} para ilustrar lo hecho en este
m\'etodo.\\

\begin{codigof}[t]
\begin{verbatim}
private void formMouseDragged(java.awt.event.MouseEvent evt) {
  //Si un Icono fue ya selecconado
  if(selectedIcon != null){
    //Se captura la posicion del puntero
    int x = evt.getX();
    int y = evt.getY();
    //Se calcula la nueva posicion del icono 
    //a partir de la posicion del mouse
    selectedIcon.setLocation(x - selectedIcon.getWidth() / 2,
                             y - selectedIcon.getHeight() /2);
    //Si un Conector fue seleccionado
  } else if(selectedConector != null){
    //se capturan las coordenadas del puntero del mouse
    xMouse = evt.getX();
    yMouse = evt.getY();
  }
  //Se redibuja la forma para actualizar los cambios
  //Redibujar el Icono o trazar una linea 
  repaint();
} 
\end{verbatim}
\caption{Evento Arrastrar y soltar}
\label{codgui5}
\end{codigof}

Para el evento \textit{MouseRelease} se debe identificar igualmente cual componente esta seleccionado si se trata
de un \'icono, este evento se limita a liberar la variable \textit{selectedIcon} para no seguir cambiando su
posici\'on, si se trata de un \textit{Conector} se debe identificar en que punto es liberado, si coincide con un
conector de otro \'icono que este disponible se establecer\'a una relaci\'on entre ambos \'iconos trazando una
l\'inea entre sus conectores involucrados.   Esta relaci\'on ser\'a almacenada en un arreglo donde se guardaran
instancias de la Clase \textit{Connections}.  Esta clase se construye a partir de los dos conectores involucrados
en la relaci\'on y servir\'a posteriormente para trazar todas las relaciones existente sobre el \'area de trabajo
durante la invocaci\'on al m\'etodo \textit{paint(Graphics g)}.  Un ejemplo de una relaci\'on establecida se
puede apreciar en la figura \ref{gui004}.  Se muestra a continuaci\'on un fragmento de c\'odigo \ref{codgui6}
explicando este m\'etodo.\\

\begin{codigof}[t]
\begin{verbatim}
private void formMouseReleased(java.awt.event.MouseEvent evt) {
  Component press = this.findComponentAt(evt.getPoint());
  //Si fue liberado sobre un Conector 
  //y ya existia un Conector seleccionado
  if(press instanceof Conector) && selectedConector != null){
    //Se captura el nuevo conector
    Conector releaseConector = (Conector)press;
    //Se marca como seleccionado
    releaseConector.selected = true;
    //Se guarda la relacion en el arreglo connections 
    //de clases Connection
    connections.add(
             new Connection(selectedConector, releaseConector));
    //Se libera la variable selectedConector
    selectedConector = null;
  }
  //Si el click se libera sobre un Icono
  if(press instanceof Icon)){
    //Solo se libera la variable selected Icon
    selectedIcon = null;
  }
  //Se repinta el area de trabajo para actualizar los cambios
  repaint();
}
\end{verbatim}
\caption{Liberaci\'on del click del mouse}
\label{codgui6}
\end{codigof}

Durante el evento \textit{MouseClicked} se revisa la posibilidad de, si el click fue sobre un conector que esta
seleccionado, eliminar esa relaci\'on o si el click fue sobre el cuerpo de un \'icono, desplegar el men\'u
emergente. El siguiente fragmento de c\'odigo \ref{codgui7} ilustra este m\'etodo.\\

\begin{codigof}[t]
\begin{verbatim}
private void formMouseClicked(java.awt.event.MouseEvent evt) {
  //Se captura el componente seleccionado con el click
  Component press = this.findComponentAt(evt.getPoint());
  //Si fue presionado un Icon
  if(press instanceof Icon){
    selectedIcon = (Icon)press;
    //Si se trata de un click secundario
    if(evt.getButton() == evt.BUTTON2 
                           || evt.getButton() == evt.BUTTON3){
      //Despliega el menu emergente
      selectedIcon.getPupMenu().show(
            evt.getComponent(), evt.getX(), evt.getY());
    }
    //Libera la seleccion
    selectedIcon = null;
  } else if(press instanceof Conector){
    selectedConector = (Conector)press;
    //Si ese conector ya esta seleccionado
    if(selectedConector.selected){
      //Se elimina
      this.removeConector(selectedConector);
    }
  }
}
\end{verbatim}
\caption{Evento Mouse Clicked}
\label{codgui7}
\end{codigof}

La clase \textit{MyCanvas} tiene sobrecargado el m\'etodo \textit{paint(Graphics g)} lo que permite aprovechar las
propiedades de dibujo de la clase que extiende (\textit{JPanel}) as\'i como adicionar nuevas figuras al
interactuar sobre la clase \textit{Graphics} asociada a esta clase y que nos provee de diferentes m\'etodos para
trazar figuras, escoger colores y repintar los componentes gr\'aficos que est\'en contenidos dentro de la clase y
que hayan cambiado de posici\'on.\\

Durante la implementaci\'on de este m\'etodo primero se hace un llamado al m\'etodo \textit{paint(Graphics g)} de
la clase superior para que haga un redibujado de los componentes que contiene y de esta manera redibujar todos los
\'iconos que posee en sus nuevas posiciones as\'i como los bordes y colores propios de la clase heredada. 
Posteriormente se recorre el arreglo \textit{connections} que posee todas la conexiones entre lo conectores y los
\'iconos a los que pertenecen trazando l\'ineas que relacionan de manera visual un \'icono con uno o m\'as de
ellos. 
Al recorrer el arreglo \textit{connections} se extraen de \'el objetos de la Clase \textit{Connection}.  Esta
clase consiste de dos instancias de la Clase \textit{Conector}, identificadas con los nombres \textit{from} y
\textit{to}, haciendo alusi\'on a el conector desde donde viene la relaci\'on y hacia donde va.  Recordemos que la
Clase \textit{Conector} hereda a la clase \textit{JComponent}, es decir que herada los m\'etodos \textit{getX()} y
\textit{getY()} para calcular su posici\'on dentro del la clase que los contiene, que en este caso es
\textit{MyCanvas}.  De esta manera podemos calcular las coordenadas de los conectores y trazar las l\'ineas que
representan la relaci\'on.  A continuaci\'on se presenta un fragmento de c\'odigo \ref{codgui8} que ilustra lo
explicado anteriormente.

\begin{codigof}[t]
\begin{verbatim}
public synchronized void paint(Graphics g){
    int  xd, yd, xh, yh;
    super.paint(g);
    Iterator it = connections.iterator();
    while(it.hasNext()){
        Connection aux = (Connection)(it.next());
        xd = aux.from.getX();
        yd = aux.from.getY();
        xh = aux.to.get_X();
        yh = aux.to.get_Y();
        g.setColor(colorEdge);
        g.drawLine(xd, yd, xh, yh);
    }
}
\end{verbatim}
\caption{Conectores y trazos de l\'ineas}
\label{codgui8}
\end{codigof}
%------------------- PAQUETE ICONS
\subsubsection{Paquete Icons}
Dentro del paquete \textit{GUI} tambi\'en se encuentra contemplado un paquete llamado \textit{Icons} que se
encarga de organizar cada una de las extensiones hechas a la clase \textit{Icon} y aquellas clases que sirven de
apoyo para las acciones propias de cada una de estas clases, por ejemplo las formas de la interfaz gr\'afica
encargadas de capturar informaci\'on del usuario y que se disparan al seleccionarlas del menu contextual de cada
\'icono.\\

Cada clase se encuentra contenida dentro de un nuevo paquete lo que quiere decir que dentro del paquete Icons
existe un total de 7 nuevos paquetes que son DBConnection, File, Filters, Association, Classification, Rules y
Tree.  Durante esta explicaci\'on se abordaran los paquetes Association, Classification, Rules y Tree.  Los tres
primeros paquetes se explicar\'an con m\'as detalle en otras secciones de este cap\'itulo dado el nivel de
complejidad que estos revisten.
%------------------- PAQUETE ASSOCIATION
\paragraph{Paquete Association}
En este paquete se encuentra la clase \textit{AssociationIcon} y la clase \textit{configureSupport}.  Esta
extiende la clase \textit{Icon} adicionando dos nuevas entradas al men\'u contextual que se corresponde a
\textit{Configure}, para configurar el soporte pedido al usuario, y \textit{Run}, que ejecuta el algoritmo
escogido por el usuario y que puede contener tres estados al hacer alusi\'on al algoritmo Apriori, FPGrowth o
EquipAsso.  Estos estados se muestran en la figura \ref{gui006}.\\

\begin{figure}[t]
\centering
\includegraphics[width=0.7\textwidth]{images/gui006.png}
\caption{Estados de la Clase AssociationIcon}
\label{gui006}
\end{figure}

En la gr\'afica se puede apreciar el contenido del men\'u desplegable de cada \'icono: \textit{Delete}, por
defecto presente en todas las clases que hereden de Icon, \textit{Configure}, desplegar\'a una instancia de la
Clase \textit{configureSupport} donde el usuario introduce el soporte del sistema para este experimento.  La
figura \ref{gui007} ilustra el contenido de esta ventana.\\

\begin{figure}[h]
\centering
\includegraphics[width=0.4\textwidth]{images/gui007.png}
\caption{Captura del soporte del sistema}
\label{gui007}
\end{figure}

La ultima opci\'on del men\'u es \textit{Run}, aqu\'i esta clase har\'a instancias de las clases \textit{Apriori},
\textit{FPGrowth} o \textit{EquipAsso} seg\'un sea el caso pasando como par\'ametros un objeto de la clase
\textit{DataSet}, que puede provenir de una instancia de la clase \textit{DBConnectionIcon} o de una instancia de
\textit{FileIcon} con la cual se haya establecido una relaci\'on, y el soporte del sistema capturado con la
ventana anteriormente descrita.  El siguiente fragmento de c\'odigo  \ref{codicon1} explica lo ejecutado por este
m\'etodo.\\

\begin{codigof}[t]
\begin{verbatim}
private void mnuRunActionPerformed(java.awt.event.ActionEvent evt) {
    if(algorithm.equals("Apriori")){
        Apriori apriori = new Apriori(dataset, support);
        apriori.start();
        trees = apriori.getFrequents();
    } else if(algorithm.equals("FPGrowth")){
        FPGrowth fpgrowth = new FPGrowth(dataset, support);
        fpgrowth.start();
        trees = fpgrowth.getFrequents();
    } else if(algorithm.equals("EquipAsso")){
        EquipAsso equipasso = new EquipAsso(dataset, support);
        equipasso.start();
        trees = equipasso.getFrequents();
    }
}
\end{verbatim}
\caption{Ejecuci\'on de un comando con el mouse}
\label{codicon1}
\end{codigof}

El resultado de ejecutar cualquiera de las clases de asociaci\'on que se han implementado devolver\'a a la Clase
\textit{AssociationIcon} un Vector el cual contiene un arreglo de los arboles Avl (instancias de la Clase
\textit{AvlTree}) que organizan el conjunto de itemset frecuentes obtenidos al ejecutar un determinado algoritmo.
 Este vector es guardado en la variabel \textit{trees}, variable global a esta clase que posteriormente ser\'a
pasada a un objeto de la Clase \textit{RulesIcon} para que despliegue las reglas obtenidas al analizar el
conjunto itemsets frecuentes.\\

Las clases \textit{Apriori}, \textit{FPGrowth} y \textit{EquipAsso} son explicadas con m\'as detalle en otras
secciones de este cap\'itulo.
%------------------- PAQUETE CLASSIFICATION
\paragraph{Paquete Classification}
En este paquete se encuentra la clase \textit{ClassificationIcon}.  Esta extiende la clase \textit{Icon}
adicionando una nueva entrada al men\'u contextual que se corresponde a \textit{Run}, que ejecuta el algoritmo
escogido por el usuario y que puede contener dos estados al hacer alusi\'on al algoritmo C45 o Mate.  Estos
estados y las entradas al men\'u contextual se muestran en la figura \ref{gui008}.\\

\begin{figure}[h]
\centering
\includegraphics[width=0.5\textwidth]{images/gui008.png}
\caption{Estados de la Clase ClassificationIcon}
\label{gui008}
\end{figure}

La \'unica opci\'on implementada en el men\'u es \textit{Run} ya que este tipo de algoritmos no requiere
informaci\'on directa del usuario aparte del conjunto de datos.  Al seleccionar la opci\'on \textit{Run} esta
clase har\'a instancias de las clases \textit{C45} o \textit{Mate} seg\'un sea el caso pasando como par\'ametros
un objeto que implemente la interfaz \textit{TableModel} dentro de la variable \textit{dataIn}, que puede
provenir de una instancia de la clase \textit{FilterIcon}, donde se ha seleccionado el atributo clase del
conjunto de datos, con la cual se ha establecido una relaci\'on.  El siguiente fragmento de c\'odigo \ref{codicon2}
explica lo ejecutado por este m\'etodo.\\

\begin{codigof}[t]
\begin{verbatim}
Private void mnuRunActionPerformed(java.awt.event.ActionEvent evt) {
    if(algorithm.equals("C45")){
        c45 c = new c45(dataIn);
        c.start();
        TreePanel = c.view;
        RulesText = c.rules();
    } else if(algorithm.equals("Mate")){
        Mate mate = new Mate(dataIn);
        mate.start();
        TreePanel = c.view;
        RulesText = c.rules();
    }
}
\end{verbatim}
\caption{Ejecuci\'on de un algoritmo}
\label{codicon2}
\end{codigof}

El resultado de ejecutar cualquiera de los algoritmos de clasificaci\'on ser\'a dos objetos de las Clases
\textit{JPanel} y \textit{ArrayList}.  El primero contendr\'a una extensi\'on de \textit{JPanel} donde se
despliega un objeto de tipo \textit{JTree} que contiene gr\'aficamente el \'arbol de decisiones que resulta del
an\'alisis y el \textit{ArrayList} tendr\'a el conjunto de reglas que est\'an contenidos dentro del \'arbol de
decisiones de una manera textual.  Estas dos clases ser\'an pasadas como par\'ametros a la Clase \textit{TreeIcon}
encargada de desplegar de manera visual su contenido.\\

Las clases \textit{C45} y \textit{Mate} son explicadas con m\'as detalle en otras secciones de este cap\'itulo.\\
\\ \\
%------------------- PAQUETE RULES
\paragraph{Paquete Rules}
Despu\'es de aplicar a un conjunto de datos, cualquiera de los tres algoritmos, Apriori, EquipAsso o FPGrowth, a
trav\'es de este paquete los itemsets frecuentes obtenidos pueden ser observados en forma de reglas de
asociaci\'on.\\

El paquete \textit{Rules} hace parte del paquete \textit{Icons}, que a su vez hace parte del paquete \textit{GUI}.
Las clases que conforman este paquete son: \textit{RulesIcon, configureConfidence, RulesTableModel y showRules}.\\

En la gr\'afica \ref{rulesIcon1} se pueden ver las funciones que esta clase implementa.\\

\begin{figure}[t]
\centering
\includegraphics[width=0.8\textwidth]{images/rulesIcon1.png}
\caption{Funciones de \textit{RulesIcon}}
\label{rulesIcon1}
\end{figure}

\textit{RulesIcon} extiende a la clase \textit{Icon} que hace parte del paquete \textit{KnowledgeFlow} y a su vez
del paquete \textit{GUI}, por tanto autom\'aticamente hereda su men\'u con la opci\'on por defecto
\textit{delete}, la cual permite al usuario eliminar el icono visualizaci\'on de reglas de asociaci\'on o en el
interfaz llamado  \textit{Generator}. Por otra parte, la clase \textit{RulesIcon} a\~nade al men\'u de
\textit{Icon} las opciones Configure y Run.\\

La opci\'on \textbf{Configure} abre una peque\~na ventana, controlada por la clase \textit{configureConfidence},
la cual hace uso de un \textit{JSpinner}, clase descendiente de \textit{javax.swing}, que permite introducir en un
campo de entrada un valor num\'erico a trav\'es del cual el usuario determina la confianza con la cual se van a
mirar las reglas de asociaci\'on. La ventaja de usar un \textit{JSpinner} es la facilidad con la que se pueden
validar las entradas, ya que mediante la clase \textit{SpinnerNumberModel} se establece el l\'imite inferior y
superior que el usuario esta en condici\'on de digitar y si este introduce una entrada invalida, en cuanto se
pierda el foco del \textit{JSpinner} este tomara el \'ultimo valor correcto.\\ 

Por otra parte, la opci\'on \textbf{Run} muestra las reglas de asociaci\'on al usuario en una \textit{JTable}. Las
reglas de asociaci\'on se deben almacenar en un array de cadenas, en la clase \textit{AssocRules}. Antes la clase
\textit{AssocRules} guarda las reglas sin codificar o sea guarda los c\'odigos del \textbf{diccionario de datos}.
Sea por ejemplo el diccionario del cuadro \ref{trules1}.\\

\begin{table}[t]
\begin{center}
\begin{tabular}{|p{15mm}|p{25mm}|}\hline
\textbf{Dato c\'odificado} & \textbf{Item}\\ \hline\hline
1 & Jab\'on\\ \hline
2 & Arroz\\ \hline
3 & Champ\'u\\ \hline
4 & Desodorante\\ \hline
\end{tabular}
\end{center}
\caption{Diccionario de datos}
\label{trules1}
\end{table}

De acuerdo a lo anterior, una posible regla por ejemplo podr\'ia ser: 1\^ 2\- >3, almacenada en la clase
\textit{Rules} la cual tiene los siguientes campos:\\

\begin{footnotesize}
\textbf{Atributos:\\}
\texttt{String antecedent} //Cadena que almacena el antecedente.\\
\texttt{String concecuent} //Cadena que almacena el concecuente.\\
\end{footnotesize}

A continuaci\'on, las reglas se decodifican, o sea que la regla 1\^ 2\- >3 se almacenar\'a en el array de la clase
\textit{AssocRules}, pero de la siguiente manera: Jab\'on\^ Arroz\- >Champ\'u. De la misma manera todas las reglas
de asociaci\'on encontradas se almacenan en la clase AssocRules.\\

A partir de las reglas almacenadas en la clase \textit{AssocRules}, se construye un modelo propio para
\textit{JTable} y de esta forma se muestran las reglas en la tabla. Para construir el modelo, entonces las reglas
almacenadas en un array de cadenas se pasan al constructor de la clase \textit{RulesTableModel}, la cual
implementa los respectivos m\'etodos para construir el modelo.\\

La tabla que muestra los datos se encuentra en la clase \textit{showRules}, e implementa un evento del mouse para
que cuando el usuario haga click en el encabezado \textbf{Rules} de la tabla, las reglas se ordenen por un
criterio que ha sido determinado ''Pepita de oro'', o sea que en las primeras posiciones se indicaran las reglas
cuyo antecedente sea menor que su concecuente, por ejemplo la regla: Arroz\- >Jab\'on, Desodorante, es una
''pepita de oro'', ya que un producto lleva a comprar dos m\'as. Y cuando el usuario haga click en el encabezado
\textit{Confidence} de la tabla, las reglas se ordenaran de mayor a menor confianza. Para implementar el click
sobre el encabezado de la tabla, se hace lo siguiente:\\

Se a\~nade un ''escucha'' a la tabla a trav\'es del m\'etodo: \textit{addJTableHeaderListener}.\\

El ''escucha'' que se encuentra en la funci\'on \textit{addJTableHeaderListener}, va a estar pendiente del evento
click del mouse mediante la clase \textit{MouseAdapter}.\\

Mediante el m\'etodo de \textit{JTable}, \textit{convertColumnIndexToModel} se obtiene la columna en la cual el
usuario hizo click. La tabla que muestra las regas tiene 3 columnas, por tanto si el c\'odigo retornado es 1
quiere decir que el usuario hizo click en la segunda columna, por tanto las reglas ser\'an ordenadas de acuerdo al
criterio ''pepita de oro''. Pero si el c\'odigo retornado es 2, entonces el criterio de ordenamiento es
descendiente de acuerdo a la confianza. Para mayor detalle se puede observar el c\'odigo fuente del m\'etodo que
supervisa los eventos del mouse, a continuaci\'on: \\

\newpage
\begin{codigof}[!t]
\begin{verbatim}



  // Se a\~nade el escucha.
  this.addJTableHeaderListener();
  public void addJTableHeaderListener() {
    MouseAdapter mouseListener = new MouseAdapter() {
      public void mouseClicked(MouseEvent e) {
        TableColumnModel columnModel = tblRules.getColumnModel();
        int viewColumn = columnModel.getColumnIndexAtX(e.getX());
        System.out.println("column model " + viewColumn);
        int column = tblRules.convertColumnIndexToModel(viewColumn);
        System.out.println("column " + column);
        // Ordenar elementos de la columna de la tabla.
        if(e.getClickCount() == 1 && column != -1) {
          // Ordenar por confianza.
          if(column == 1) {
            rules.sortByGoldStone(0);
            rules.sortByGoldStone(1);
            RulesTableModel rtm = new RulesTableModel(
                                  rules.getRules());
            tblRules.setModel(rtm);
          } else if(column == 2) {
              RulesTableModel rtm = new RulesTableModel(
                                    rules.sortByConfidence());
              tblRules.setModel(rtm);
          }
        }
      }
    };
    JTableHeader header = tblRules.getTableHeader();
    header.addMouseListener(mouseListener);
  }



\end{verbatim}
\caption{M\'etodo \textit{addJTableHeaderListener}}
\end{codigof}

%------------------- PAQUETE CONEXION
 \subsection{Paquete Conexi\'on}
 El paquete DBConnection se encuentra contenido dentro del paquete \textit{Icons} que a su vez se encuentra dentro
 del paquete \textit{GUI}.  En este paquete se encuentran las clases necesarias que soportan una conexi\'on a
 bases de datos a trav\'es de un driver JDBC.  Hacen parte de este paquete las clases  \textit{DBConnectionIcon,
 connectionWizard, Table, MyCanvasTable, SelectorTable y ScrollableTableModel}.\\

\subsubsection{DBConnectionIcon} 
El paquete DBConnectionIcon es una clase que extiende a Icon del paquete GUI KnowledgeFlow y se encarga de
proveer un men\'u desplegable que gu\'ie por las etapas de conexi\'on, selecci\'on y carga de datos desde una
base de datos.  La figura \ref{db001} muestra la implementaci\'on de la clase DBConnectionIcon con sus opciones
de men\'u.\\

\begin{figure}[ht]
\centering
\includegraphics[width=0.5\textwidth]{images/db001.png}
\caption{Implementaci\'on de la Clase DBConnectionIcon}
\label{db001}
\end{figure}

Al heredar de la clase \textit{Icon, DBConnectionIcon} posee por defecto la opci\'on \textit{Delete} en su men\'u
para eliminar este \'icono del \'area de trabajo.  Al escoger la siguiente opci\'on, \textit{Configure}, se abre
una instancia de la clase \textit{connectionWizard}, la implementaci\'on de esta clase se puede ver en la figura
\ref{db002}.\\

\begin{figure}[ht]
\centering
\includegraphics[width=1\textwidth]{images/db002.png}
\caption{Implementaci\'on de la Clase DBConnectionIcon}
\label{db002}
\end{figure}

En esta se recoge la informaci\'on necesaria para establecer una conexi\'on a trav\'es del controlador JDBC
escogido en el campo JDBC Driver.  En este punto hay que aclarar que se hace uso del \textit{API SQL} de Java,
que es un conjunto de clases dispuestas a la interacci\'on con controladores JDBC y al cual se accede importando
el paquete \textit{java.sql}.  Las clases e interfaces involucradas en la conexi\'on a bases de datos son:\\

\textit{java.sql.DriverManager.\\}
Provee los servicios b\'asicos para el manejo de un conjunto de controladores JDBC\\

\textit{java.sql.DatabaseMetaData.\\}
Informaci\'on comprensiva sobre la base de datos en su totalidad.\\

\textit{java.sql.Connection.\\}
Una conexi\'on (sesi\'on) con una base de datos espec\'ifica. Se ejecutan las sentencias SQL y los resultados se
devuelven dentro del contexto la conexi\'on.\\ \\

\textit{java.sql.Statement.\\}
El objeto usado para ejecutar una sentencia est\'atica SQL y devolver los resultados que esta produce.\\

\textit{java.sql.ResultSet.\\}
Una tabla de los datos que representan un sistema del resultado de la base de datos, que es generado generalmente
ejecutando una sentencia SQL que consulta a la base de datos.\\

\textit{java.sql.SQLException.\\}
Una excepci\'on que proporciona la informaci\'on de un error durante el acceso a bases de datos u otros errores
durante la conexi\'on.\\

A partir del nombre del Driver capturado en la forma se instancia la clase del controlador a trav\'es de la
instrucci\'on:

\begin{verbatim}
  Class.forName(JDBCDriver name);
\end{verbatim}

Para el caso de PostgreSQL la instrucci\'on ser\'ia la siguiente:

\begin{verbatim}
  Class.forName("org.postgresql.Driver");
\end{verbatim}

Con la informaci\'on referente se construye la url hacia la base de datos y usando la clase
\textit{DriverManager} se obtiene una instancia de la Interface \textit{Connection}.  El siguiente fragmento de
c\'odigo ilustra esta acci\'on:

\begin{verbatim}
  url = CabeceraDelControlador + "//" + Servidor + ":" + Puerto 
                  + "/" + NombreDeLaBaseDeDatos;
  connection = DriverManager.getConnection(url, usuario, password);
\end{verbatim}

Un ejemplo de la construcci\'on de una conexi\'on a una base de datos PostgreSQL llamada \"mineria\" a trav\'es
del usuario ''nceron'' con los valores por defecto para el Servidor y el Puerto ser\'ia la siguiente:

\begin{verbatim}
  url = "jdbc:postgresql://localhost:5432/mineria";
  connection = DriverManager.getConnection(url,"nceron","t3o4g2o");
\end{verbatim}

La interface \textit{Connection} obtenida al pulsar el bot\'on Accept de la Clase \textit{connectionWizard}
devuelve una instancia de esta conexi\'on al \textit{DBConnectionIcon} que se almacena el la variable
\textit{connection}.\\

La siguiente opci\'on en el men\'u desplegable de \textit{DBConnectionIcon} es \textit{Attribute Selection}. 
Esta alternativa genera un objeto de la clase \textit{SelectorTable}.  En la figura \ref{db003} se muestra la
implementaci\'on de esta clase.  Esta clase utiliza diferentes objetos del \textit{API Swing} de Java para
desplegar y solicitar al usuario informaci\'on referente al conjunto de datos que quiere minar.  Usa un 
\textit{JComboBox} para listar las tablas que pertenecen a la bases de datos con la cual se estableci\'o conexi\'on, \textit{JRadioButtons} para solicitar al usuario como debe ser considerado el conjunto de datos, como 
un conjunto multivaluado o bajo la metodolog\'ia de canasta de mercado, un \textit{JTextArea} para mostrar la 
sentencia SQL que se esta construyendo, un \textit{JTable} que visualizar\'a el contenido de la consulta que se
logre construir y una instancia de la clase \textit{MyCanvasTable}, que es una extensi\'on de la clase
\textit{JPanel} y es utilizada para desplegar las tablas y relaciones que se establezcan para generar una
consulta de manera visual usando la metodolog\'ia Drag'n Drop.\\

\begin{figure}[ht]
\centering
\includegraphics[width=1\textwidth]{images/db003.png}
\caption{Implementaci\'on de la Clase DBConnectionIcon}
\label{db003}
\end{figure}

Durante el constructor de la clase se carga el JComboBox con los nombres de las tablas que contiene la base de
datos conectada.  Para obtener esta informaci\'on se utiliza el m\'etodo \textit{getTables} de la interface
\textit{DatabaseMetaData}.  Este m\'etodo devuelve un \textit{ResultSet} que se recorre para alimentar un Vector
que es pasado como par\'ametro en la construcci\'on del \textit{JComboBox}.  El siguiente fragmento de c\'odigo
ilustra lo anteriormente explicado.\\

\begin{codigof}
\fontsize{11}{3}
\begin{verbatim}
  public Vector getTables(){
  ResultSet rs;
  Vector names = new Vector();
  try{
  // Se instancia  DatabaseMetaData a partir de la
  // interfase connection.
  DatabaseMetaData dbmd = connection.getMetaData();
  String[] types = { "TABLE" };
  //Se captura los nombres de la tabla en un ResulSet
  //y se recorre alimentando el Vector names
  rs = dbmd.getTables("%", "%", "%", types);
  while(rs.next()){
    names.addElement(rs.getString(3));
  }
  rs.close();
  }catch(SQLException e){
   e.printStackTrace();
  }finally{
   return names;
  }
}
\end{verbatim}
\caption{Funci\'on \textit{getTables}}
\end{codigof}

A partir de las tablas que despliegue el \textit{JComboBox} podemos escoger de esa lista la(s) tabla(s) que
queremos relacionar en el an\'alisis.  La figura \ref{db004} muestra esta acci\'on.  Estas aparecer\'an dentro del
\'area del \textit{Attribute Selector}, una instancia de la Clase \textit{MyCanvasTable}, donde podemos interactuar
con las tablas de forma gr\'afica y establecer relaciones al tiempo que construimos la sentencia SQL para consultar
la base de datos.  Esta sentencia se construye al interior de la Clase \textit{JTextArea} como se puede apreciar en
la figura \ref{db005}.\\

\begin{figure}[!h]
\centering
\includegraphics[width=0.5\textwidth]{images/db004.png}
\caption{Tablas de la conexi\'on organizadas en un JComboBox}
\label{db004}
\end{figure}  

\begin{figure}[!h]
\centering
\includegraphics[width=1\textwidth]{images/db005.png}
\caption{Implementaci\'on de la Clase MyCanvasTable}
\label{db005}
\end{figure}

Para la interacci\'on de las tablas y las relaciones dentro de \textit{MyCanvasTable} se crearon las Clases
\textit{Edge, ConectorTable, Attribute,  Table}.  \textit{ConectorTable} es similar a la Clase \textit{Conector}
del paquete \textit{GUI KnowledgeFlow} y es usado para establecer y marcar relaciones entre los atributos de las
tablas involucradas.  La Clase \textit{Attribute} esta conformado por una instancia de \textit{ConectorTable} m\'as
un \textit{JLabel} con el nombre del atributo y la posibilidad de una imagen que indique su selecci\'on.  La
particularidad de la Clase \textit{ConectorTable} es que dependiendo del tipo de atributo el desplegar\'a una
imagen diferente como conector si se trata de un atributo de tipo llave primaria, llave for\'anea o llave mixta. 
Un ejemplo de esta situaci\'on se puede apreciar en la figura \ref{db007}.\\

El conjunto de objetos \textit{Attribute} conforman un objeto \textit{Table} que puede estar conformado por un
nmero n de atributos m\'as un \textit{JLabel} que sirva de t\'itulo a la tabla.  La implementaci\'on final de la
Clase \textit{Table} se puede apreciar en la figura \ref{db006}.\\

\begin{figure}[ht]
\centering
\includegraphics[width=0.5\textwidth]{images/db006.png}         
\caption{Implementaci\'on de la Clase Table}
\label{db006}
\end{figure}  

La Clase \textit{Edge} cumple la funci\'on de guardar los objetos \textit{ConectorTable} involucrados en una
relaci\'on.  De esta manera, la Clase \textit{MyCanvasTable} guarda en un arreglo los objetos \textit{Edges} que
representan las relaciones establecidas hasta el momento de manera que al recorrerlo, pueda identificar los
conectores involucrados, ubicarlos dentro de la forma y trazar l\'ineas para representar las relaciones entre
tablas.  Este procedimiento es similar al efectuado en la interfaz principal (Clase \textit{MyCanvas}) con los
diferentes tipos de \'iconos y sus relaciones.  La figura \ref{db005} ilustra esta implementaci\'on.\\

La Clase \textit{MyCanvasTable} se encarga tambi\'en de monitorear los eventos del mouse para identificar clicks
dentro de las tablas seleccionadas y marcarlos, seleccionar una tabla para su desplazamiento y marcar relaciones
entre los atributos de las tablas.\\

Al medida que se marca un atributo de una determinada tabla, este se adicionar\'a al \textit{JTextArea} que
construye la sentencia SQL dentro de su campo SELECT y se incluir\'a el nombre de la tabla de ese atributo dentro
del campo FROM, de la misma forma, al establecer una relaci\'on se identifican las tablas relacionadas y se
incluyen en el campo WHERE de la consulta.  Un ejemplo de esta situaci\'on se muestra en la figura \ref{db007}.\\

\begin{figure}[!h]
\centering
\includegraphics[width=0.8\textwidth]{images/db007.png}
\caption{Construcci\'on de la Sentencia SQL}
\label{db007}
\end{figure}

Cuando la selecci\'on de atributos ha finalizado se puede ver una preliminar de la consulta al pulsar el bot\'on
Execute.  Aqu\'i, se instancia un objeto de la Clase \textit{ScrollableTableModel} al cual se pasa como
par\'ametros el atributo connection de la clase \textit{MyCanvasTable}, que representa la conexi\'on a la base de
datos, junto con una versi\'on textual del contenido del \textit{JTextArea}, que se constituye como la sentencia
SQL que queremos consultar.\\

La Clase \textit{ScrollableTableModel} se encarga de realizar las consultas a la base de datos usando las
interfaces \textit{Statament} y \textit{ResulSet}.  La interface \textit{Statement} se crea a partir de la interface
\textit{Connection} que llega como par\'ametro en el constructor de la clase.  Esta interface, a trav\'es de su
m\'etodo \textit{executeQuery(String query)} dispara una consulta a la base de datos y retorna el resultado en una
variable de tipo \textit{ResulSet}.  El query que se pasa como par\'ametro a este m\'etodo es el que se construy\'on% previamente y se
paso como par\'ametro al constructor de esta clase.  El cuadro de c\'odigo \ref{codCone1} ilustra
este procedimiento:\\

\begin{codigof}[t]
\begin{verbatim}

  Statement statement;
  ResultSet resulset;
  try {
  //Crea una objeto Statement cuyos resultados
  //seran sensibles al scroll y de solo lectura
  statement = connection.createStatement(
        ResultSet.TYPE_SCROLL_SENSITIVE,
        ResultSet.CONCUR_READ_ONLY);
  //Ejecuta la sentencia SQL contenida en query
  resultset = statement.executeQuery(query);
  } catch (SQLException e) {
  throw new SQLException(e.getMessage(), e);
  }
\end{verbatim}
\caption{Funci\'on \textit{executeQuery(String query)}}
\label{codCone1}
\end{codigof}

La Clase \textit{ScrollableTableModel} es la encargada de recorrer el \textit{ResulSet} que contiene el resultado
de la consulta y construir un modelo de tabla que es pasado al \textit{JTable} de la Clase \textit{MyCanvasTable}
para visualizar los resultados de la consulta como se puede apreciar en la figura \ref{db008}.\\

\begin{figure}[h]
\centering
\includegraphics[width=1\textwidth]{images/db008.png}         
\caption{Construcci\'on de la Previsualizaci\'on de datos en un JTable}
\label{db008}
\end{figure}

Una vez validados los datos que se quiere minar se confirma la aceptaci\'on al pulsar el bot\'on Accept.  En ese
momento devolver\'ia la interfaz principal y el ultimo paso para construir el conjunto de datos se realiza a
trav\'es de la opci\'on Start \textit{Loading} del men contextual de la clase \textit{DBConnectionIcon} donde,
segn sea el caso, se invocar\'ia lo m\'etodos \textit{loadMarketBasketDataSet()}, para cargar una instancia de
la \textit{DataSet} que cubre el modelo de canasta de marcado (Tablas Univaluadas), o
\textit{loadMultivaluedDataSet()}, para cargar instancias de \textit{DataSet} de conjuntos multivaluados.  Esta
instancia de la Clase \textit{DataSet} reposar\'a como atributo de la Clase \textit{DBConnectionIcon} para ser
pasada en el momento de una conexi\'on a instancias de las Clases \textit{FilterIcon o AssociationIcon}.\\

%------------------- PAQUETE FILE
\subsubsection{Paquete File}
A trav\'es de este paquete el usuario puede establecer una conexi\'on con un Archivo Plano, para de esta forma
obtener el conjunto de datos. El paquete File hace parte del paquete Icons, que a su vez hace parte del paquete
GUI. Las clases que conforman este paquete son: \textit{FileIcon, OpenFile y FileTableModel}.\\

En la gr\'afica \ref{fileIcon1} se pueden ver las funciones que esta clase implementa.\\

\begin{figure}[t]
\centering
\includegraphics[width=0.8\textwidth]{images/fileIcon1.png}
\caption{Funciones de \textit{FileIcon}}
\label{fileIcon1}
\end{figure}

FileIcon extiende a la clase \textit{Icon} que hace parte del paquete \textit{KnowledgeFlow} y a su vez del
paquete \textit{GUI}, por tanto autom\'aticamente hereda su men\'u con la opci\'on por defecto \textit{delete}, la
cual permite al usuario eliminar el icono de conexi\'on al Archivo Plano. Por otra parte, la clase
\textit{FileIcon} a\~nade al men\'u de Icon las opciones Open y Load.\\

La opci\'on \textbf{Open} es la encargada de abrir una ventana en la cual el usuario elige el conjunto de datos
que va a minar, esta opci\'on es manejada a trav\'es de la clase \textit{OpenFile}. La clase \textit{OpenFile}
muestra el conjunto de datos a trav\'es de una JTable.\\

La clase de \textit{Swing, JFileChooser} mediante su m\'etodo \textit{getAbsolutePath}, retorna como una cadena de
texto la ruta completa en donde se encuentra el archivo.\\ 

La cadena de texto obtenida pasa como par\'ametro al constructor de la clase \textit{FileTableModel}, la cual se
encarga de construir una JTable a trav\'es de la cual se muestran los datos del archivo plano, tal y como se puede
observar en la figura \ref{fileIcon2}.\\

\begin{figure}[t]
\centering
\includegraphics[width=0.8\textwidth]{images/fileIcon2.png}
\caption{Visualizaci\'on del conjunto de datos}
\label{fileIcon2}
\end{figure}

A trav\'es de la clase \textit{FileTableModel} se construye un modelo propio para la JTable que indicara los
datos. Construir un modelo para una JTable es suministrarle la informaci\'on concerniente al nombre y n\'umero de
las columnas y las filas, en el caso de las filas se deben suministrar los datos.\\

Para construir un modelo propio para JTable, la clase \textit{FileTableModel}, debe extender a
\textit{AbstractTableModel}, qui\'en implementa los m\'etodos necesarios para la construcci\'on de una JTable, se
debe tambi\'en implementar los m\'etodos \textit{getColumnName(int col), getRowCount(), getColumnCount() y
getValueAt(int rowIndex, int colIndex)} de acuerdo al  tipo de estructura que almacena los datos, entonces, a
trav\'es de la cadena de texto que indica la ruta completa en donde se encuentra el archivo plano, se abre un
flujo con tal archivo y mediante el m\'etodo \textit{dataAndAtributes} de la clase \textit{FileManager}, encargada
de administrar todo lo referente a Archivos Planos, se obtiene la siguiente informaci\'on del archivo seleccionado
por el usuario: en un array de objetos retorna los nombres de las columnas y en una matriz de objetos los datos en
si. De esta forma y teniendo los m\'etodos antes mencionados debidamente implementados, los datos se mostrar\'an
en una JTable de acuerdo al modelo proporcionado. Para ilustrar mejor la forma de construir un modelo de datos
propio se puede observar en el c\'odigo fuente \ref{codfile1}, la clase \textit{FileTableModel}.\\

\begin{codigof}[!h]
\fontsize{11.5}{12}
\begin{verbatim}
public class FileTableModel extends AbstractTableModel {
  //La ruta del archivo de acceso aleatorio de tipo .arff
  private String filePath;
  //Los nombres de las columnas
  Object[] columnNames;
  //Los datos provenientes de un archivo de acceso aleatorio .arff
  Object [][] data;
  //Creates a new instance of FileTableModel
  public FileTableModel(String file) {
    filePath = file;
    FileManager fileMngt = new FileManager(filePath);
    fileMngt.dataAndAttributes(true);
    int size = fileMngt.getAttributes().length;
    columnNames = new Object[size];
    columnNames = fileMngt.getAttributes();
    int rows = fileMngt.getData().length;
    int cols = fileMngt.getData()[0].length;
    data = new Object[rows][cols+1];
    data = fileMngt.getData();
  }
  public String getColumnName(int column) {
    if (column==0) {
      return "#";
    } else {
        if (columnNames[column-1] != null) {
          return (String) columnNames[column-1];
        } else return "";
    }
  }
  //||||||||||||||| AbstractTableModel implemented methods
  public int getRowCount() {
    return data.length;
  }
  public int getColumnCount() {
    return columnNames.length+1;
  }
  public Object getValueAt(int rowIndex, int columnIndex) {
    if (columnIndex==0) return rowIndex+1;
    else return data[rowIndex][columnIndex-1];
  }
}
\end{verbatim}
\caption{Clase \textit{FileTableModel}}
\label{codfile1}
\end{codigof}

\newpage

La opci\'on \textbf{Load} se encarga de, a partir del flujo de comunicaci\'on abierto con un archivo plano, 
construir un DataSet o estructura en forma de \'arbol N-Ario para almacenar los datos de manera comprimida.

%------------------- PAQUETE FILTROS
\subsection{Paquete Filtros}

El m\'odulo \textit{filtros o data cleaning}, se encarga de hacer un refinamiento de los datos en  dos etapas, por un lado hace un proceso de limpieza sobre datos corruptos, vacios, ruidosos, inconsistentes, duplicados, alterados etc, por otro lado hace una selecci\'on de estos datos para escoger aquellos que brinden informaci\'on de calidad, aplicando distintas t\'ecnicas como son muestreos, discretizaciones y otras. De esta forma obtenemos datos depurados seg\'un el objetivo del analista, para que posteriormente se pueda aplicar el n\'ucleo \textit{KDD o de Miner\'ia de Datos} sobre datos coherentes, limpios y consistentes.\\

A este m\'odulo, pertenecen los filtros: \textit{Remove Missing, Update Missing, Selection, Range, Reduction, Codification, Replace Value, Numeric Range y Discretize}, los cuales extienden la clase  \textit{AbstractTableModel de Java}, con el objetivo de alimentarse y presentar sus resultados a trav\'es de la clase denominada \textit{TableModel}, que es el medio por el cual comunican sus flujos de datos, es as\'i como los filtros pueden recibir los datos de entrada de otros filtros o de la conexi\'on directa de a las bases de datos, pero siempre utilizando la clase \textit{TableModel}. De la misma forma los datos de salida en los filtros, se enviaran utilizando el mismoFiltro RemoveMissing formato.\\

Las clases que se encargan de Mostrar resultados y de hacer la configuraci\'on de los filtros, son todos los moodulos \textit{Show y Open}, respectivamente, estos extienden la clase de \textit{Java} denominada \textit{javax.swing.JFrame}, ya que presentan una iterfaz a modo de ventana que permite mantener un dialogo constante con el analista.\\ 

\subsubsection{Filtro RemoveMissing}
El objetivo especifico de este filtro, es eliminar todas las transacciones que contengan  campos vacios.
El filtro \textit{RemoveMissing}  consta de 2 Clases, las cuales son: \textit{RemoveMissing y  ShowRemoveMissing}. \\ \\
\textbf{RemoveMissing} \\ 
\textit{RemoveMissing} es el n\'ucleo principal del filtro \textit{RemoveMissing}, y se encarga de eliminar todas las transacciones que contengan datos nulos o vacios, haciendo una b\'usqueda de los mismos en toda la tabla de la base de datos seleccionada, y creando un nuevo conjunto de datos a partir de las transacciones completas.
A continuaci\'on se presenta el codigo \ref{PseudoRemoveMissing}, de la funci\'on encargada de hacer  la transformaci\'on. \\ \\ \\

\begin{codigof}[t]
\begin{verbatim}
  Rows = Numero de Transacciones
  Columns = Numero de Atributos
  fv = 0
    for( f = 0 ;  f < Rows;  f = f+1) {
      for( c =  0;  c < Columns;  c = c+1 ) {
        if( Atributo(f,c)  == "NULL O VACIO") { 
          if( f ? Rows -1) {
            for( i  = 0;  i < Columns;  i = i +  1) { 
              datosSalida(fv,i) = datosEntrada(f+1,i)
              datosEntrada(f, i) = Null
            }
          }   
          else
            for(  i = 0;  i < Columns;  i = i + 1 ) {
              datosSalida(f, i) = Null
            }
        }
        c = Columns
      }
      if( c == Columns -1) {
        fv = fv  + 1
          for( i = 0;  i <  Columns;  i = i + 1 ) {
            datosSalida(f, i) = Null
          }
      }
    }
\end{verbatim}
\caption{Pseudo Codigo del Filtro RemoveMissing}
\label{PseudoRemoveMissing}
\end{codigof} 

\textbf{ShowRemoveMissing}\\ 
\textit{ShowRemoveMissing} Es la clase que se encarga, tanto de mostrar los tipos de variables contenidos en la tabla, como de mostrar los datos de entrad y de salida, reportando los registros eliminados de la tabla original y los registros actuales del nuevo conjunto de datos.
Ejemplo de funcionamiento del filtro \textit{RemoveMissing}.
A continuaci\'on se presenta la tabla con los datos de entrada, ates de ser aplicado el filtro \textit{RemoveMissing}.  En la tabla se puede apreciar que hay 17 atributos vacios, en las transacciones  2,4,6,7,8,9,10,11,12,14, como se muestra en la figura \ref{IN_RemoveMissing}.\\ 

\begin{figure}[h]
\centering
\includegraphics[width=1\textwidth]{images/IN_RemoveMissing.png}
\caption{Datos de entrada antes de aplicar RemoveMissing}
\label{IN_RemoveMissing}
\end{figure}

Al aplicar el filtro \textit{RemoveMissing}, se eliminaran las trasacciones que contengan atributos vacios, las cuales son las siguientes:  2,4,6,7,8,9,10,11,12,14, dejando solo las trasacciones 1,3,5 y 13. 
La tabla resultante o la de los datos de Salida despu\'es de haber aplicado el filtro \textit{RemoveMissing}, se muestra en la grafica \ref{OUT_RemoveMissing}.\\

\begin{figure}[h]
\centering
\includegraphics[width=1\textwidth]{images/OUT_RemoveMissing.png}
\caption{Datos de Salida despues de aplicar RemoveMissing}
\label{OUT_RemoveMissing}
\end{figure}

\subsubsection{Filtro UpdateMissing}
El objetivo especifico de este filtro, es remplazar los campos vacios de un atributo especifico, por un valor otorgado por el analista.
El filtro \textit{UpdateMissing} consta de 3 clases las cuales son \textit{OpenUpdateMissing, ShowUpdate Missing} y el n\'ucleo principal la clase \textit{UpdateMissing}.\\ \\ 
\textbf{OpenUpdateMissing} \\ 
\textit{OpenUpdateMissing} extiende a la clase de \textit{Java: javax.swing.JFramees. OpenUpdateMissing} es la clase encargada de configurar el filtro seg\'un las necesidades del objetivo del analista, preguntando por el atributo a seleccionar, en el cual tendr\'an efecto los cambios que aplique el filtro seg\'un el valor a remplazar. \\ \\
\textbf{UpdateMissing} \\
\textit{UpdateMissing} es la principal clase del filtro \textit{UpdateMissing}, ya que se encarga de remplazar los campos vacios de un atributo, con un valor otorgado por el analista, creando de este modo un nuevo conjunto de datos, libre de campos vacios y con el mismo numero de transacciones originales. \\
Se presenta el codigo \ref{PseudoUpdateMissing} de la funci\'on encargada de hacer  esta transformaci\'on. \\ 

\begin{codigof}
\begin{verbatim}      
  Rows = Numero de Transacciones
  Columns = Numero de Atributos
  colRem  =  valor num\'erico de la columana del atributo seleccionado
  valRem =  Valor a remplazar
  for( f = 0;  f < Rows; f = f + 1 ) {
    if( datosEntrada(f,colRem ) == VACIO ) { 
      datosSalida(f,colRem) = valRem
    }
  }
\end{verbatim}
\caption{Pseudo Codigo del Filtro UpdateMissing}
\label{PseudoUpdateMissing}
\end{codigof}   
      
\textbf{ShowUpdateMissing} \\
\textit{ShowUpdateMissing} es la clase del filtro \textit{UpdateMissing} encargada de mostrar los tipos de variables contenidos en la tabla, y de mostrar los datos de entrada y su respectiva transformaci\'on, reportando los registros que fueron remplazados de la tabla original y los registros actuales del nuevo conjunto de datos.
Ejemplo de funcionamiento del filtro \textit{UpdateMissing}:\\ \\
A continuaci\'on se presenta la tabla con los datos de entrada, ates de ser aplicado el filtro \textit{UpdateMissing}.  En la tabla se puede observar que en el atributo \textit{''ALERGIA\_ANTIBIOTICO''} existen 4 campos vacios en las transacciones 7,8,11,12 como se muestra en la figura \ref{IN_UpdateMissing}.\\ 

\begin{figure}[h]
\centering
\includegraphics[width=1\textwidth]{images/IN_UpdateMissing.png}
\caption{Datos de Entrada antes de aplicar UpdateMissing}
\label{IN_UpdateMissing}
\end{figure}

Al aplicar el filtro \textit{UpdateMissing}, se remplaza en el atributo \textit{''ALERGIA\_ANTIBIOTICO''}, todos los campos vacios, por el valor "JC"
La tabla resultante o la de los datos de Salida despu\'es de haber aplicado el filtro \textit{UpdateMissing}, quedar\'ia como se muestra en la figura \ref{OUT_UpdateMissing}. \\

\begin{figure}[h]
\centering
\includegraphics[width=1\textwidth]{images/OUT_UpdateMissing.png}
\caption{Datos de Salida despues de aplicar UpdateMissing}
\label{OUT_UpdateMissing}
\end{figure}

\subsubsection{Filtro Selection}. \\ \\
El objetivo especifico de este filtro, es hacer una selecci\'on de atributos y del atributo objetivo sobre un conjunto de entrada.
El filtro  \textit{Selection} consta de 3 clases las cuales son \textit{OpenSelection, ShowSelection} y el n\'ucleo principal  \textit{Selection.} \\ \\

\textbf{OpenSelection} \\
\textit{OpenSelection} es la clase encargada de configurar este filtro, con el objetivo de escoger las columnas que el analista desee vincular al proceso minero. En el caso espec\'ifico de minar con un algoritmo de clasificaci\'on, el analista deber\'a escoger la columna objetivo.\\

\textbf{Selection} \\
\textit{Selection} es la principal clase del filtro \textit{Selection}, y se encarga de efectuar la selecci\'on sobre el conjunto de datos original, creando un nuevo conjunto de datos,  con los atributos escogidos y dejando el atributo objetivo al final de la tabla, para el posterior proceso minero. \\
Se presenta el c\'odigo \ref{PseudoSelection}, de la funci\'on encargada de hacer la selecci\'on. \\ 

\begin{codigof}
\begin{verbatim}       
  Rows = Numero de Transacciones
  Columns = Numero de Atributos
  colsel[ ] = Array de columnas seleccionadas
  colObje = Columna Objetivo       
  for( f == 0;  f < Rows;  f = f + 1 ) {
    for( c = 0;  c <  Columns;  c = c + 1 ) {
      if( c == Columns -1) datosSalida[f][c] = datosEntrada(f,colObje)
        else datosSalida[f][c] = datosEntrada(f,colsel[c]);
    }      
  }
\end{verbatim}
\caption{Pseudo Codigo del Filtro Selection}
\label{PseudoSelection}
\end{codigof}  
 
\textbf{ShowSelection} \\
\textit{ShowSelection} es la clase del filtro \textit{Selection} encargada de mostrar los tipos de variables contenidos en la tabla, y de mostrar los datos de entrada con el n\'umero de atributos en su forma original. Tambi\'en se encarga de mostrar el nuevo conjunto de datos, el cual solo tiene los atributos seleccionados, y al final de la tabla el atributo objetivo. Este modulo tambi\'en brinda informaci\'on sobre el numero de atributos que se eliminaron y el numero de atributos que permanecen en el conjunto de datos.\\
Ejemplo de funcionamiento del filtro Selection:\\ \\
A continuaci\'on se presenta la tabla con los datos de entrada, ates de aplicar el filtro \textit{Selection}, en la figura \ref{IN_Selection}.  En la tabla original se puede observar 6 Atributos los cuales son: \textit{''PRESION\_ARTERIAL, AZUCAR\_SANGRE, INDICE\_COLESTEROL, ALERGIA\_ANTIBIOTICO, OTRAS\_ALERGIAS, ADMINISTRAR\_FARMACO''}.
 
\begin{figure}[h]
\centering
\includegraphics[width=1\textwidth]{images/IN_Selection.png}
\caption{Datos de Entrada antes de aplicar Selection}
\label{IN_Selection}
\end{figure}

Al aplicar el filtro \textit{Selection}, se puede observar que la tabla resultante, solo contiene los atributos sobre los cuales hicimos la selecci\'on, los cuales son: \textit{AZUCAR\_SANGRE, ALERGIA\_ANTIBIOTICO } y al final de la tabla como atributo objetivo  \textit{ADMINISTRAR\_FARMACO.}
 
\begin{figure}[h]
\centering
\includegraphics[width=1\textwidth]{images/OUT_Selection.png}
\caption{Datos de Salida despues de aplicar Selection}
\label{IN_Selection}
\end{figure}

\subsubsection{Filtro Range}
El objetivo principal de este filtro, es  escoger una muestra sobre un conjunto de entrada, especialmente \'util para miner\'ia con algoritmos de clasificaci\'on. \\
El filtro  \textit{Range} consta de 3 clases las cuales son \textit{OpenRange, ShowRange} y el n\'ucleo principal, la clase \textit{Range}. \\ \\

\textbf{OpenRange} \\
\textit{OpenRange} es la clase encargada de configurar este filtro, brindando tres alternativas de muestreo, los cuales son: \textit{muestreo aleatorio, de 1 en n y los primeros n}. \\

\textbf{Range} \\
\textit{Range} es la principal clase del filtro \textit{Range}, y se encarga de hacer un muestreo sobre los datos de entrada, utilizando distintas t\'ecnicas. Es muy \'util en clasificaci\'on, ya que este tipo de miner\'ia trabaja a partir de un conjunto de entrenamiento, que lo podemos construir con este tipo de filtro.
A continuaci\'on se presenta el pseudo c\'odigo y las t\'ecnicas encargadas de hacer  el muestreo. \\ 
 
\textbf{Tecnica de Aleatorios:} \\ 
Esta t\'ecnica se encarga de seleccionar una muestra del conjunto de entrada aleatoriamente, a partir de una semilla y la funci\'on \textit{Random de Java}. \\
La estructura de esta funcion se presenta en el codigo \ref{PseudoAleatorios}.\\ 

\begin{codigof}
\begin{verbatim} 
  r  = Numero generado aleatoriamente
  Rows = Numero de Transacciones
  Columns = Numero de Atributos
  valmues = Tama\~no de la Muestra
  for( f = 0; f < valmues; f = f + 1) { 
    cfila = Numero aleatorio comprendido entre el numero de transacciones
    for( c = 0;  c < Columns;  c = c + 1) { 
      datosSalida(f,c) = datosEntrada(cfila,c) 
    }
  }  
  for( f = valmues; f < Rows; f = f + 1) { 
    for( c = 0;  c < Columns; c = c + 1 ) {
      datosSalida(f,c) = null
    }
  }
\end{verbatim}
\caption{Pseudo Codigo de la Tecnica de Aleatorios}
\label{PseudoAleatorios}
\end{codigof} 
      
\textbf{Tecnica de 1 en n:} \\ 
Esta t\'ecnica se encarga de seleccionar una muestra, que tomara cada transacci\'on en saltos de n en n, n es el valor de salto otorgado por el analista.\\
La estructura de esta funcion se presenta en el codigo \ref{Pseudode1en}. \\
         
\begin{codigof}
\begin{verbatim}	 
  filn = 0;
  valmues = Salto de Muestra
  Rows = Numero de Transacciones
  Columns = Numero de Atributos
  for( f = 0; f < Rows; f = f + 1) {
    if( f == filn) {
      for( c = 0; c < Columns; c = c + 1 ) { 
        datosSalida(fp,c) = datosEntrada.getValueAt(filn,c)
      }
      filn = filn + valmues
      fp = fp + 1
    }
 }
\end{verbatim}
\caption{Pseudo Codigo de la Tecnica de de 1 en n}
\label{Pseudode1en}
\end{codigof}

\textbf{Tecnica de Primeros n:} \\
Esta t\'ecnica se encarga de seleccionar una muestra, a partir de las primerar n transacciones del conjunto de datos de entrada.\\
La estructura de esta funcion se presenta en el codigo \ref{PseudoPrimerosn}. 
         
\begin{codigof}
\begin{verbatim}  
  valmues = Tama\~no de la Muestra
    for( f = valmues; f < Rows; f = f + 1) {  
      for( c = 0;  c < Columns; c = c + 1) {
        datosSalida(f,c) = null
      }
    }
\end{verbatim}
\caption{Pseudo Codigo de la Tecnica de Primeros n}
\label{PseudoPrimerosn}
\end{codigof}

\textbf{ShowRange} \\
\textit{ShowRange} es la clase del filtro \textit{Range} encargada de mostrar los tipos de variables contenidos en la tabla, y de mostrar los datos de entrada con el n\'umero de Transacciones original. Tambi\'en se encarga de mostrar el nuevo conjunto de datos, con las transacciones elegidas por las t\'ecnicas de muestreo. Este modulo tambi\'en brinda informaci\'on sobre el numero de Transacciones  que se eliminaron y el numero de Transacciones que permanecen en el conjunto de datos final.\\

Ejemplo de funcionamiento del filtro \textit{Range}: \\

Se presenta la tabla con los datos de entrada en la figura \ref{IN_Range}, ates de aplicar el filtro \textit{Range}.  En la tabla original se puede observar 14 Transacciones.

\begin{figure}[h]
\centering
\includegraphics[width=1\textwidth]{images/IN_Range.png}
\caption{Datos de Entrada antes de aplicar las Tecnicas de Muestreo}
\label{IN_Range}
\end{figure}

En primera instancia aplicamos la t\'ecnica de  \textit{Aleatorios}, escogiendo un tama\~no de muestra de 5 transacciones. Podemos observar que las transacciones elegidas por este m\'etodo son 5, reduciendo el cojunto original en 9 transacciones. Como se muestra en la figura \ref{aleatorio}.

\begin{figure}[h]
\centering
\includegraphics[width=1\textwidth]{images/OUT_Range_aleatorio.png}
\caption{Datos de Salida despues de aplicar la Tecnica de Aleatorios}
\label{aleatorio}
\end{figure}

Ahora aplicamos la t\'ecnica de  \textit{1 en n}, escogiendo como salto de muestra el valor 3. Podemos observar que las transacciones elegidas por este m\'etodo se eligien a partir de la primera transacci\'on en saltos de 3 en 3, reduciendo el cojunto original. Como se muestra en la figura \ref{enN}.

\begin{figure}[h]
\centering
\includegraphics[width=1\textwidth]{images/OUT_Range_1enN.png}
\caption{Datos de Salida despues de aplicar la Tecnica de 1 en n}
\label{enN}
\end{figure}

Ahora aplicamos la t\'ecnica de \textit{Primeros n}, escogiendo como Tama\~no de muestra el valor 7. Podemos observar que las transacciones elegidas por este m\'etodo son las 7 primeras transacciones del conjunto de entrada, eliminando de esta forma las 7 siguientes. Como se muestra en la figura \ref{primerosN}.

\begin{figure}[h]
\centering
\includegraphics[width=1\textwidth]{images/OUT_Range_PrimerosN.png}
\caption{Datos de Salida despues de aplicar la Tecnica de primeros n}
\label{primerosN}
\end{figure}

\subsubsection{Filtro  Reduction}
El objetivo espec\'ifico de este filtro, es hacer una reducci\'on en el n\'umero de transacciones, manteni\'endolas o elimin\'andolas, con diferentes t\'ecnicas y par\'ametros de selecci\'on.
El filtro  Reduction consta de 3 clases las cuales son \textit{OpenReduction, ShowReduction} y el n\'ucleo principal  \textit{Reduction}.\\

\textbf{OpenReduction} \\
\textit{OpenReduction} es la clase encargada de configurar este filtro, teniendo en cuenta distintas t\'ecnicas como son: por \textit{rango y por valor} que a su vez selecciona las transacciones por \textit{atributo num\'erico o alfab\'etico}, adem\'as tambi\'en se tiene en cuenta los par\'ametros de mantener o eliminar las transacciones seleccionadas.\\

\textbf{Reduction} \\
\textit{Reduction} es la principal clase del filtro \textit{Reduction}, y se encarga aplicar las t\'ecnicas de reducci\'on sobre el conjunto de datos original, creando un nuevo conjunto de datos,  con las transacciones elegidas, bien sea manteni\'endolas o elimin\'andolas dependiendo del objetivo del analista.
A continuaci\'on se presenta el pseudo c\'odigo de las distintas t\'ecnicas de reducci\'on que aplica este filtro. \\

\textbf{Tecnica de Reduccion por Rango:} \\
Esta t\'ecnica se encarga de reducir el conjunto de entrada, en un rango de transacciones, a partir de una transacci\'on inicial y otra final, de acuerdo al par\'ametro escogido por el analista, el cual puede ser eliminar o mantener dichas transacciones. \\ 
Se presenta la estructura de este procedimiento en el codigo \ref{ReduccionxRango}. \\

\begin{codigof}[!h]
\fontsize{8}{2}
\begin{verbatim} 
  Rows = Numero de Transacciones
  Columns = Numero de Atributos
  filIni = Transaccion Limitrofe Inferior
  filFin = Transaccion Limitrofe Superior
  if( Mantener las transacciones) { 
    numfil = (filFin - filIni) + 1;
    pf = 0;
    for( f = filIni;  f < filFin + 1;  f = f + 1) {
       for( c = 0;  c < Rows; c = c + 1 ) {
         datosSalida(pf,c) = datosEntrada(f,c)
       }
       pf = f + 1;
    }
    for(f = pf; f < Rows; f = f + 1) {
      for(c = 0; c < Columns c = c + 1 ) {
        datosEntrada(f,c) = null;
      }
    }
  }
  else if( Remover las Transacciones) { 
    numfil =  Rows -((filFin - filIni)+1);
    dfi = filIni;
    for( f = filFin + 1;  f < Rows; f = f + 1) {
      for( c = 0; c < Columns; c = c + 1) {
        datosSalida(dfi,c) = datosEntrada(f,c)        
      }
      dfi ++
    } 
    for(f = dfi + 1; f < Rows; f = f + 1) {
      for( c = 0; c < Columns; c = c + 1) {
        datosEntrada(f,c) = null        
      }
    } 
  } 
\end{verbatim}
\caption{Pseudo Codigo de la Tecnica de Reduccion por Rango}
\label{ReduccionxRango}
\end{codigof}
      
\textbf{Tecnica de  Reduccion de Transacciones por Atributo:} \\ 
Esta t\'ecnica se encarga de reducir el conjunto de entrada, dependiendo del tipo de datos que contenga el atributo seleccionado, los cuales pueden ser alfab\'eticos o num\'ericos, si son num\'ericos se debe suministrar un valor lim\'itrofe y si son alfab\'eticos el analista deber\'a seleccionar los atributos de su inter\'es. Adem\'as de acuerdo al par\'ametro escogido por el analista, el filtro eliminara o mantendr\'a las transacciones seleccionadas. \\
Se presenta la estructura de este procedimiento en el codigo \ref{ReduccionxAtributo}. \\

\begin{codigof}[!h]
\fontsize{8}{2}
\begin{verbatim} 
  Rows = Numero de Transacciones
  Columns = Numero de Atributos
  numatrisel = Numero de Atributos Seleccionados
  String valsAtri[ ] = Array de los valores de los Atributos 
  colsel = Columna Seleccionada
  if(Valor Numerico) {  
    numfil = 0;   
    menores = Valor lim\'itrofe Superior           
    for( f = 0; f < Rows; f = f + 1) {                 
      if(Mantener las transacciones) { 
        if( datosEntrada(f,colsel) < menores) {   
          for( c = 0; c < Columns; c = c + 1) {  
            datosSalida(numfil,c) = datosEntrada(f,c)
          }
          numfil ++;
        }    
      } 
      else if( Remover las transacciones) { 
        if(datosEntrada(f,colsel) >= menores) {   
          for( c = 0; c < Columns; c = c + 1) {  
            datosSalida(num\textbf{fil,c) = datosEntrada(f,c)
          }
          numfil ++;
        }  
      }                           
    }
    for(int f = numfil; f < Rows; f = f + 1ShowReduction ) {
      for(int c = 0; c < Columns; c = c + 1) {
        datosEntrada(f,c) = null;        
      }
    }
  }
  else if(Valor es Alfab\'etico) { 
    valsAtri[ ] = Valores de los Atributos
    if(Mantener Transacciones) { 
      for(int f = 0; f < Rows; f = f + 1) {
        filen = 0
        for(i i = 0; i < numatrisel; i++) {
          if(datosEntrada(f,colsel) = (valsAtri[i])) {
            filen ++;
          }
        }
        if(filen != 0) {
          for(int c = 0; c < Columns; c = c + 1) {  
            datosSalida(numfil,c) = datosEntrada(f,c)
          }
          numfil ++;
        }
      }
     }
     else if(Remover Transacciones) {
       for( f = 0; f < Rows; f = f + 1) {
         filen = 0
         for( i = 0; i < numatrisel; i++) {
           if(datosEntrada(f,colsel) = (valsAtri[i])) {
             filen ++
           }
         }
         if(filen == 0) {
           for( c = 0; c < Columns; c = c + 1) {  
             datosSalida(numfil,c) = datosEntrada(f,c)
           }
           numfil ++;
         }
       }        
     }
     for( f = numfil; f < Rows; f = f + 1) {
       for( c = 0; c < Columns; c = c + 1) {
         datosEntrada(f,c) = null        
       }
     }///////////
  }
\end{verbatim}
\caption{Pseudo Codigo de la Tecnica de Reduccion por Atributo}
\label{ReduccionxAtributo}
\end{codigof}


\textbf{ShowReduction} \\
\textit{ShowReduction} es la clase del filtro \textit{Range} encargada de mostrar los tipos de variables contenidos en la tabla, y de mostrar los datos de entrada con el n\'umero de transacciones original. Tambi\'en se encarga de mostrar el nuevo conjunto de datos, con las transacciones que permanecieron despu\'es de aplicar la reducci\'on. Este modulo tambi\'en brinda informaci\'on sobre el numero de Transacciones  que se eliminaron y el numero de Transacciones que permanecen en el conjunto de datos final. \\

Ejemplo de funcionamiento del filtro Reduction:
Se presenta la tabla con los datos de entrada, ates de aplicar el filtro Reduction, en la figura \ref{IN_Reduction}.  En la tabla original se puede observar 14 Transacciones. \\

\begin{figure}[h]
\centering
\includegraphics[width=1\textwidth]{images/IN_Reduction.png}
\caption{Datos de Entrada antes de aplicar las Tecnicas de Reducci\'on}
\label{IN_Reduction}
\end{figure}

En primera instancia aplicamos la t\'ecnica de \textit{Reducci\'on por Rango},  y como par\'ametro de selecci\'on, \textit{mantener el rango}. Escogemos la transacci\'on 5 como limite inicial y la transacci\'on 10 como limite final, lo cual significa que se mantendr\'a el rango a partir de la transacci\'on 5 (sin ser incluida), hasta la transacci\'on 10 (siendo incluida).\\   
Se observar que las transacciones elegidas por este m\'etodo son 5, las cuales son: 6,7,8,9 y 10, reduciendo el conjunto original en 9 transacciones, como se muestra en la figura \ref{Reduction_Prango_M}.\\

\begin{figure}[h]
\centering
\includegraphics[width=1\textwidth]{images/OUT_Reduction_Prango_M.png}
\caption{Reducci\'on por rango, Manteniendo los datos}
\label{Reduction_Prango_M}
\end{figure}

Ahora aplicamos la  misma t\'ecnica \textit{de Reducci\'on por rango}, pero como par\'ametro de selecci\'on, removemos dicho rango. Escogemos la transacci\'on 5 como limite inicial y la transacci\'on 10 como limite final, lo cual significa que se remover\'a el rango comprendido a partir de la transacci\'on 5 (sin ser incluida), hasta la transacci\'on 10 (siendo incluida), como se muestra en la figura \ref{Reduction_Prango_R}.\\
   
\begin{figure}[h]
\centering
\includegraphics[width=1\textwidth]{images/OUT_Reduction_Prango_R.png}
\caption{Reducci\'on por rango, Removiendo los datos}
\label{Reduction_Prango_R}
\end{figure}

Apliquemos ahora la \textit{T\'ecnica de  Reducci\'on de Transacciones por Atributo}, con valores Alfab\'eticos, y escojamos como atributo de reducci\'on a \textit{''INDICE\_COLESTEROL''}, y como valor de este atributo a \textit{''ALTO''}. Tambi\'en escogemos como par\'ametro de selecci\'on, \textit{Mantener el conjunto}. Esto significa que el filtro buscara cualquier valor diferente a \textit{''ALTO''} en este atributo, y eliminara su transacci\'on correspondiente.
Podemos observar que las transacciones elegidas por este m\'etodo son las siguientes: 1,2,4,5,6,7,10,12 y 13, eliminando de esta forma 5 transacciones, como se muestra en la figura \ref{Reduction_PatributoA_M}.\\
   
\begin{figure}[h]
\centering
\includegraphics[width=1\textwidth]{images/OUT_Reduction_PatributoA_M.png}
\caption{Reducci\'on por atributo Alfabetico, Manteniendo los datos}
\label{Reduction_PatributoA_M}
\end{figure}

Ahora aplicamos la misma T\'ecnica de Reducci\'on de Transacciones por Atributo, con valores Alfab\'eticos, y escogemos como atributo de reducci\'on a \textit{''INDICE\_COLESTEROL''}, y como valor de este atributo a \textit{''ALTO''}. Pero ahora escogemos como par\'ametro de selecci\'on, \textit{Remover el conjunto}. Esto significa que el filtro buscara los valores del atributo iguales a \textit{''ALTO''} y eliminara su transacci\'on correspondiente.
Podemos observar que las transacciones elegidas por este m\'etodo son las siguientes: 3,8,9,11 y 14, eliminando de esta forma 9 transacciones, como se muestra en la figura \ref{Reduction_PatributoA_R}.\\
   
\begin{figure}[h]
\centering
\includegraphics[width=1\textwidth]{images/OUT_Reduction_PatributoA_R.png}
\caption{Reducci\'on por atributo Alfabetico Removiendo los datos}
\label{Reduction_PatributoA_R}
\end{figure}

Apliquemos ahora la \textit{T\'ecnica de  Reducci\'on de Transacciones por Atributo}, con valores  Num\'ericos, y escogemos como atributo de reducci\'on a \textit{''ADMINISTRAR\_FARMACO''}, y  como limite restrictivo superior, al valor 5. \\
Tambi\'en escogemos como par\'ametro de selecci\'on, \textit{Mantener el conjunto}. Esto significa que el filtro eliminara todas las transacciones, con valores superiores o iguales a 5 en este atributo.
Podemos observar que las transacciones elegidas por este m\'etodo son las siguientes: 1,2,5,7,11 y 13, eliminando de esta forma 8 transacciones, como se muestra en la figura \ref{Reduction_PatributoN_M}.\\
   
\begin{figure}[h]
\centering
\includegraphics[width=1\textwidth]{images/OUT_Reduction_PatributoN_M.png}
\caption{Reducci\'on por atributo Numerico Manteniendo los datos}
\label{Reduction_PatributoN_M}
\end{figure}

Ahora aplicamos la misma T\'ecnica de \textit{Reducci\'on de Transacciones por Atributo, con valores  Num\'ericos}, y escogemos como atributo de reducci\'on a \textit{''ADMINISTRAR\_FARMACO''}, y  como limite restrictivo superior, al valor 5. Pero ahora escogemos como par\'ametro de selecci\'on, \textit{Remover el conjunto}. Esto significa que el filtro eliminara todas las transacciones, con valores inferiores a 5 en este atributo.
Podemos observar que las transacciones elegidas por este m\'etodo son las siguientes: 3,4,6,8,9,10,12 y 14, eliminando de esta forma 6 transaccionesm, como se muestra en la figura \ref{Reduction_PatributoN_R}.\\
   
\begin{figure}[h]
\centering
\includegraphics[width=1\textwidth]{images/OUT_Reduction_PatributoN_R.png}
\caption{Reducci\'on por atributo Numerico Removiendo los datos}
\label{Reduction_PatributoN_R}
\end{figure}

\subsubsection{Filtro Codification}
El objetivo espec\'ifico de este filtro, es realizar una codificaci\'on sobre el conjunto de datos de entrada.
El filtro  Codification consta de 2 clases las cuales son \textit{ShowCodification} y el n\'ucleo principal la clase \textit{Codification}. \\

\textbf{Codification} \\
\textit{Codification} es la principal clase del filtro \textit{Codification}, ya que se encarga de realizar la codificaci\'on, asignando un c\'odigo \'unico a cada valor de un atributo, a lo largo de toda la tabla, adem\'as crea un diccionario de datos, para su posterior decodificaci\'on.
Se presenta la estructura de la funci\'on encargada de hacer  la codificaci\'on, en el c\'odigo \ref{codigoCodification}. \\

\begin{codigof}[!h]
\fontsize{8}{2}
\begin{verbatim}  
  Rows = Numero de Transacciones
  Columns = Numero de Atributos
  for(f = 0; f < Rows; f = f +1 ) {
    for(int c = 0; c < Columns; c = c + 1 ) {
     for(int i = 0; i < valatricod.getRowCount(); I = I + 1 ) {              
      if(NobreAtributo(c)==(valatricod(i,1))_
         &&datosEntrada(f,c)==(valatricod(i,2)) ) {
        datos[f][c] = (i,0);
        break;
      } 
     } 
    } 
   }
   for(int c = 0; c < Columnsc++) {
     nomcol[c] = NombreAtributo(c); 
   }
\end{verbatim}
\caption{Pseudo Codigo de Codification}
\label{codigoCodification}
\end{codigof}

\textbf{ShowCodification} \\
\textit{ShowCodification} es la clase del filtro \textit{Codification} encargada de mostrar los tipos de variables contenidos en la tabla, y de mostrar los datos de entrada originales. Tambi\'en se encarga de mostrar la tabla con los datos codificados, y el diccionario de datos con su respectivo \'indice de codificaci\'on, valor y atributo al que pertenece.\\
Ejemplo de funcionamiento del filtro Codification:\\
Se presenta la tabla con los datos de entrada, ates de aplicar el filtro Codification, en la figura \ref{figCodification}. \\

\begin{figure}[h]
\centering
\includegraphics[width=1\textwidth]{images/IN_Reduction.png}
\caption{Reducci\'on por atributo Numerico Removiendo los datos}
\label{figCodification}
\end{figure}


Despues de aplicar el filtro \textit{codification} al conjunto de entrada, la tabla codificada queda como se muestra en la figura \ref{OUT_Codification}. \\

\begin{figure}[h]
\centering
\includegraphics[width=1\textwidth]{images/OUT_Codification.png}
\caption{Datos de Salida, despues de aplicar la Codificaci\'on}
\label{OUT_Codification}
\end{figure}

Paralela mente se crea el \textit{diccionario de datos}, para la posterior decodificaci\'on, como se muestra en la figura \ref{Diccionario}. \\

\begin{figure}[h]
\centering
\includegraphics[width=1\textwidth]{images/Diccionario_Codification.png}
\caption{Diccionario de datos}
\label{Diccionario}
\end{figure}


\subsubsection{Filtro ReplaceValue}
El objetivo espec\'ifico de este filtro, es remplazar uno o varios valores de un atributo seleccionado, por otro valor suministrado por el analista.
El filtro  RemplaceValue consta de 3 clases las cuales son: \textit{OpenRemplaceValue, ShowRemplaceValue} y el n\'ucleo principal la clase \textit{RemplaceValue}. \\

\textbf{OpenRemplaceValue} \\ 
\textit{OpenRemplaceValue} es la clase encargada de configurar este filtro, solicitando seleccionar un atributo y los valores del mismo sobre los cual se har\'a el remplazo.\\

\textbf{RemplaceValue} \\ 
\textit{RemplaceValue} es la principal clase del filtro \textit{RemplaceValue}, ya que se encarga de realizar el remplazo en los valores seleccionados del atributo escogido, por el nuevo valor.
Se presenta la estructura de la funci\'on encargada de hacer  el remplazo, en el c\'odigo \ref{codRemplaceValue}. \\
      
\begin{codigof}[!h]
\begin{verbatim}       
  Rows = Numero de Transacciones
  Columns = Numero de Atributos
  ColSel = Atributo Seleccionado
  atrisel[ ] =  Array de Valores Seleccionados del Atributo
  remcon = Valor con el cual se har\'a el remplazo. 
  numatri = Numero de Atributos     
  for( f = 0;  f < Rows;  f = f + 1 ) {
    filen = 0;
    for( i = 0; i < numatri; i++) {
      if(datosEntrada(f,ColSel) == (atrisel[i])) {
        filen ++;
      }
    }
    if(filen != 0) {                     
      datosEntrada(f,ColSel) = remcon;
    }
  } 
\end{verbatim}
\caption{Pseudo Codigo de RemplaceValue}
\label{codRemplaceValue}
\end{codigof}

\textbf{ShowRemplaceValue} \\ 
\textit{ShowRemplaceValue}
es la clase del filtro \textit{RemplaceValue} encargada de mostrar los tipos de variables contenidos en la tabla, y de mostrar los datos de entrada originales. Tambi\'en se encarga de mostrar la nueva tabla con los remplazos realizados. \\
 
Ejemplo de funcionamiento del filtro \textit{RemplaceValue}:
Se presenta la tabla con los datos de entrada, ates de aplicar el filtro RemplaceValue, en la figura\ref{IN_RemplaceValue}. \\

\begin{figure}[h]
\centering
\includegraphics[width=1\textwidth]{images/IN_RemplaceValue.png}
\caption{Datos de Entrada, antes de aplicar el filtro RemplaceValue}
\label{IN_RemplaceValue}
\end{figure}

El atributo \textit{''PRESION\_ARTERIAL''} tiene 3 valores, los cuales son \textit{''Baja'',  ''Media'' y ''Alta''}, seleccionamos los valores \textit{''Baja'' y ''Alta''} del atributo, para ser remplazados por el valor \textit{''Extremos''}, por consiguiente la nueva tabla quedar\'ia como se muestra en la figura \ref{OUT_RemplaceValue}. \\

\begin{figure}[h]
\centering
\includegraphics[width=1\textwidth]{images/OUT_RemplaceValue.png}
\caption{Datos de Salida, despues de aplicar el filtro RemplaceValue}
\label{OUT_RemplaceValue}
\end{figure}

\subsubsection{Filtro NumericRange}
El objetivo espec\'ifico de este filtro, es eliminar los valores de un atributo num\'erico, que est\'en por fuera de un rango determinado por el analista.
El filtro  \textit{NumericRange} consta de 3 clases las cuales son \textit{OpenNumericRange, ShowNumericRange} y el n\'ucleo principal la clase \textit{NumericRange}. \\

\textbf{OpenNumericRange} \\ 
\textit{OpenNumericRange} es la clase encargada de configurar el filtro seg\'un las necesidades del objetivo del analista, desplegando todos los atributos num\'ericos que posea el conjunto de datos, para ser la selecci\'on sobre un atributo determinado, tambi\'en se debe suministra el rango comprendido entre un m\'inimo y un m\'aximo valor. \\

\textbf{NumericRange} \\ 
\textit{NumericRange} es la principal clase del filtro \textit{NumericRange}, ya que se encarga de eliminar los valores de un atributo num\'erico que est\'en por fuera de un rango determinado, remplaz\'andolos con valores nulos.
Se presenta la estructura de la funcion encargada de hacer la transformacion, en el c\'odigo \ref{codNumericRange}. \\
      
\begin{codigof}[!h]
\begin{verbatim}      . 
  Rows = Numero de Transacciones
  Columns = Numero de Atributos
  colSel = Atributo Numerico Seleccinado 
  min = Limite inferior del rango 
  max = Limite Superior del rango
  for( f = 0; f < Rows; f = f + 1 ){ 
    if(datosEntrada(f,colSel)<min \'o datosEntrada(f,colSel) > max) { 
      datosEntrada(f,colSel) = null
    }
  }
\end{verbatim}
\caption{Pseudo Codigo de NumericRange}
\label{codNumericRange}
\end{codigof}
      
\textbf{ShowNumericRange} \\ 
\textit{ShowNumericRange}       
Es la clase del filtro \textit{NumericRange} encargada de mostrar los tipos de variables contenidos en la tabla, y de mostrar los datos de entrada originales. Tambi\'en se encarga de mostrar la tabla con los valores eliminados.
Ejemplo de funcionamiento del filtro \textit{NumericRange}:
Se presenta la tabla con los datos de entrada, ates de aplicar el filtro NumericRange, en la figura \ref{IN_NumericRange}. \\

\begin{figure}[h]
\centering
\includegraphics[width=1\textwidth]{images/IN_NumericRange.png}
\caption{Datos de Entrada, antes de aplicar el filtro NumericRange}
\label{IN_NumericRange}
\end{figure}

Al aplicar el filtro \textit{NumericRange}, con un l\'imite inferior igual a 2 y un l\'imite superior igual a 5, sobre el atributo num\'erico \textit{''ADMINISTRAR\_FARMACO''}, se eliminan 9 valores, permaneciendo en la tabla los valores comprendidos en dicho rango incluyendo los valore limites 2 y 5. El nuevo conjunto de datos se muestra en la figura \ref{OUT_NumericRange}. \\


\begin{figure}[h]
\centering
\includegraphics[width=1\textwidth]{images/OUT_NumericRange.png}
\caption{Datos de Salida, despues de aplicar el filtro NumericRange}
\label{OUT_NumericRange}
\end{figure}

\subsubsection{Filtro Discretize}
El objetivo espec\'ifico de este filtro, transformar un valor num\'erico discontinuo en un rango continuo.
El filtro \textit{Discretize} consta de 3 clases las cuales son \textit{OpenDiscretize}, \textit{ShowDiscretize} y el n\'ucleo principal la clase \textit{Discretize}. \\

\textbf{OpenDiscretize} \\ 
\textit{OpenDiscretize} es la clase encargada de configurar el filtro \textit{Discretize} desplegando todos los atributos num\'ericos que posea el conjunto de datos, para ser la selecci\'on sobre un atributo determinado, tambi\'en presenta dos opciones de discretizaci\'on las cul\'es son: \textit{discretizar con n\'umero de rangos y discretizar con el tama\~no del rango}. \\

\textbf{Discretize} \\ 
\textit{Discretize} es la principal clase del filtro \textit{Discretize}, ya que se encarga de efectuar la discretizacion, por medio de dos t\'ecnicas: \textit{discretizar con n\'umero de rangos y discretizar con el tama\~no del rango}, llevando un valor discontinuo a un formato continuo.
Se presenta las variables que usa \textit{Discretize}, para la aplicacion de sus tecnicas en el c\'odigo \ref{codDiscretizeVar}. \\
 
\begin{codigof}[!h]
\begin{verbatim}
  float rangos[ ] = Array de valores limites de rangos
  rangString = Formato texto del rango
  colSel = Atributo Num\'erico Seleccionado
  val = Valor de divisi\'on de los rangos
  Rows = Numero de Transacciones
  Columns = Numero de Atributos
  min = limite Inferior;
  max = limite Superior;
  for( f = 1; f < rows; f++ ){ 
     if(datosEntrada(f,colSel) < min) {
       min = datosEntrada(f,colSel); 
     }
     if(datosEntrada(f,colSel) > max) {  
        max = datosEntrada(f,colSel); 
     }
  } 
\end{verbatim}
\caption{Argumentos de las variables}
\label{codDiscretizeVar}
\end{codigof}
   
\textbf{Tecnica de discretizar con n\'umero de rangos:} \\ 
Esta t\'ecnica se encarga de tomar el m\'inimo y el m\'aximo valor del atributo num\'erico seleccionado, con el objetivo de hacer una divisi\'on clasificatoria,  dependiendo del n\'umero de rangos otorgado por analista y tambi\'en teniendo en cuenta los rangos extremos: desde \-infinito hasta el m\'inimo valor y desde el m\'aximo valor hasta +infinito, como se muestra en el codigo \ref{codDconNRangos}. \\
 
\begin{codigof}[!h]
\fontsize{9}{2}
\begin{verbatim} 
  val = val - 2;    
  aux = max - min;
  aux = aux / val;
  incre = min;  
  con = 0;
  while( con < val) {
    rangos[con] = incre;
    incre = incre + aux;
    con ++;
  }
  rangos[con] = max - 1;
  for(int f = 0; f < rows; f = f + 1 ){ 
    for( i = 0; i < val; I = I + 1 ){
      if(datosEntrada(f,colSel) == (min)) {
        rangString = "(- Infinity : " + min + " ]"; 
        data(f,colSel) = rangString;
      }else if(datosEntrada(f,colSel) == (max)) {
          rangString = "[ " + max + " : + Infinity )"; 
          data(f,colSel) = rangString;
      }elseif(datosEntrada(f,colSel)>rangos[i] &&_
       datosEntrada(f,colSel) <= rangos[i+1]) { 
           rangString = "( " + rangos[i] + " : " + rangos[i+1] + " ]"; 
           data(f,colSel) = rangString;
        }
      }
  } 
\end{verbatim}
\caption{Pseudo Codigo de discretizaci\'on con n\'umero de rangos:}
\label{codDconNRangos}
\end{codigof}


\textbf{T\'ecnica de discretizar con el tama\~no del rango} \\        
Esta t\'ecnica se encarga de  delimitar rangos, en incrementos, seg\'un el tama\~no del rango otorgado por el analista, dentro de los valores comprendidos entre el m\'inimo y m\'aximo valor del atributo num\'erico seleccionado, tambi\'en se tiene en cuenta los rangos extremos los cuales son desde el \-infinito hasta el minimo valor, y desde el m\'aximo valor hasta el + infinito, como se muestra en el codigo\ref{codDconTRangos}. \\
 
\begin{codigof}[!h]
\begin{verbatim} 
  pr = 0;
  incre = min;
  while(incre < max){
    rangos[pr] = incre;
    incre = incre + val;
    pr ++;
  }
  rangos[pr] = max - 1;        
  for(f = 0; f < rows; f++ ){ 
    for( i = 0; i < pr+1; i++ ){
      if(datosEntrada(f,colSel) == (min)) {
        rangString = "(- Infinity : " + min + " ]"; 
        data(f,colSel) = rangString;
      }else if(datosEntrada(f,colSel) == (max)) {
         rangString = "[ " + max + " : + Infinity )"; 
         data(f,colSel) = rangString;
      }else if(datosEntrada(f,colSel)>rangos[i] && 
       datosEntrada(f,colSel) <= rangos[i+1]) { 
         rangString = "( " + rangos[i] + " : " + rangos[i+1] + " ]"; 
         data(f,colSel) = rangString;
      }
    }
  } 
\end{verbatim}
\caption{Pseudo Codigo de discretizaci\'on con el tama\~no del rango}
\label{codDconTRangos}
\end{codigof}	         


\textbf{ShowDiscretize} \\ 
\textit{ShowDiscretize} es la clase del filtro \textit{Discretize} encargada de mostrar los tipos de variables contenidos en la tabla, y de mostrar los datos de entrada. Tambi\'en se encarga de mostrar el nuevo conjunto de datos, con los valores num\'ericos discretizados en el atributo num\'erico seleccionado. \\
Ejemplo de funcionamiento del filtro \textit{Discretize:}

Se presenta la tabla con los datos de entrada, ates de aplicar el filtro Discretize en la figura\ref{INDiscretize}. \\

\begin{figure}[h]
\centering
\includegraphics[width=1\textwidth]{images/IN_Reduction.png}
\caption{Datos de Entrada, antes de aplicar el filtro Discretize}
\label{INDiscretize}
\end{figure}  

En primera instancia aplicamos la t\'ecnica de discretizar con n\'umero de rangos, seleccionamos al atributo num\'erico \textit{''ADMINISTRAR\_FARMACO''}, y segmentaremos al conjunto en 3 rangos, lo cual significa que se construir\'an 3 rangos con valores continuos, de los cuales 2 pertenecen a los rangos extremos: desde el m\'inimo valor hasta \-infinito y desde el m\'aximo valor hasta +infinito, ubicando de esta forma el valor del atributo en un rango determinado.\\ 
La tabla con los valores discretizados con esta t\'ecnica, se muestra en la figura\ref{figDiscretizeNumeroRangos}. \\

\begin{figure}[h]
\centering
\includegraphics[width=1\textwidth]{images/OUT_Discretize_NumeroRangos.png}
\caption{Datos de Salida, antes de aplicar el filtro Discretize con Numero de Rangos}
\label{figDiscretizeNumeroRangos}
\end{figure}  

Ahora aplicamos la t\'ecnica de discretizar con \textit{el tama\~no del rango}, seleccionamos el atributo num\'erico \textit{''ADMINISTRAR\_FARMACO''}, y escogemos como tama\~no del rango el valor 3, lo cual significa que los rangos se construir\'an en incrementos  de 3, de esta forma se  crean 5 rangos con valores continuos, de los cuales 2 pertenecen a los rangos extremos: desde el m\'inimo valor hasta \-infinito y desde el m\'aximo valor hasta +infinito. 
La tabla con los valores discretizados con esta t\'ecnica,se muestra en la figura\ref{DiscretizeTamanoRango}. \\

\begin{figure}[h]
\centering
\includegraphics[width=1\textwidth]{images/OUT_Discretize_TamanoRango.png}
\caption{Datos de Salida, antes de aplicar el filtro Discretize con el tama\~no del rango}
\label{DiscretizeTamanoRango}
\end{figure}

%\section{Implementaci\'on}
\subsection{Arquitectura de TariyKDD}
Para el desarrollo de TariyKDD se utilizaron computadores con procesador AMD 64 bits, disco duro Serial ATA,
\'util  al tomar los datos desde un repositorio y al momento de realizar pruebas de rendimiento de los
algoritmos, ya que su velocidad de transferencia es de 150 MB/sg; adem\'as la RAM que se utilizo fu\'e
superior a los 512 MB, ya que la Miner\'ia de Datos requiere grandes cantidades de memoria por el tama\~no de
los conjuntos de datos.\\
\\
El sistema operativo sobre el cual se trabajo durante la implementaci\'on de TariyKDD es Fedora Core en su
versiones 3 y 5. El lenguaje de programaci\'on en el que esta elaborado TariyKDD es Java 5.0, 
actualizaci\'on 06.\\
\\
Dentro del proceso de Descubrimiento de Conocimiento, TariyKDD comprende las etapas de Selecci\'on,
Preprocesamiento, Miner\'ia de Datos y Visualizaci\'on de Resultados. De esta forma la implementaci\'on de la
herramienta se hizo a trav\'es de los siguientes m\'odulos de software cuya estructura se muestra en la figura 7.81.

\begin{figure}[ht]
\centering
\includegraphics[width=0.5\textwidth]{images/arquitectura.png}
\caption{\'Arquitectura TariyKDD}
\end{figure}

\subsubsection{M\'odulo de Conexi\'on}
El M\'odulo de Conexi\'on permite al usuario acceder a los conjuntos de datos a trav\'es de un Archivo Plano o
una Base de Datos.\\
\\
La opci\'on Archivo Plano le permite al usuario seleccionar un conjunto de datos que se encuentra en disco, en un
archivo de acceso aleatorio, el formato para el archivo debe ser ARFF\cite{arff}, debido a que este es uno de los
m\'as conocidos y tiene una estructura que lo hace f\'acil de comprender por ser estandar (debido a su estructura
de etiquetas).\\
\\
En cuanto a la Conexi\'on a Bases de Datos, TariyKDD puede conectarse con PostgreSQL a trav\'es de su manejador
JDBC tipo 4 \cite{abcjdbc}. Este driver es el m\'as eficiente ya que traduce de forma directa las peticiones del
API Java al protocolo nativo del Sistema Gestor, con la ventaja de que resulta sencilla la migraci\'on a otro
diferente, lo \'unico que habr\'ia que hacer ser\'ia descargar el driver del fabricante adecuado.

\subsubsection{Almacenamiento de datos en memoria}
Una vez se ha hecho la conexi\'on al conjunto de datos, ya sea a trav\'es de Archivo Plano o mediante Bases de
Datos, el siguiente paso que se hace en TariyKDD es almacenar los datos en memoria principal, en una estructura
especial que admi\-nistra de manera optima el tama\~no de los datos. La cual se describe en este m\'odulo.\\
\\
Una de las principales dificultades dentro del proceso de descubrimiento de conoci\-miento es el uso adecuado de
los recursos del sistema y en especial de la memoria principal si tenemos en cuenta que se pretende trabajar con
amplios vol\'umenes de datos.  Ya sea cargando un conjunto de datos desde un archivo plano o directamente desde
una conexi\'on a un SGBD se espera organizar estos datos de una manera compacta con el objetivo de almacenar esta
informaci\'on en memoria principal evitando repetidas llamadas a disco lo que significa un aumento en los tiempos
de ejecuci\'on de la herramienta.\\
\\
Formatos tradicionales para el almacenamiento de transacciones, como el formato ARFF, trabajan con cabeceras
donde se registran los diferentes campos o atributos del conjunto de datos seguidos de las transacciones como tal,
separadas una de otra por cambios de l\'inea donde cada atributo esta separado a su vez por comas. En conjuntos de
datos discretizados cuyos atributos pueden tomar un rango determinado de valores, dentro de la parte en donde se
almacenan los datos es com\'un encontrar segmentos de transacciones que coinciden o incluso transacciones
completas que se repiten.\\
\\
Es posible aprovechar estas coincidencias dentro de una estructura de datos como un \'arbol N-Ario donde cada
rama represente una posible transacci\'on y donde las bifurcaciones dentro de esa rama representen segmentos
compartidos con otras transacciones o inclusive transacciones que est\'en contenidas dentro de esa misma rama.\\
\\
Para explicar de mejor manera esta propuesta consideremos el conjunto de datos representado en la siguiente
tabla, hay que tener en cuenta que los items del conjunto de datos original son codificados para mejorar la
administraci\'on de la memoria.\\

\begin{center}
\begin{tabular}{|c|c|c|c|c|c|c|} \hline
\textbf{T} & \textbf{A} & \textbf{B} & \textbf{C} & \textbf{D} & \textbf{E} \\ \hline
1 & 1 & 1 & 3 & 4 & 5 \\ \hline
2 & 1 & 1 & 3 & 4 & 6 \\ \hline
3 & 1 & 2 & 3 & 6 & 7 \\ \hline
4 & 2 & 3 & 4 & 1 & 3 \\ \hline
5 & 1 & 2 & 3 & 6 & 7 \\ \hline
6 & 2 & 3 & 4 & 6 & 5 \\ \hline
\end{tabular}
\end{center}

Se puede ver que los cuatro primeros campos de las transacciones 1 y 2 son iguales por lo que pueden compartir
nodos dentro del \'arbol N-Ario.\\

\begin{figure}[ht]
\centering
\includegraphics[width=0.5\textwidth]{images/nario2.png}
\caption{\'Arbol N-Ario}
\label{nario}
\end{figure}

Como se puede ver en la figura \ref{nario}, los valores entre par\'entesis en las hojas del \'arbol representan 
el n\'umero de repeticiones de la transacci\'on, por ejemplo la n\'umero 3 es la misma transacci\'on 5, por tanto
en el \'arbol N-Ario estos dos registros ser\'an almacenados en una sola rama, con la precauci\'on de contar su
soporte.\\
\\
Entre menos n\'umero de valores tenga un atributo y entre m\'as transacciones formen el conjunto de datos, existe
mayor posibilidad de encontrar coincidencias y aprovechar segmentos de transacciones ya almacenadas para guardar
las nuevas que se repitan.\\
\\
Un an\'alisis del presente formato para compresi\'on de datos con respecto al formato ARFF se muestra en los
anexos, en el cuadro \ref{formatos}, donde se registra el tama\~no en disco de cada formato al almacenar
conjuntos de datos con diferente n\'umero de transacciones y atributos.

\subsubsection{M\'odulo Filtros}
El m\'odulo filtros o data cleaning, se encarga de hacer un refinamiento de los datos en  dos etapas, por un lado
hace un proceso de limpieza sobre datos corruptos, vacios, ruidosos, inconsistentes, duplicados, alterados etc,
por otro lado hace una seleccion de estos datos para escojer aquellos que brinden informacion de calidad,
aplicando muestreos, discretizaciones, etc. De esta forma obtenemos datos depurados seg\'un el objetivo del
analista, para que posteriormente se pueda aplicar el nucleo KDD o de Miner\'ia de Datos sobre datos coherentes,
limpios y consistentes.\\
\\
A este m\'odulo, pertenecen los filtros de Remove Missing, Update Missing, Selection, Range, Reduction,
Codification, Replace Value, Numeric Range y Discretize, los cuales se alimentan y presentan sus resultados a
trav\'es de un TableModel que es el medio por el cual comunican sus flujos de datos.

\subsubsection{M\'odulo algoritmos}
Dentro de este m\'odulo se encuentran dos tipos de algoritmos, los de asociaci\'on y los de clasificaci\'on:

\begin{enumerate}
\item Asociaci\'on: Los algoritmos de asociaci\'on implementados en TariyKDD son EquipAsso, FPGrowth y Apriori.
Los tres algoritmos utilizan un vector de \'arboles AVL balanceados para almacenar los itemsets frecuentes. Cada
posici\'on del vector almacena un tipo de itemsets frecuentes en un \'arbol AVL. De esta forma la posici\'on 0 del
vector almacena un \'arbol AVL que contiene los itemsets frecuentes tipo 1. En este m\'odulo de TariyKDD, la mayor
ventaja que proporcionan los \'arboles AVL es la rapidez con la que se realizan las busquedas a la hora de 
determinar si un itemset frecuente ya existe.\\
\\
As\'i mismo los tres algoritmos usan el \'arbol N-Ario que se describio anteriormente para almacenar el conjunto
de datos que se vaya a minar. Los datos comprimidos en esta estructura son tomados como entradas por los tres
algoritmos, a partir de estos se aplican las t\'ecnicas de Miner\'ia de Datos correspondientes y el producto final
son las reglas.

\item Clasificaci\'on: Dentro de este m\'odulo en TariyKDD se implementaron los algoritmos Mate y C4.5. Para los
algoritmos de clasificaci\'on igual que en los de asociaci\'on la estructura que se utilizo para su 
implementaci\'on fu\'e un \'arbol N-Ario. En C4.5 y Mate para almacenar los datos en primera instancia. Ya que ha
medida que se desarrollan los algoritmos este \'arbol va cambiando, as\'i, al final el \'arbol tiene las reglas
de clasificaci\'on.\\
\\
La estructura utilizada para mostrar las reglas de clasificaci\'on es un \'arbol N-Ario que implementa los
m\'etodos de la interfaz de Java, TreeModel , la cual define un modelo de datos adecuado para poder visualizarlos
en un JTree o un control de Java que despliega las reglas jerarquicamente de acuerdo a la estructura en la que
estan almacenadas, es decir el \'arbol N-Ario.
\end{enumerate}

\subsubsection{M\'odulo Visores}
Este m\'odulo implementa las clases necesarias para mostrar de forma gr\'afica las reglas que se obtienen 
despu\'es de haber realizado Miner\'ia de Datos.\\
\\
Si el usuario ha utilizado algoritmos de asociaci\'on obtiene como resultado reglas, que puede observar a
trav\'es de una JTable, la cual es utilizada para desplegar y editar tablas de dos dimensiones. Para desplegar
estas reglas, primero, se tiene un array con los datos que se van a mostrar, a partir de estos se construye un
m\'odelo propio de tabla implementando los m\'etodos de la interfaz Java, TableModel. Des\-pu\'es simplemente el 
m\'odelo se pasa al constructor de JTable para que esta clase se encarge de desplegar las reglas.\\
\\
Pero si durante el proceso de Miner\'ia de Datos se utilizaron algoritmos de clasificaci\'on, las reglas son 
desplegadas en un \'arbol N-Ario que dentro de TariyKDD se ha llamado \'Arbol de Resultados. La construcci\'on de
este se la realiza como se dijo anteriormente implementando los m\'etodos de la interfaz TreeModel, modelo que 
luego es usado por un JTree para visualizar las reglas de manera jer\'arquica.

\subsubsection{M\'odulo GUI}

El m\'odulo de GUI es usado en los 4 m\'odulos anteriores y es el encargado de brindrar una interfaz gr\'afica amigable al usuario.  Para su desarrollo se utilizaron las funcionalidades del proyecto Matisse, proyecto encargado de proveer facilidades para la construccion de entornos gr\'aficos a los proyectos desarrollados con NetBeans y que da un soporte a las aplicaciones que usan Swing, el cual es un conjunto de clases y componentes usados en interfaces gr\'aficas desde botones hasta tablas y estructuras de \'arboles.
%%%%%%%%%%%%%%%%%%%%%%%%%%%%%%%%%%%%%%%%%%% DESCRIPCION DE CLASES %%%%%%%%%%%%%%%%%%%%%%%%%%%%%%%%%%%%%%%%%%%%%%%%
\subsection{Descripci\'on de clases}

\subsubsection{Paquete Utils}
\begin{description}
\item [Clase DataSet] En esta estructura los algoritmos de asociaci\'on almacenan los datos o items de 
forma comprimida, ocupando menos espacio en memoria. La estructura utilizada por DataSet es un \'arbol
N-Ario que almacena los datos en cada nodo como tipo short. Lo particular de esta estructura es el 
aprovechamiento de la memoria principal, ya que en una sola rama almacena items de diferentes 
transacciones, controlando individualmente su n\'umero de apariciones.
\item [Clase FileManager] Esta clase gestiona todo lo relacionado con flujos de archivos, como por 
ejemplo crear un archivo plano, construir el diccionario de datos a partir de un archivo de acceso
aleatorio, entre otras funciones.
\item [Clase BaseDatos] Esta clase gestiona todo lo relacionado con el manejo de las Bases de Datos, 
como la conexi\'on, y la selecci\'on, de atributos.
\end{description}

\begin{description}
\item [Clase NodeNoF] Esta clase representa un nodo b\'asico del DataSet, este nodo no tiene soporte.
\item [Clase NodeF] Esta clase extiende a la clase NodeF y agrega el soporte a cada nodo del DataSet.
\item [Clase AvlTree] Los itemsets frecuentes generados por los algoritmos de asociaci\'on son
almacenados en un \'arbol AVL balanceado, cuya estructura se encuentra en esta clase.
\item [Clase AvlNode] Es en si, un nodo del \'arbol AVL que almacena los itemsets frecuentes. Tiene un
campo de tipo ItemSet en donde se guarda el dato que va en el nodo y tiene los punteros derecho e 
izquierdo a los dem\'as nodos del \'arbol.
\item [Clase ItemSet] La clase ItemSet almacena un conjunto de items o itemsets en un vector as\'i como
su respectivo soporte.
\item [Clase Transaction] Esta clase gestiona todas las operaciones que deben hacerse sobre las
transacciones. Como por ejemplo cargar las transacciones para los diferentes algoritmos y as\'i como 
tambi\'en realiza los diferentes ordenamientos de las transacciones, por item y por soporte.
\end{description}

\subsubsection{Paquete Apriori}
\begin{description}
\item [Clase Apriori] Esta clase implementa todos los m\'etodos necesarios para ejecutar el algoritmo Apriori. Los
parametros necesarios para comenzar el algoritmo son: un soporte de tipo short y un dataset (estructura de tipo
\'arbol N-Ario en la cual los datos son comprimidos) y sobre el cual se realizan tantos recorridos como itemsets
frecuentes existan.
\end{description}

\subsubsection{Paquete EquipAsso}
\begin{description}
\item [Clase EquipAsso] Para ejecutar el algoritmo EquipAsso los parametros necesa\-rios son: un soporte de tipo
short y un dataset (estructura de tipo \'arbol N-Ario en la cual los datos son comprimidos). Basicamente para
obtener los itemsets frecuentes, lo primero que se debe hacer es recorrer el \'arbol N-Ario tomar cada una de
sus transacciones, realizar todas sus combinaciones y ver cual de ellas pasa soporte y clasifica como itemset
frecuente.
\item [Clase Combinations] Recibe como parametros el tipo y el itemset a combinar. El tipo es un n\'umero que
indica hasta que profundidad se desea combinar el itemset en cuestion.
\end{description}

\subsubsection{Paquete FPGrowth}
\begin{description}
\item [Clase FPGrowth] El algoritmo FPGrowth tiene su propio \'arbol N-Ario para almacenar los datos. Recorre el
\'arbol y toma cada una de sus ramas, a partir de estas construye los Patrones Condicionales Base, luego los
Patrones Condicionales y a partir de estos determina cuales son los itemsets frecuentes, los cuales se almacenan en
un \'arbol AVL balanceado.
\item [Clase FPGrowthNode] Clase que tiene la estructura del \'arbol N-Ario del algoritmo FPGrowth. Es decir tiene
los punteros necesarios para armar un \'arbol N-Ario, tiene un puntero al hijo, al padre y al hermano.
\item [Clase BaseConditional] Clase que almacena los Patrones Condicionales Base a partir del \'arbol N-Ario de la 
clase FPGrowth.
\item [Clase BaseConditonals] Solo los nodos que pasan el soporte min\'imo se consi\-deran frecuentes, estos, tienen
un puntero a cada uno de sus Patrones Condicionales Base, a partir de los cuales se obtienen los itemsets
frecuentes.
\end{description}

\subsubsection{Paquete MateBy}
\begin{description}
\item [Clase MateBy] As\'i como los dem\'as algoritmos, MateBy utiliza el dataset o estructura de tipo
\'arbol N-Ario para comprimir los datos que se van a minar. A partir de estos datos se realizan
combinaciones y se calcula su entrop\'ia y su ganancia. Las combinaciones con la mayor ganancia se
almacenan en un \'arbol de reglas, conformando as\'i los resultados de MateBy.
\item [Clase Entro] Agrupa los nodos del \'arbol de acuerdo a su padre o rama y determina quienes tienen
la mayor ganancia, de acuerdo a esto se construye el \'arbol de reglas.
\end{description}

%%%%%%%%%%%%%%%%%%%%%%%%%%%%CASOS DE USO REALES%%%%%%%%%%%%%%%%%%%%%%%%%%%%%%%%%%%%%%%%%%%%%%%%
\newpage
\subsection{Casos de uso reales}
\subsubsection{Ingreso a la aplicaci\'on}
\begin{figure}[ht]
\centering
\includegraphics[width=1\textwidth]{images/01.png}
\caption{Ingreso a la aplicaci\'on}
\end{figure}
% TABLA 1
\begin{center}
\begin{tabular}{|p{60mm}|p{60mm}|} \hline
ACCI\'ON DEL ACTOR & RESPUESTA DEL SISTEMA \\ \hline
1. El usuario ejecuta la aplicaci\'on & 2. La interfaz gr\'aica de la aplicaci\'on aparece como se muestra en la figura. A: \'Area de pesta\~nas a trav\'es de las cuales se puede acceder a los difernetes m\'odulos de la aplicaci\'on. Ej, el m\'oulo por defecto es `Connections`. B: \'Area en la que aparecen las opciones de cada m\'odulo. Ej, las opciones del m\'odulo 'Connections' son 'Archivo Plano' y 'Conexi\'on DB'.   C: \'Area de trabajo sobre la que se arman los proyectos de Miner\'ia de Datos\\ \hline
\end{tabular}
\end{center}
\newpage
\subsubsection{M\'odulo filtros}
\begin{figure}[ht]
\centering
\includegraphics[width=1\textwidth]{images/02.png}
\caption{M\'odulo filtros}
\end{figure}
% TABLA 2
\begin{center}
\begin{tabular}{|p{60mm}|p{60mm}|} \hline
ACCI\'ON DEL ACTOR & RESPUESTA DEL SISTEMA \\ \hline
1. El usuario hace click en la pesta\~na A 'Filtros' & 2. Aparecen B Las opciones del m\'odulo 'Filtros'\\ \hline
\end{tabular}
\end{center}

\newpage
\subsubsection{M\'odulo algoritmos}
\begin{figure}[ht]
\centering
\includegraphics[width=1\textwidth]{images/03.png}
\caption{M\'odulo algoritmos}
\end{figure}
% TABLA 3
\begin{center}
\begin{tabular}{|p{60mm}|p{60mm}|} \hline
ACCI\'ON DEL ACTOR & RESPUESTA DEL SISTEMA \\ \hline
1. El usuario hace click en la pesta\~na A 'Algoritmos' & 2. Aparecen B Las opciones del m\'odulo 'Algoritmos' \\ \hline
\end{tabular}
\end{center}

\newpage
\subsubsection{M\'odulo visualizaci\'on}
\begin{figure}[ht]
\centering
\includegraphics[width=1\textwidth]{images/04.png}
\caption{M\'odulo visualizaci\'on}
\end{figure}
% TABLA 4
\begin{center}
\begin{tabular}{|p{60mm}|p{60mm}|} \hline
ACCI� DEL ACTOR & RESPUESTA DEL SISTEMA \\ \hline
1. El usuario hace click en la pesta\~na A 'Visualizaci\'on' & 2. Aparecen B Las opciones del m\'odulo 'Visualizaci\'on' \\ \hline
\end{tabular}
\end{center}

\newpage
\subsubsection{Conexi\'on a un archivo plano}
\begin{figure}[ht]
\centering
\includegraphics[width=1\textwidth]{images/05.png}
\caption{Conexi\'on a un archivo plano}
\end{figure}
% TABLA 5
\begin{center}
\begin{tabular}{|p{60mm}|p{60mm}|} \hline
ACCI\'ON DEL ACTOR & RESPUESTA DEL SISTEMA \\ \hline
1. El usuario hace click sobre el \'icono 'Archivo de Texto' & 2. El \'icono 'Archivo de Texto' aparece sobre A: \'area de trabajo.\\ \hline
\end{tabular}
\end{center}

\newpage
\subsubsection{Conexi\'on a una base de datos}
\begin{figure}[ht]
\centering
\includegraphics[width=1\textwidth]{images/06.png}
\caption{Conexi\'on a una base de datos}
\end{figure}
% TABLA 6
\begin{center}
\begin{tabular}{|p{60mm}|p{60mm}|} \hline
ACCI\'ON DEL ACTOR & RESPUESTA DEL SISTEMA \\ \hline
1. El usuario hace click sobre el \'icono 'Conexi\'on BD' & 2. El \'icono 'Conexi\'on BD' aparece sobre A: \'area de trabajo.\\ \hline
\end{tabular}
\end{center}

\newpage
\subsubsection{Men\'u emergente conexi\'on BD}
\begin{figure}[ht]
\centering
\includegraphics[width=1\textwidth]{images/07.png}
\caption{Men\'u emergente conexi\'on BD}
\end{figure}
% TABLA 7
\begin{center}
\begin{tabular}{|p{60mm}|p{60mm}|} \hline
ACCI� DEL ACTOR & RESPUESTA DEL SISTEMA \\ \hline
1. El usuario hace click derecho sobre el \'icono 'Conexi\'on BD' & 2. Se depliega A: menu del \'icono 'Conexi\'on BD'. Las opciones son: 'Delete': usada para eliminar el \'icono del \'area de trabajo. 'Configure': usada para configurar la conexi\'on a una base de datos. 'Selecci\'on de atributos': usada para seleccionar de forma gr\'afica los datos que ser\'ia usados m\'as adelante. 'Cargar': ejecuta el query que se gener\'a en la selecci\'on de atributos\\ \hline
\end{tabular}
\end{center}

\newpage
\subsubsection{Configuraci\'on conexi\'on BD}
\begin{figure}[ht]
\centering
\includegraphics[width=1\textwidth]{images/08.png}
\caption{Configuraci\'on conexi\'on BD}
\end{figure}
% TABLA 8
\begin{center}
\begin{tabular}{|p{60mm}|p{60mm}|} \hline
ACCI\'ON DEL ACTOR & RESPUESTA DEL SISTEMA \\ \hline
1. El usuario hace click derecho sobre el \'icono 'Conexi\'on BD' y selecciona la opci\'on 'Configure'  & 2. Emerge una ventana de configuraci\'on de conexi\'on a bases de datos\\ \hline
\end{tabular}
\end{center}


\newpage
\subsubsection{Ventana de conexi\'on BD}
\begin{figure}[ht]
\centering
\includegraphics[width=1\textwidth]{images/11.png}
\caption{Ventana de conexi\'on BD}
\end{figure}
% TABLA 8
\begin{center}
\begin{tabular}{|p{60mm}|p{60mm}|} \hline
ACCI\'ON DEL ACTOR & RESPUESTA DEL SISTEMA \\ \hline
1. El usuario desea configurar la conexi\'on a una base de datos'  & 2. Las opciones de la ventana de configuraci\'on de conexi\'on a bases de datos tiene los siguientes campos: A: Lista de controladores ODBC para varios tipos de bases de datos. B: Nombre del usuario de la base de datos. C: Nombre de la base de datos. D: Nombre del servidor. E: 'Password': clave de acceso a la base de datos. F: nmero del puerto utilizado para la comunicaci\'on con la base de datos. G: bot\'on de conexi\'on. H: bot\'on para aceptar la conexi\'on hecha. I: mensaje que indica el estado de la conexi\'on. \\ \hline
\end{tabular}
\end{center}

\newpage
\subsubsection{Selecci\'on de atributos}
\begin{figure}[ht]
\centering
\includegraphics[width=1\textwidth]{images/09.png}
\caption{Selecci\'on de atributos}
\end{figure}
% TABLA 8
\begin{center}
\begin{tabular}{|p{60mm}|p{60mm}|} \hline
ACCI\'ON DEL ACTOR & RESPUESTA DEL SISTEMA \\ \hline
1. El usuario hace click derecho sobre el \'icono 'Conexi\'on BD' para hacer la selecci\'on de atributos  & 2. Aparece el men\'u emergente del \'icono.y se ejecuta la ventana de selecci\'on de atributos A. \\ \hline 3. El usuario hace click en la opci\'on B: 'Selecci\'on de Atributos' & 4. Aparece la ventana de selecci\'on de atributos B.  \\ \hline
\end{tabular}
\end{center}

\newpage
\subsubsection{Ventana selecci\'on de atributos}
\begin{figure}[ht]
\centering
\includegraphics[width=1\textwidth]{images/attverMark.png}
\caption{Ventana selecci\'on de atributos}
\end{figure}
% TABLA 8
\begin{center}
\begin{tabular}{|p{60mm}|p{60mm}|} \hline
ACCI\'ON DEL ACTOR & RESPUESTA DEL SISTEMA \\ \hline
1. El usuario desea hacer la selecci\'on de atributos  & 2. Aparece la ventana de selecci\'on de atributos. A: lista desplegable de las tablas de la base de datos a la que se ha conectado. Al seleccionar una de ellas su representaci\'on gr\'afica aparecera en el espacio de trabajo E. B: opci\'on que permite ver las relaciones establecidas a trav\'es de la l\'inea de conexi\'on de atributos entre las tablas. C: esta opci\'on es \'util cuando se trabajan problemas de canasta de mercado. D: opci\'on para trabajar tablas multivaluadas. F:l\'inea que permite realizar las relaciones entre atributos de dos tablas. El resultado de la relaci\'on establecida se refleja en el query. G:Si se hace click sobre uno de los atributos aparece un \'icono de verificaci\'on que indica los campos que ser\'an mostrados al ejecutar el query. H: espacio en el que se crea el query. Es posible editarlo manualmente. I: bot\'on de ejecuci\'on del query. J: tabla en la que se muestra el resultado de la ejecuci\'on del query. K: bot\'on para aceptar las operaciones realizadas.   \\ \hline
\end{tabular}
\end{center}

\newpage
\subsubsection{Filtro Remove Missing}
\begin{figure}[ht]
\centering
\includegraphics[width=1\textwidth]{images/17.png}
\caption{Filtro Remove Missing}
\end{figure}
% TABLA 12
\begin{center}
\begin{tabular}{|p{60mm}|p{60mm}|} \hline
ACCI\'ON DEL ACTOR & RESPUESTA DEL SISTEMA \\ \hline
1. El usuario hace click sobre uno de los \'iconos del m�ulo A: 'Filtros'.   & 2. En el \'area de opciones del m\'odulo aparecen los 9 \'iconos correspondientes a los filtros\\ \hline
3. El usuario hace click sobre uno de los \'iconos correspondientes a los filtros. & 4. El \'icono correspondiente aparace en el \'area de trabajo B.\\ \hline
\end{tabular}
\end{center}

\newpage
\subsubsection{Conexi\'on filtros a BD}
\begin{figure}[ht]
\centering
\includegraphics[width=1\textwidth]{images/19.png}
\caption{Conexi\'on filtros a BD}
\end{figure}
% TABLA 12
\begin{center}
\begin{tabular}{|p{60mm}|p{60mm}|} \hline
ACCI\'ON DEL ACTOR & RESPUESTA DEL SISTEMA \\ \hline
1. El usuario conecta una base de datos a alguno o varios de de los filtros A. & 2. Los \'iconos pueden ser conectados por medio de una l\'inea B.\\ \hline
\end{tabular}
\end{center}

\newpage
\subsubsection{Men\'u emergente de filtros}
\begin{figure}[ht]
\centering
\includegraphics[width=1\textwidth]{images/f1.png}
\caption{Men\'u emergente de filtros}
\end{figure}
% TABLA 13
\begin{center}
\begin{tabular}{|p{60mm}|p{60mm}|} \hline
ACCI\'ON DEL ACTOR & RESPUESTA DEL SISTEMA \\ \hline
1. El usuario hace click sobre el \'icono 'Remove Missing'. & 2. El \'icono aparece en el \'area de trabajo A.\\ \hline
3. El usuario conecta el filtro a la base de datos. & 4. Aparece un hilo que conecta los \'iconos B.\\ \hline
5. EL usuario hace click derecho sobre el filtro. & 6. Aparece el menu emergente del \'icono C. La opci\'on Delete, borra el filtro del \'area de trabajo. Este filtro no tiene ventana de configuraci\'on. La opci\'on 'Run' ejecuta la aplicaci\'on del filtro. La opci\'on 'View' muestra la ventana de vizualizaci\'on de datos que ser\'ia descrita en el siguiente caso de uso\\ \hline
\end{tabular}
\end{center}

\newpage
\subsubsection{Visualizaci\'on de datos filtrados}
\begin{figure}[ht]
\centering
\includegraphics[width=0.8\textwidth]{images/fv1.png}
\caption{Visualizaci\'on de datos filtrados}
\end{figure}
\begin{figure}[ht]
\centering
\includegraphics[width=0.7\textwidth]{images/fv2.png}
\caption{caso nueve}
\end{figure}

\newpage
% TABLA 12
\begin{center}
\begin{tabular}{|p{60mm}|p{60mm}|} \hline
ACCI\'ON DEL ACTOR & RESPUESTA DEL SISTEMA \\ \hline
1. El usuario hace click sobre la opci\'on 'View' del menu desplegable filtro en el \'area de trabajo . & 2. Aparece la ventana de vizualizaci\'on de datos filtrados y no filtrados. Los campos son, A: Variables o nombres de los campos de la tabla. B: Datos de entrada que son los datos que llegaron al filtro inicialmente.  C: Datos filtrados que son el resultado de haber aplicado el filtro. D: nmero de registros eliminados al aplicar el filtro. E: Nmero de registros despu\'es de aplicar el filtro. En la figura 16 se ve la grilla sobre la que se muetran los datos en el caso 'Datos de entrada'\\ \hline
\end{tabular}
\end{center}

\newpage
\subsubsection{Configuraci\'on filtro Update Missing}
\begin{figure}[ht]
\centering
\includegraphics[width=1\textwidth]{images/fi2.png}
\caption{Configuraci\'on filtro Update Missing}
\end{figure}
% TABLA 13
\begin{center}
\begin{tabular}{|p{60mm}|p{60mm}|} \hline
ACCI\'ON DEL ACTOR & RESPUESTA DEL SISTEMA \\ \hline
1. El usuario hace click derecho sobre el filtro A y elige la opci\'on 'Configuraci\'on'& 2. Se muestra B la ventana de configuraci\'on correspondiente al filtro 'Update Missing'. Los campos son: Atributo, en el cual se escribe el nombre del atributo a buscar en el conjunto de datos. Reemplazar con, aqui se escribe el nuevo valor del atributo \\ \hline
\end{tabular}
\end{center}

%------------------------hacer de aqui en adelante------------------------------------------------------------

\newpage
\subsubsection{Configuraci\'on filtro Selection}
\begin{figure}[ht]
\centering
\includegraphics[width=1\textwidth]{images/fi3ver.png}
\caption{Configuraci\'on filtro Selection}
\end{figure}
% TABLA 13
\begin{center}
\begin{tabular}{|p{60mm}|p{60mm}|} \hline
ACCI\'ON DEL ACTOR & RESPUESTA DEL SISTEMA \\ \hline
1. El usuario hace click derecho sobre el filtro A y elige la opci\'on 'Configuraci\'on'& 2. Se muestra B la ventana de configuraci\'on correspondiente al filtro 'Selection'. Los campos son: C: Atributo, en esta grilla se muestran los nombres de los atributos seleccionados. D: Tipo, muestra el tipo de datos de los atributos. E: cajas de verificaci\'on para escoger los atributos a utilizar. F: es posible escoger un atributo clase haciendo click sobre estos campos. Esto es \'util en experimentos de clasificaci\'on. G: el bot\'on 'Aplicar' debe ser precionado para que el filtro sea aplicado. \\ \hline
\end{tabular}
\end{center}
%------------------------hacer de aqui en adelante------------------------------------------------------------

\newpage
\subsubsection{Configuraci\'on filtro Range}
\begin{figure}[ht]
\centering
\includegraphics[width=1\textwidth]{images/fi4.png}
\caption{Configuraci\'on filtro Range}
\end{figure}
% TABLA 13
\begin{center}
\begin{tabular}{|p{60mm}|p{60mm}|} \hline
ACCI\'ON DEL ACTOR & RESPUESTA DEL SISTEMA \\ \hline
1. El usuario hace click derecho sobre el filtro A y elige la opci\'on 'Configuraci\'on'& 2. Se muestra B la ventana de configuraci\'on correspondiente al filtro 'Range'. Los campos son:\textbf{Aleatorio}, en donde se escribe el n\'umero \textbf{n} de filas que se desea sean escogidas aleatoriamente. \textbf{1 en n}, donde \textbf{n} es el periodo utilizado para seleccionar los datos a utilizar. \textbf{Primeros n}, donde \textbf{n}es el nmero campos a incluir en la selecci\'on a partir del primero. \\ \hline
\end{tabular}
\end{center}
%------------------------hacer de aqui en adelante------------------------------------------------------------u

\newpage
\subsubsection{Configuraci\'on filtro Reduction}
\begin{figure}[ht]
\centering
\includegraphics[width=1\textwidth]{images/fi5.png}
\caption{Configuraci\'on filtro Reduction}
\end{figure}
% TABLA 13
\begin{center}
\begin{tabular}{|p{60mm}|p{60mm}|} \hline
ACCI\'ON DEL ACTOR & RESPUESTA DEL SISTEMA \\ \hline
1. El usuario hace click derecho sobre el filtro A y elige la opci\'on 'Configuraci\'on'& 2. Se muestra la ventana B de configuraci\'on correspondiente al filtro 'Reduction'. Los campos son: \textbf{Por rango}, los campos son 'Fila ini cial' donde se escribe la fila a partir de la cual inicia el rango y 'Fila final' que es el l\'imite superior del rango. \textbf{Por Valor:} Se elige el nombre del atributo y luego en caso de que los valores a quitar sean num\'ericos en el campo 'Menores que' se especifica el nmero a partir del cual se hace la reducci\'on. Si el atributo es alfab\'etico se escribe su valor en el \'area de texto y en las casillas de selecci\'on se especifica si ese valor se desea 'Mantener' o 'Remover'. \textbf{Aplicar}: ejecuta el filtro.  \\ \hline
\end{tabular}
\end{center}
%------------------------hacer de aqui en adelante------------------------------------------------------------v

\newpage
\subsubsection{Configuraci\'on filtro Codification}
\begin{figure}[ht]
\centering
\includegraphics[width=1\textwidth]{images/fi6.png}
\caption{Configuraci\'on filtro Codification}
\end{figure}
% TABLA 13
\begin{center}
\begin{tabular}{|p{60mm}|p{60mm}|} \hline
ACCI\'ON DEL ACTOR & RESPUESTA DEL SISTEMA \\ \hline
1. El usuario hace click derecho sobre el filtro A y elige la opci\'on 'Configuraci\'on'& 2. Se muestra la ventana de configuraci\'on correspondiente al filtro 'Codifcation'. Este filtro no tiene ventana de configuraci\'on. Se aplica para asignar un n\'umero a valores alfab\'eticos\\ \hline
\end{tabular}
\end{center}
%------------------------hacer de aqui en adelante------------------------------------------------------------w

\newpage
\subsubsection{Configuraci\'on filtro Replace Value}
\begin{figure}[ht]
\centering
\includegraphics[width=1\textwidth]{images/fi7.png}
\caption{Configuraci\'on filtro Replace Value}
\end{figure}
% TABLA 13
\begin{center}
\begin{tabular}{|p{60mm}|p{60mm}|} \hline
ACCI\'ON DEL ACTOR & RESPUESTA DEL SISTEMA \\ \hline
1. El usuario hace click derecho sobre el filtro A y elige la opci\'on 'Configuraci\'on'& 2. Se muestra la ventana de configuraci\'on correspondiente al filtro 'Replace Value'. Los campos son: Atributo, en el cual se elige el nombre del atributo a buscar en el conjunto de datos. Reemplazar con, aqui se escribe el nuevo valor del atributo. \textbf{Aplicar}: ejecuta el filtro.  \\ \hline
\end{tabular}
\end{center}
%------------------------hacer de aqui en adelante------------------------------------------------------------

\newpage
\subsubsection{Configuraci\'on filtro Numeric Range}
\begin{figure}[ht]
\centering
\includegraphics[width=1\textwidth]{images/fi8.png}
\caption{Configuraci\'on filtro Numeric Range}
\end{figure}
% TABLA 13
\begin{center}
\begin{tabular}{|p{60mm}|p{60mm}|} \hline

ACCI\'ON DEL ACTOR & RESPUESTA DEL SISTEMA \\ \hline
1. El usuario hace click derecho sobre el filtro A y elige la opci\'on 'Configuraci\'on'& 2. Se muestra la ventana B de configuraci\'on correspondiente al filtro 'Numeric Range'. Los campos son: \textbf{Atributo}, en el cual se escribe el nombre del atributo a discretizar de tipo num\'erico. \textbf{Reemplazar rango con valores nulos}: aqui es posible especificar un rango de datos que seran convertidos a n\'ulos. \textbf{M\'imimo valor}: l\'imite inferiror del rango. \textbf{M\'inimo valor}: l\'imite superior del rango.  \textbf{Aplicar}: ejecuta el filtro. \textbf{Resetear}: deja los campos en blanco  \\ \hline
\end{tabular}
\end{center}
%------------------------hacer de aqui en adelante------------------------------------------------------------

\newpage
\subsubsection{Configuraci\'on filtro Discretize}
\begin{figure}[ht]
\centering
\includegraphics[width=1\textwidth]{images/fi9.png}
\caption{Configuraci\'on filtro Discretize}
\end{figure}
% TABLA 13
\begin{center}
\begin{tabular}{|p{60mm}|p{60mm}|} \hline
ACCI\'ON DEL ACTOR & RESPUESTA DEL SISTEMA \\ \hline
1. El usuario hace click derecho sobre el filtro A y elige la opci\'on 'Configuraci\'on'& 2. Se muestra la ventana B de configuraci\'on correspondiente al filtro 'Discretize'. Los campos son: \textbf{Atributo}, en el cual se escribe el nombre del atributo a discretizar. \textbf{Discretizar por}: 'N\'umero de rango': se puede establecer el n\'umero de rangos a crear. 'Tama\~no del rango': se especifica el tama\~no del rango \textbf{Aplicar}: ejecuta el filtro. \textbf{Resetear}: deja los campos en blanco  \\ \hline
\end{tabular}
\end{center}

\newpage
\subsubsection{Algoritmos}
\begin{figure}[h]
 \centering
 \includegraphics[width=1\textwidth]{images/a1.png}
 \caption{Algoritmo Apriori}
\end{figure}

\begin{center}
\begin{tabular}{|p{60mm}|p{60mm}|}\hline
ACCI\'ON DEL ACTOR & RESPUESTA DEL SISTEMA \\ \hline
1. Si el usuario quiere minar los datos con Apriori y presiona sobre A (\'Area de opciones), en el icono respectivo.
& 2. En B (\'Area de trabajo) aparece el icono del algoritmo Apriori. \\ \hline
\end{tabular}
\end{center}
\newpage

\subsubsection{Opci\'on Delete}
\begin{figure}[h]
 \centering
 \includegraphics[width=1\textwidth]{images/am1.png}
 \caption{Opci\'on Delete}
\end{figure}

\begin{center}
\begin{tabular}{|p{60mm}|p{60mm}|}\hline
ACCI\'ON DEL ACTOR & RESPUESTA DEL SISTEMA \\ \hline
1. El usuario hace click derecho sobre A: el icono del algoritmo (cualquiera que este sea, Apriori, EquipAsso,
FPGrowth, MateBy o C4.5) y elige la opci\'on delete del men\'u de configuraci\'on.
& 2. El icono del algoritmo es borrado del \'area de trabajo. \\ \hline
\end{tabular}
\end{center}
\newpage

\subsubsection{Opci\'on Configure}
\begin{figure}[h]
 \centering
 \includegraphics[width=0.9\textwidth]{images/am2.png}
 \caption{Opci\'on Configure}
\end{figure}

\begin{center}
\begin{tabular}{|p{60mm}|p{60mm}|}\hline
ACCI\'ON DEL ACTOR & RESPUESTA DEL SISTEMA \\ \hline
1. El usuario hace click derecho sobre A: el icono del algoritmo (cualquiera que este sea, Apriori, EquipAsso,
FPGrowth, MateBy o C4.5) y elige configurar sus parametros.
& 2. Sobre el \'area de trabajo aparece una ventana B, para que el usuario configure el soporte del algoritmo. \\
\hline
\end{tabular}
\end{center}
\newpage

\subsubsection{Opci\'on Run}
\begin{figure}[h]
 \centering
 \includegraphics[width=0.9\textwidth]{images/am3.png}
 \caption{Opci\'on Run}
\end{figure}

\begin{center}
\begin{tabular}{|p{60mm}|p{60mm}|}\hline
ACCI\'ON DEL ACTOR & RESPUESTA DEL SISTEMA \\ \hline
1. El usuario hace click derecho sobre A: el icono del algoritmo (cualquiera que este sea, Apriori, EquipAsso,
FPGrowth, MateBy o C4.5) y elige la opci\'on run.
& 2. El icono del algoritmo cambia por una animaci\'on, as\'i como se muestra en B. \\
\hline
\end{tabular}
\end{center}
\newpage

\subsubsection{Algoritmo FPGrowth}
\begin{figure}[h]
 \centering
 \includegraphics[width=1\textwidth]{images/a2.png}
 \caption{Algoritmo FPGrowth}
\end{figure}

\begin{center}
\begin{tabular}{|p{60mm}|p{60mm}|}\hline
ACCI\'ON DEL ACTOR & RESPUESTA DEL SISTEMA \\ \hline
1. Si el usuario quiere minar los datos con FPGrowth y presiona sobre A (\'Area de opciones), en el icono respectivo.
& 2. En B (\'Area de trabajo) aparece el icono del algoritmo FPGrowth. \\ \hline
\end{tabular}
\end{center}
\newpage

\subsubsection{Algoritmo EquipAsso}
\begin{figure}[h]
 \centering
 \includegraphics[width=1\textwidth]{images/a3.png}
 \caption{Algoritmo EquipAsso}
\end{figure}

\begin{center}
\begin{tabular}{|p{60mm}|p{60mm}|}\hline
ACCI\'ON DEL ACTOR & RESPUESTA DEL SISTEMA \\ \hline
1. Si el usuario quiere minar los datos con EquipAsso y presiona sobre A (\'Area de opciones), en el icono
respectivo. & 2. En B (\'Area de trabajo) aparece el icono del algoritmo EquipAsso. \\ \hline
\end{tabular}
\end{center}
\newpage

\subsubsection{Algoritmo C4.5}
\begin{figure}[h]
 \centering
 \includegraphics[width=1\textwidth]{images/a4.png}
 \caption{Algoritmo C4.5}
\end{figure}

\begin{center}
\begin{tabular}{|p{60mm}|p{60mm}|}\hline
ACCI\'ON DEL ACTOR & RESPUESTA DEL SISTEMA \\ \hline
1. Si el usuario quiere minar los datos con C4.5 y presiona sobre A (\'Area de opciones), en el icono respectivo.
& 2. En B (\'Area de trabajo) aparece el icono del algoritmo C4.5. \\ \hline
\end{tabular}
\end{center}
\newpage

\subsubsection{Algoritmo Mate}
\begin{figure}[h]
 \centering
 \includegraphics[width=1\textwidth]{images/a5.png}
 \caption{Algoritmo Mate}
\end{figure}

\begin{center}
\begin{tabular}{|p{60mm}|p{60mm}|}\hline
ACCI\'ON DEL ACTOR & RESPUESTA DEL SISTEMA \\ \hline
1. Si el usuario quiere minar los datos con Mate y presiona sobre A (\'Area de opciones), en el icono respectivo.
& 2. En B (\'Area de trabajo) aparece el icono del algoritmo Mate. \\ \hline
\end{tabular}
\end{center}
\newpage

\subsubsection{Diagrama de Visualizaci\'on}
\begin{figure}[h]
 \centering
 \includegraphics[width=1\textwidth]{images/v01.png}
 \caption{Diagrama de Visualizaci\'on}
\end{figure}

\begin{center}
\begin{tabular}{|p{60mm}|p{60mm}|}\hline
ACCI\'ON DEL ACTOR & RESPUESTA DEL SISTEMA \\ \hline
1. Cuando el usuario ha construido una secuencia de Miner\'ia de Datos, con cualquiera de los algoritmos, en A se
encuentra en la secci\'on de vistas y en B (\'Area de opciones) ha hecho click en el icono generador.
& 2. Entonces en C (\'Area de trabajo) aparece el icono del generador, a trav\'es del cual el usuario puede acceder a las opciones de este m\'odulo. \\ \hline
\end{tabular}
\end{center}
\newpage

\subsubsection{Opci\'on Delete}
\begin{figure}[h]
 \centering
 \includegraphics[width=1\textwidth]{images/v0m1.png}
 \caption{Opci\'on Delete}
\end{figure}

\begin{center}
\begin{tabular}{|p{60mm}|p{60mm}|}\hline
ACCI\'ON DEL ACTOR & RESPUESTA DEL SISTEMA \\ \hline
1. El usuario hace click derecho sobre A: el icono generador y elige la opci\'on Delete.
& 2. El icono desaparece del \'area de trabajo, esperando un nuevo icono en la secuencia de Miner\'ia de Datos. \\
\hline
\end{tabular}
\end{center}
\newpage

\subsubsection{Opci\'on Configure}
\begin{figure}[h]
 \centering
 \includegraphics[width=1\textwidth]{images/v0m2.png}
 \caption{Opci\'on Configure}
\end{figure}

\begin{center}
\begin{tabular}{|p{60mm}|p{60mm}|}\hline
ACCI\'ON DEL ACTOR & RESPUESTA DEL SISTEMA \\ \hline
1. El usuario hace click sobre A: el icono del generador y elige configurar sus parametros.
& 2. Sobre el \'area de trabajo aparece una ventana B, para que el usuario configure la confianza con la cual se van a
filtrar las reglas de asociaci\'on. \\ \hline
\end{tabular}
\end{center}
\newpage

\subsubsection{Opci\'on Run}
\begin{figure}[h]
 \centering
 \includegraphics[width=1\textwidth]{images/v0m3.png}
 \caption{Opci\'on Run}
\end{figure}

\begin{center}
\begin{tabular}{|p{60mm}|p{60mm}|}\hline
ACCI\'ON DEL ACTOR & RESPUESTA DEL SISTEMA \\ \hline
1. El usuario hace click derecho sobre A: el icono generador y elige la opci\'on Run.
& 2. En el \'area de trabajo aparece una ventana B, con las reglas obtenidas a partir de los algoritmos de Miner\'ia de
Datos (La cual se explica en la siguiente figura). \\ \hline
\end{tabular}
\end{center}
\newpage

\subsubsection{Visor de Reglas}
\begin{figure}[h]
 \centering
 \includegraphics[width=1\textwidth]{images/v0r.png}
 \caption{Visor de Reglas}
\end{figure}

\begin{center}
\begin{tabular}{|p{60mm}|p{60mm}|}\hline
ACCI\'ON DEL ACTOR & RESPUESTA DEL SISTEMA \\ \hline
1. El usuario tiene la opci\'on de hacer click en A, o en B.
& 2. Al hacer click en A, la ventana de reglas desaparece y si hace click en B el usuario tiene la opci\'on de guardar
el reporte de las reglas de asociaci\'on. \\ \hline
\end{tabular}
\end{center}

\include{pruebas}
\chapter{CONCLUSIONES}
En este proyecto se dise\~n\'o e implement\'o una herramienta d\'ebilmente acoplada con el SGBD PostgreSQL que da soporte a las etapas de conexi\'on, preprocesamiento, miner\'ia y visualizaci\'on del proceso KDD.  Igualmente se incluyeron en el estudio nuevos algoritmos de asociaci\'on y clasificaci\'on propuestos por Timaran \cite{32}.\\
\\
Para el desarrollo del proyecto se hizo un an\'alisis de varias herramientas de software libre que abordan tareas similares a las que se pretend\'ia en este trabajo.  Se identific\'o las limitaciones y virtudes de estas aplicaciones y se dise\~n\'o una metodolog\'ia para el desarrollo de una herramienta que cubriera las falencias encontradas.\\
\\
Teniendo en cuenta la intenci\'on de liberar la herramienta se establecieron patrones de dise\~no que hicieran posible el acoplamiento de nuevas funcionalidades a cada uno de los m\'odulos que lo componen, facilitando as\'i la inclusi\'on futura de nuevas caracter\'isticas y el mejoramiento continuo de la aplicaci\'on.\\
\\
La construcci\'on de TariyKDD comprendi\'o el desarrollo de cuatro m\'odulos que cubrieron, el proceso de conexi\'on a datos, tanto a archivos planos como a bases de datos relacionales, la etapa de preprocesamiento, donde se implementaron 9 filtros para la selecci\'on, transformaci\'on y preparaci\'on de los datos, el proceso de miner\'ia, que comprendi\'o tareas de asociaci\'on y clasificaci\'on, implementando 5 algoritmos, Apriori, FPGrowth y EquipAsso para asociaci\'on y C4.5 y MateBy para clasificaci\'on y el proceso de visualizaci\'on de resultados, utilizando tablas y \'arboles para generar reportes de los resultados y reglas obtenidas.  Estos desarrollos fueron logrados usando en su totalidad herramientas de c\'odigo abierto y software libre.\\
\\
Se desarroll\'o un modelo de datos que facilit\'o la aplicaci\'on de algoritmos de asociaci\'on sobre bases de datos enmarcadas en el concepto de canasta de mercado donde la longitud de cada transacci\'on es variable.\\
\\
Se realizaron pruebas para evaluar la validez de los algoritmos implementados. Para el plan de pruebas de la tarea de asociaci\'on, se utilizaron conjuntos de datos
reales de transacciones de un supermercado de la Caja de Compensaci\'on familiar
de Nari\~no. Para Clasificaci\'on, se trabaj\'o 
%% trabajará %%
con conjuntos de datos especializados para este
tipo de algoritmos disponibles en \cite{data}.\\
\\
Analizando las pruebas obtenidas para Asociaci\'on, en esta arquitectura d\'ebilmente acoplada con PostgreSQL, el rendimiento es muy significativo obteniendo muy buenos tiempos de respuesta al aplicar el algoritmo EquipAsso. \\
\\
Fruto de este estudio se public\'o y sustent\'o un art\'iculo internacional en el marco del Congreso Latinoamericano de Estudios Informaticos - CLEI 2006 realizado en la ciudad de Santiago de Chile.\\
\\
Se cuenta con una versi\'on de TariyKDD con la capacidad de extraer reglas asociaci\'on y clasificaci\'on bajo una arquitectura d\'ebilmente acoplada con el SGBD PostgreSQL desarrollada bajo los lineamientos del software libre.\\
\\
Una vez que se han descrito los resultados m\'as relevantes que se han obtenido durante la realizacion de este proyecto, se sugiere una serie de recomendaciones como punto de partida para futuros trabajos:\\
\\
\begin{enumerate}
\item Realizar mayores pruebas de rendimiento de esta arquitectura e implementar otras primitivas que Timaran propone para tareas de Asociaci\'on y Clasificaci\'on.
\item Implementar otras tares y algoritmos de miner\'ia de datos, asi como nuevos filtros e interfaces de visualizaci\'on que permitan el mejoramiento continuo de TariyKDD.
\item Implementar nuevas interfaces gr\'aficas que permitan la visualizaci\'on de informaci\'on de una manera m\'as amigable para el usuario.
\item Probar el m\'odulo de clasificaci\'on con bases de datos reales.
\item Implementar una funcionalidad que permita aplicar el modelo de clasificaci\'on construido y clasificar datos cuya clase se desconoce.
\item Liberar y compartir una versi\'on de TariyKDD con la capacidad de descubrir conocimiento en bases de datos.
\end{enumerate}

Finalmente este trabajo nos permitio aplicar los conocimientos adquiridos en el programa de Ingenier\'ia de Sistemas y en especial los de la electiva de bases de datos, asi como nuestro trabajo y aprendizaje dentro del Grupo de Investigacion GRiAS L\'inea KDD.\\

\appendix
\chapter{ANEXOS}
\section{Rendimiento formato de comprensi\'on Tariy - Formato ARFF}

Un an\'alisis del formato para compresi\'on de datos descrito en el Cap\'itulo 7 del presente trabajo, en la
secci\'on Arquitectura Tariy, con respecto al formato ARFF de la herramienta de Miner\'ia de Datos WEKA se
muestra en el siguiente cuadro, donde se registra el tama\~no en disco de cada formato al almacenar conjuntos de
datos con diferente n\'umero de transacciones y atributos.

\begin{table}[h]
\caption{An\'alisis formatos de almacenamiento}
\label{formatos}
\end{table}
\begin{center}
\begin{tabular}{|p{30mm}|p{20mm}|p{20mm}|p{20mm}|p{20mm}|} \hline
\textbf{Archivo ARFF}    & \textbf{N\'um. Instancias} & \textbf{N\'um. Atributos} & \textbf{Tam. ARFF (KB)} &
\textbf{Tam. Tariy (KB)}\\ \hline
mushroom        & 8124              & 23               & 726.30         & 133.81\\ \hline
titanic         & 2201              &  4               &  64.60         &   0.17\\ \hline
tictactoe       &  958              & 10               &  26.50         &  14.00\\ \hline
soybean         &  683              & 36               & 194.10         &  55.46\\ \hline
vote            &  435              & 17               &  32.20         &   8.87\\ \hline
contact-lenses  &   24              &  5               &   1.10         &   0.23\\ \hline
weather.nominal &   14              &  5               &   0.57         &   0.16\\ \hline
\end{tabular}
\end{center}

\begin{figure}[h]
\centering
\includegraphics[width=0.7\textwidth]{images/formatos.png}
\caption{Rendimiento formatos de almacenamiento}
\end{figure}

\bibliographystyle{plain}
\bibliography{bibliografia}
\end{document}
