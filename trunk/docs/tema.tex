\chapter{TEMA}
\section{Titulo}
TariyKDD: Una herramienta gen\'erica para el Descubrimiento de Conocimiento en Bases de Datos, d\'ebilmente acoplada con el SGBD PostgreSQL.\\
\\
\section{L\'inea de Investigaci\'on}
El presente trabajo de grado, se encuentra inscrito bajo la \textbf{l\'inea de software y
manejo de informaci\'on}, se enmarca dentro del \'area de las Bases de Datos y
espec\'ificamente en la sub\'area  de arquitecturas de integraci\'on del Proceso de
descubrimiento en Bases de datos con Sistemas Gestores de bases de Datos.\\
\\
\section{Alcance y Delimitaci\'on}
TARIYKDD es una herramienta que contempla todas las etapas del proceso DCBD, es decir:
Selecci\'on, preprocesamiento, transformaci\'on, miner\'ia de datos y visualizaci\'on
\cite{1, 4, 16}. En la etapa de miner\'ia de datos se implementaron las tareas de
Asociaci\'on y Clasificaci\'on. En estas dos tareas, se utilizaron los operadores
algebraicos relacionales  y primitivas SQL para DCBD, desarrollados por Timar\'an
\cite{32, 35}, con los algoritmos EquipAsso \cite{33, 34, 35} y Mate-tree \cite{35}. En
la etapa de Visualizaci\'on se desarrollo una interfaz gr\'afica que le permite al
usuario interactuar de manera f\'acil con la herramienta.\\
\\
Para los algoritmos implementados, se hicieron pruebas de rendimiento, utilizando
conjuntos de datos reales y se los comparo con los algoritmos Apriori H\'ibrido \cite{4},
FP-Growth \cite{17, 19} y C.4.5 \cite{25, 26}.\\
\\
