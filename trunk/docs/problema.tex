\chapter{PROBLEMA OBJETO DE ESTUDIO}
\section{Descripci\'on del Problema}
Muchos investigadores \cite{7, 6, 17, 28} han reconocido la necesidad de integrar los sistemas de descubrimiento de conocimiento y bases de datos, haciendo de esta un \'area activa de investigaci\'on. La gran mayor\'ia de herramientas de DCBD tienen una arquitectura d\'ebilmente acoplada con un Sistema Gestor de Bases de Datos \cite{30}.\\
\\
Algunas herramientas como Alice \cite{20}, C5.0 RuleQuest \cite{27}, Qyield \cite{24},  Cover\-Story \cite{22} ofrecen soporte \'unicamente en la etapa de miner\'ia de datos y requieren un pre y un post procesamiento de los datos. Hay una gran cantidad de este tipo de herramientas \cite{21}, especialmente para clasificaci\'on apoyadas en \'arboles de decisi\'on, redes neuronales y aprendizaje basado en ejemplos. El usuario de este tipo de herramientas puede integrarlas  a otros m\'odulos como parte de una aplicaci\'on completa \cite{23}.\\
\\
Otros ofrecen soporte en  m\'as de una etapa del proceso de DCBD  y una variedad de tareas de descubrimiento, t\'ipicamente, combinando clasificaci\'on, visualizaci\'on, consulta y clustering, entre otras \cite{23}. En este grupo est\'an Clementine \cite{29}, DBMiner \cite{11, 12, 13}, DBLearn \cite{14}, Data Mine \cite{19}, IMACS \cite{5}, Intelligent Miner \cite{18}, Quest \cite{2} entre otras. Una evaluaci\'on de un gran n\'umero de herramientas de este  tipo se puede encontrar en \cite{10}.\\
\\
Todas estas herramientas necesitan de la adquisici\'on de costosas licencias para su utilizaci\'on. Este hecho limita a las peque\~nas y medianas empresas u organizaciones, al acceso de herramientas DCBD para la toma de decisiones, que inciden directamente en la obtenci\'on de mayores ganancias y en el aumento de su competitividad.\\
\\
Por esta raz\'on, se plantea el desarrollo de una herramienta gen\'erica de DCBD, d\'ebilmente acoplada, bajo software libre, que permita el acceso a este tipo de herramientas sin ning\'un tipo de restricciones, a las peque\~nas y medianas  empresas u organizaciones de nuestro pa\'is o de cualquier parte del mundo.\\
\\
\section{Formulaci\'on del Problema}
?`El desarrollo de una herramienta gen\'erica para el Descubrimiento de Conocimiento en bases de datos d\'ebilmente acoplada con el SGBD PostgreSQL bajo software libre, facilitar\'a a las peque\~nas y medianas empresas la toma de decisiones?
