\pagestyle{plain}
\setcounter{page}{5}
\chapter{INTRODUCCI\'ON}
El proceso de extraer conocimiento a partir de grandes vol\'umenes de datos ha sido reconocido por muchos
investigadores como un t\'opico de investigaci\'on clave en los sistemas de bases de datos, y por muchas compa\~n\'ias industriales como una importante \'area y una oportunidad para obtener mayo\-res ganancias \cite{31}.\\
\\
El Descubrimiento de Conocimiento en Bases de Datos (DCBD) es b\'asicamente un proceso autom\'atico en el que se combinan descubrimiento y an\'alisis. El proceso consiste en extraer patrones en forma de reglas o funciones, a partir de los datos, para que el usuario los analice. Esta tarea implica generalmente preprocesar los datos, hacer miner\'ia de datos (data mining) y presentar resultados \cite{1,4,7,19,23}. El DCBD se puede aplicar en diferentes dominios por ejemplo, para determinar perfiles de clientes fraudulentos (evasi\'on de impuestos), para descubrir relaciones impl\'icitas existentes entre s\'intomas y enfermedades, entre caracter\'isticas t\'ecnicas y diagn\'ostico del estado de equipos y m\'aquinas, para determinar perfiles de estudiantes ''acad\'emicamente exitosos'' en t\'erminos de sus caracter\'isticas socioecon\'omicas, para determinar patrones de compra de los clientes en sus canastas de mercado, entre otras.\\
\\
Las investigaciones en DCBD, se centraron inicialmente en definir nuevas operaciones de descubrimiento de patrones y desarrollar algoritmos para estas. Investigaciones posteriores se han focalizado en el problema de integrar DCBD con Sistemas Gestores de  Bases de Datos (SGBD) ofreciendo como resultado el desarrollo de herramientas  DCBD cuyas arquitecturas se pueden clasificar en una de tres categor\'ias: d\'ebilmente acopladas, medianamente acopladas y fuertemente acopladas con el SGBD \cite{30}.\\
\\
Una herramienta DCBD debe integrar una variedad de componentes (t\'ecnicas de miner\'ia de datos, consultas, m\'etodos de visualizaci\'on, interfaces, etc.), que juntos puedan eficientemente identificar y extraer patrones interesantes y \'utiles de los datos almacenados en las bases de datos.  De acuerdo a las tareas que desarrollen, las herramientas DCBD se clasifican en tres grupos: herramientas gen\'ericas de tareas sencillas, herramientas gen\'ericas de tareas m\'ultiples y herramientas de dominio espec\'ifico \cite{23}.\\
\\
En este documento se presenta el trabajo de grado para optar por el t\'itulo de Ingeniero de Sistemas. Fruto de la presente investigaci\'on es el desarrollo de ''TariyKDD: Una herramienta gen\'erica de Descubrimiento de Conocimiento en Bases de Datos d\'ebilmente acoplada con el SGBD PostgreSQL'',  en la cual tambi\'en se implementaron los algoritmos EquipAsso \cite{33, 34, 35} y MateTree \cite{35} para las tareas de Asociaci\'on y Clasificaci\'on propuestos por Timar\'an en \cite{35} y sobre los cuales se realizar\'on ciertas pruebas para medir su rendimiento.\\
\\
El resto de este documento esta organizado de la siguiente manera. En la siguiente secci\'on se especifica el Tema de la propuesta, se lo enmarca dentro de una l\'inea de investigaci\'on y se lo delimita. A continuaci\'on se describe el problema objeto de estudio. En la secci\'on 4  se especifican los objetivos generales y espec\'ificos del anteproyecto. En la secci\'on 5 se presenta la justificaci\'on de la propuesta de trabajo de grado. En la secci\'on 6 se presenta el  estado general del arte en el \'area de integraci\'on de DCBD y SGBD. En la secci\'on 7 se desarrolla todo lo concerniente al an\'alisis orientado a objetos UML que se realizo para construir la herramienta y finalmente en la secci\'on 8 se presentan las conclusiones, recomendaciones, referencias bibliogr\'aficas y anexos.
